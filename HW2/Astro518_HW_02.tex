
% Default to the notebook output style

    


% Inherit from the specified cell style.






    
\documentclass{article}

    
        
    
    \usepackage{graphicx} % Used to insert images
    \usepackage{adjustbox} % Used to constrain images to a maximum size 
    \usepackage{color} % Allow colors to be defined
    \usepackage{enumerate} % Needed for markdown enumerations to work
    \usepackage{geometry} % Used to adjust the document margins
    \usepackage{amsmath} % Equations
    \usepackage{amssymb} % Equations
    \usepackage[mathletters]{ucs} % Extended unicode (utf-8) support
    \usepackage[utf8x]{inputenc} % Allow utf-8 characters in the tex document
    \usepackage{fancyvrb} % verbatim replacement that allows latex
    \usepackage{grffile} % extends the file name processing of package graphics 
                         % to support a larger range 
    % The hyperref package gives us a pdf with properly built
    % internal navigation ('pdf bookmarks' for the table of contents,
    % internal cross-reference links, web links for URLs, etc.)
    \usepackage{hyperref}
    \usepackage{longtable} % longtable support required by pandoc >1.10
    \usepackage{booktabs}  % table support for pandoc > 1.12.2
    

    \usepackage{tikz} % Needed to box output/input
    \usepackage{scrextend} % Used to indent output
    \usepackage{needspace} % Make prompts follow contents
    \usepackage{framed} % Used to draw output that spans multiple pages


    
    
    \definecolor{orange}{cmyk}{0,0.4,0.8,0.2}
    \definecolor{darkorange}{rgb}{.71,0.21,0.01}
    \definecolor{darkgreen}{rgb}{.12,.54,.11}
    \definecolor{myteal}{rgb}{.26, .44, .56}
    \definecolor{gray}{gray}{0.45}
    \definecolor{lightgray}{gray}{.95}
    \definecolor{mediumgray}{gray}{.8}
    \definecolor{inputbackground}{rgb}{.95, .95, .85}
    \definecolor{outputbackground}{rgb}{.95, .95, .95}
    \definecolor{traceback}{rgb}{1, .95, .95}
    % ansi colors
    \definecolor{red}{rgb}{.6,0,0}
    \definecolor{green}{rgb}{0,.65,0}
    \definecolor{brown}{rgb}{0.6,0.6,0}
    \definecolor{blue}{rgb}{0,.145,.698}
    \definecolor{purple}{rgb}{.698,.145,.698}
    \definecolor{cyan}{rgb}{0,.698,.698}
    \definecolor{lightgray}{gray}{0.5}
    
    % bright ansi colors
    \definecolor{darkgray}{gray}{0.25}
    \definecolor{lightred}{rgb}{1.0,0.39,0.28}
    \definecolor{lightgreen}{rgb}{0.48,0.99,0.0}
    \definecolor{lightblue}{rgb}{0.53,0.81,0.92}
    \definecolor{lightpurple}{rgb}{0.87,0.63,0.87}
    \definecolor{lightcyan}{rgb}{0.5,1.0,0.83}
    
    % commands and environments needed by pandoc snippets
    % extracted from the output of `pandoc -s`
    \DefineVerbatimEnvironment{Highlighting}{Verbatim}{commandchars=\\\{\}}
    % Add ',fontsize=\small' for more characters per line
    \newenvironment{Shaded}{}{}
    \newcommand{\KeywordTok}[1]{\textcolor[rgb]{0.00,0.44,0.13}{\textbf{{#1}}}}
    \newcommand{\DataTypeTok}[1]{\textcolor[rgb]{0.56,0.13,0.00}{{#1}}}
    \newcommand{\DecValTok}[1]{\textcolor[rgb]{0.25,0.63,0.44}{{#1}}}
    \newcommand{\BaseNTok}[1]{\textcolor[rgb]{0.25,0.63,0.44}{{#1}}}
    \newcommand{\FloatTok}[1]{\textcolor[rgb]{0.25,0.63,0.44}{{#1}}}
    \newcommand{\CharTok}[1]{\textcolor[rgb]{0.25,0.44,0.63}{{#1}}}
    \newcommand{\StringTok}[1]{\textcolor[rgb]{0.25,0.44,0.63}{{#1}}}
    \newcommand{\CommentTok}[1]{\textcolor[rgb]{0.38,0.63,0.69}{\textit{{#1}}}}
    \newcommand{\OtherTok}[1]{\textcolor[rgb]{0.00,0.44,0.13}{{#1}}}
    \newcommand{\AlertTok}[1]{\textcolor[rgb]{1.00,0.00,0.00}{\textbf{{#1}}}}
    \newcommand{\FunctionTok}[1]{\textcolor[rgb]{0.02,0.16,0.49}{{#1}}}
    \newcommand{\RegionMarkerTok}[1]{{#1}}
    \newcommand{\ErrorTok}[1]{\textcolor[rgb]{1.00,0.00,0.00}{\textbf{{#1}}}}
    \newcommand{\NormalTok}[1]{{#1}}
    
    % Define a nice break command that doesn't care if a line doesn't already
    % exist.
    \def\br{\hspace*{\fill} \\* }
    % Math Jax compatability definitions
    \def\gt{>}
    \def\lt{<}
    % Document parameters
    \title{Astro518\_HW\_02}
    
    
    

    % Pygments definitions
    
\makeatletter
\def\PY@reset{\let\PY@it=\relax \let\PY@bf=\relax%
    \let\PY@ul=\relax \let\PY@tc=\relax%
    \let\PY@bc=\relax \let\PY@ff=\relax}
\def\PY@tok#1{\csname PY@tok@#1\endcsname}
\def\PY@toks#1+{\ifx\relax#1\empty\else%
    \PY@tok{#1}\expandafter\PY@toks\fi}
\def\PY@do#1{\PY@bc{\PY@tc{\PY@ul{%
    \PY@it{\PY@bf{\PY@ff{#1}}}}}}}
\def\PY#1#2{\PY@reset\PY@toks#1+\relax+\PY@do{#2}}

\expandafter\def\csname PY@tok@gd\endcsname{\def\PY@tc##1{\textcolor[rgb]{0.63,0.00,0.00}{##1}}}
\expandafter\def\csname PY@tok@gu\endcsname{\let\PY@bf=\textbf\def\PY@tc##1{\textcolor[rgb]{0.50,0.00,0.50}{##1}}}
\expandafter\def\csname PY@tok@gt\endcsname{\def\PY@tc##1{\textcolor[rgb]{0.00,0.27,0.87}{##1}}}
\expandafter\def\csname PY@tok@gs\endcsname{\let\PY@bf=\textbf}
\expandafter\def\csname PY@tok@gr\endcsname{\def\PY@tc##1{\textcolor[rgb]{1.00,0.00,0.00}{##1}}}
\expandafter\def\csname PY@tok@cm\endcsname{\let\PY@it=\textit\def\PY@tc##1{\textcolor[rgb]{0.25,0.50,0.50}{##1}}}
\expandafter\def\csname PY@tok@vg\endcsname{\def\PY@tc##1{\textcolor[rgb]{0.10,0.09,0.49}{##1}}}
\expandafter\def\csname PY@tok@m\endcsname{\def\PY@tc##1{\textcolor[rgb]{0.40,0.40,0.40}{##1}}}
\expandafter\def\csname PY@tok@mh\endcsname{\def\PY@tc##1{\textcolor[rgb]{0.40,0.40,0.40}{##1}}}
\expandafter\def\csname PY@tok@go\endcsname{\def\PY@tc##1{\textcolor[rgb]{0.53,0.53,0.53}{##1}}}
\expandafter\def\csname PY@tok@ge\endcsname{\let\PY@it=\textit}
\expandafter\def\csname PY@tok@vc\endcsname{\def\PY@tc##1{\textcolor[rgb]{0.10,0.09,0.49}{##1}}}
\expandafter\def\csname PY@tok@il\endcsname{\def\PY@tc##1{\textcolor[rgb]{0.40,0.40,0.40}{##1}}}
\expandafter\def\csname PY@tok@cs\endcsname{\let\PY@it=\textit\def\PY@tc##1{\textcolor[rgb]{0.25,0.50,0.50}{##1}}}
\expandafter\def\csname PY@tok@cp\endcsname{\def\PY@tc##1{\textcolor[rgb]{0.74,0.48,0.00}{##1}}}
\expandafter\def\csname PY@tok@gi\endcsname{\def\PY@tc##1{\textcolor[rgb]{0.00,0.63,0.00}{##1}}}
\expandafter\def\csname PY@tok@gh\endcsname{\let\PY@bf=\textbf\def\PY@tc##1{\textcolor[rgb]{0.00,0.00,0.50}{##1}}}
\expandafter\def\csname PY@tok@ni\endcsname{\let\PY@bf=\textbf\def\PY@tc##1{\textcolor[rgb]{0.60,0.60,0.60}{##1}}}
\expandafter\def\csname PY@tok@nl\endcsname{\def\PY@tc##1{\textcolor[rgb]{0.63,0.63,0.00}{##1}}}
\expandafter\def\csname PY@tok@nn\endcsname{\let\PY@bf=\textbf\def\PY@tc##1{\textcolor[rgb]{0.00,0.00,1.00}{##1}}}
\expandafter\def\csname PY@tok@no\endcsname{\def\PY@tc##1{\textcolor[rgb]{0.53,0.00,0.00}{##1}}}
\expandafter\def\csname PY@tok@na\endcsname{\def\PY@tc##1{\textcolor[rgb]{0.49,0.56,0.16}{##1}}}
\expandafter\def\csname PY@tok@nb\endcsname{\def\PY@tc##1{\textcolor[rgb]{0.00,0.50,0.00}{##1}}}
\expandafter\def\csname PY@tok@nc\endcsname{\let\PY@bf=\textbf\def\PY@tc##1{\textcolor[rgb]{0.00,0.00,1.00}{##1}}}
\expandafter\def\csname PY@tok@nd\endcsname{\def\PY@tc##1{\textcolor[rgb]{0.67,0.13,1.00}{##1}}}
\expandafter\def\csname PY@tok@ne\endcsname{\let\PY@bf=\textbf\def\PY@tc##1{\textcolor[rgb]{0.82,0.25,0.23}{##1}}}
\expandafter\def\csname PY@tok@nf\endcsname{\def\PY@tc##1{\textcolor[rgb]{0.00,0.00,1.00}{##1}}}
\expandafter\def\csname PY@tok@si\endcsname{\let\PY@bf=\textbf\def\PY@tc##1{\textcolor[rgb]{0.73,0.40,0.53}{##1}}}
\expandafter\def\csname PY@tok@s2\endcsname{\def\PY@tc##1{\textcolor[rgb]{0.73,0.13,0.13}{##1}}}
\expandafter\def\csname PY@tok@vi\endcsname{\def\PY@tc##1{\textcolor[rgb]{0.10,0.09,0.49}{##1}}}
\expandafter\def\csname PY@tok@nt\endcsname{\let\PY@bf=\textbf\def\PY@tc##1{\textcolor[rgb]{0.00,0.50,0.00}{##1}}}
\expandafter\def\csname PY@tok@nv\endcsname{\def\PY@tc##1{\textcolor[rgb]{0.10,0.09,0.49}{##1}}}
\expandafter\def\csname PY@tok@s1\endcsname{\def\PY@tc##1{\textcolor[rgb]{0.73,0.13,0.13}{##1}}}
\expandafter\def\csname PY@tok@sh\endcsname{\def\PY@tc##1{\textcolor[rgb]{0.73,0.13,0.13}{##1}}}
\expandafter\def\csname PY@tok@sc\endcsname{\def\PY@tc##1{\textcolor[rgb]{0.73,0.13,0.13}{##1}}}
\expandafter\def\csname PY@tok@sx\endcsname{\def\PY@tc##1{\textcolor[rgb]{0.00,0.50,0.00}{##1}}}
\expandafter\def\csname PY@tok@bp\endcsname{\def\PY@tc##1{\textcolor[rgb]{0.00,0.50,0.00}{##1}}}
\expandafter\def\csname PY@tok@c1\endcsname{\let\PY@it=\textit\def\PY@tc##1{\textcolor[rgb]{0.25,0.50,0.50}{##1}}}
\expandafter\def\csname PY@tok@kc\endcsname{\let\PY@bf=\textbf\def\PY@tc##1{\textcolor[rgb]{0.00,0.50,0.00}{##1}}}
\expandafter\def\csname PY@tok@c\endcsname{\let\PY@it=\textit\def\PY@tc##1{\textcolor[rgb]{0.25,0.50,0.50}{##1}}}
\expandafter\def\csname PY@tok@mf\endcsname{\def\PY@tc##1{\textcolor[rgb]{0.40,0.40,0.40}{##1}}}
\expandafter\def\csname PY@tok@err\endcsname{\def\PY@bc##1{\setlength{\fboxsep}{0pt}\fcolorbox[rgb]{1.00,0.00,0.00}{1,1,1}{\strut ##1}}}
\expandafter\def\csname PY@tok@kd\endcsname{\let\PY@bf=\textbf\def\PY@tc##1{\textcolor[rgb]{0.00,0.50,0.00}{##1}}}
\expandafter\def\csname PY@tok@ss\endcsname{\def\PY@tc##1{\textcolor[rgb]{0.10,0.09,0.49}{##1}}}
\expandafter\def\csname PY@tok@sr\endcsname{\def\PY@tc##1{\textcolor[rgb]{0.73,0.40,0.53}{##1}}}
\expandafter\def\csname PY@tok@mo\endcsname{\def\PY@tc##1{\textcolor[rgb]{0.40,0.40,0.40}{##1}}}
\expandafter\def\csname PY@tok@kn\endcsname{\let\PY@bf=\textbf\def\PY@tc##1{\textcolor[rgb]{0.00,0.50,0.00}{##1}}}
\expandafter\def\csname PY@tok@mi\endcsname{\def\PY@tc##1{\textcolor[rgb]{0.40,0.40,0.40}{##1}}}
\expandafter\def\csname PY@tok@gp\endcsname{\let\PY@bf=\textbf\def\PY@tc##1{\textcolor[rgb]{0.00,0.00,0.50}{##1}}}
\expandafter\def\csname PY@tok@o\endcsname{\def\PY@tc##1{\textcolor[rgb]{0.40,0.40,0.40}{##1}}}
\expandafter\def\csname PY@tok@kr\endcsname{\let\PY@bf=\textbf\def\PY@tc##1{\textcolor[rgb]{0.00,0.50,0.00}{##1}}}
\expandafter\def\csname PY@tok@s\endcsname{\def\PY@tc##1{\textcolor[rgb]{0.73,0.13,0.13}{##1}}}
\expandafter\def\csname PY@tok@kp\endcsname{\def\PY@tc##1{\textcolor[rgb]{0.00,0.50,0.00}{##1}}}
\expandafter\def\csname PY@tok@w\endcsname{\def\PY@tc##1{\textcolor[rgb]{0.73,0.73,0.73}{##1}}}
\expandafter\def\csname PY@tok@kt\endcsname{\def\PY@tc##1{\textcolor[rgb]{0.69,0.00,0.25}{##1}}}
\expandafter\def\csname PY@tok@ow\endcsname{\let\PY@bf=\textbf\def\PY@tc##1{\textcolor[rgb]{0.67,0.13,1.00}{##1}}}
\expandafter\def\csname PY@tok@sb\endcsname{\def\PY@tc##1{\textcolor[rgb]{0.73,0.13,0.13}{##1}}}
\expandafter\def\csname PY@tok@k\endcsname{\let\PY@bf=\textbf\def\PY@tc##1{\textcolor[rgb]{0.00,0.50,0.00}{##1}}}
\expandafter\def\csname PY@tok@se\endcsname{\let\PY@bf=\textbf\def\PY@tc##1{\textcolor[rgb]{0.73,0.40,0.13}{##1}}}
\expandafter\def\csname PY@tok@sd\endcsname{\let\PY@it=\textit\def\PY@tc##1{\textcolor[rgb]{0.73,0.13,0.13}{##1}}}

\def\PYZbs{\char`\\}
\def\PYZus{\char`\_}
\def\PYZob{\char`\{}
\def\PYZcb{\char`\}}
\def\PYZca{\char`\^}
\def\PYZam{\char`\&}
\def\PYZlt{\char`\<}
\def\PYZgt{\char`\>}
\def\PYZsh{\char`\#}
\def\PYZpc{\char`\%}
\def\PYZdl{\char`\$}
\def\PYZhy{\char`\-}
\def\PYZsq{\char`\'}
\def\PYZdq{\char`\"}
\def\PYZti{\char`\~}
% for compatibility with earlier versions
\def\PYZat{@}
\def\PYZlb{[}
\def\PYZrb{]}
\makeatother


    % NB prompt colors
    \definecolor{nbframe-border}{rgb}{0.867,0.867,0.867}
    \definecolor{nbframe-bg}{rgb}{0.969,0.969,0.969}
    \definecolor{nbframe-in-prompt}{rgb}{0.0,0.0,0.502}
    \definecolor{nbframe-out-prompt}{rgb}{0.545,0.0,0.0}

    % NB prompt lengths
    \newlength{\inputpadding}
    \setlength{\inputpadding}{0.5em}
    \newlength{\cellleftmargin}
    \setlength{\cellleftmargin}{0.15\linewidth}
    \newlength{\borderthickness}
    \setlength{\borderthickness}{0.4pt}
    \newlength{\smallerfontscale}
    \setlength{\smallerfontscale}{9.5pt}

    % NB prompt font size
    \def\smaller{\fontsize{\smallerfontscale}{\smallerfontscale}\selectfont}

    % Define a background layer, in which the nb prompt shape is drawn
    \pgfdeclarelayer{background}
    \pgfsetlayers{background,main}
    \usetikzlibrary{calc}

    % define styles for the normal border and the torn border
    \tikzset{
      normal border/.style={draw=nbframe-border, fill=nbframe-bg,
        rectangle, rounded corners=2.5pt, line width=\borderthickness},
      torn border/.style={draw=white, fill=white, line width=\borderthickness}}

    % Macro to draw the shape behind the text, when it fits completly in the
    % page
    \def\notebookcellframe#1{%
    \tikz{%
      \node[inner sep=\inputpadding] (A) {#1};% Draw the text of the node
      \begin{pgfonlayer}{background}% Draw the shape behind
      \fill[normal border]%
            (A.south east) -- ($(A.south west)+(\cellleftmargin,0)$) -- 
            ($(A.north west)+(\cellleftmargin,0)$) -- (A.north east) -- cycle;
      \end{pgfonlayer}}}%

    % Macro to draw the shape, when the text will continue in next page
    \def\notebookcellframetop#1{%
    \tikz{%
      \node[inner sep=\inputpadding] (A) {#1};    % Draw the text of the node
      \begin{pgfonlayer}{background}    
      \fill[normal border]              % Draw the ``complete shape'' behind
            (A.south east) -- ($(A.south west)+(\cellleftmargin,0)$) -- 
            ($(A.north west)+(\cellleftmargin,0)$) -- (A.north east) -- cycle;
      \fill[torn border]                % Add the torn lower border
            ($(A.south east)-(0,.1)$) -- ($(A.south west)+(\cellleftmargin,-.1)$) -- 
            ($(A.south west)+(\cellleftmargin,.1)$) -- ($(A.south east)+(0,.1)$) -- cycle;
      \end{pgfonlayer}}}

    % Macro to draw the shape, when the text continues from previous page
    \def\notebookcellframebottom#1{%
    \tikz{%
      \node[inner sep=\inputpadding] (A) {#1};   % Draw the text of the node
      \begin{pgfonlayer}{background}   
      \fill[normal border]             % Draw the ``complete shape'' behind
            (A.south east) -- ($(A.south west)+(\cellleftmargin,0)$) -- 
            ($(A.north west)+(\cellleftmargin,0)$) -- (A.north east) -- cycle;
      \fill[torn border]               % Add the torn upper border
            ($(A.north east)-(0,.1)$) -- ($(A.north west)+(\cellleftmargin,-.1)$) -- 
            ($(A.north west)+(\cellleftmargin,.1)$) -- ($(A.north east)+(0,.1)$) -- cycle;
      \end{pgfonlayer}}}

    % Macro to draw the shape, when both the text continues from previous page
    % and it will continue in next page
    \def\notebookcellframemiddle#1{%
    \tikz{%
      \node[inner sep=\inputpadding] (A) {#1};   % Draw the text of the node
      \begin{pgfonlayer}{background}   
      \fill[normal border]             % Draw the ``complete shape'' behind
            (A.south east) -- ($(A.south west)+(\cellleftmargin,0)$) -- 
            ($(A.north west)+(\cellleftmargin,0)$) -- (A.north east) -- cycle;
      \fill[torn border]               % Add the torn lower border
            ($(A.south east)-(0,.1)$) -- ($(A.south west)+(\cellleftmargin,-.1)$) -- 
            ($(A.south west)+(\cellleftmargin,.1)$) -- ($(A.south east)+(0,.1)$) -- cycle;
      \fill[torn border]               % Add the torn upper border
            ($(A.north east)-(0,.1)$) -- ($(A.north west)+(\cellleftmargin,-.1)$) -- 
            ($(A.north west)+(\cellleftmargin,.1)$) -- ($(A.north east)+(0,.1)$) -- cycle;
      \end{pgfonlayer}}}

    % Define the environment which puts the frame
    % In this case, the environment also accepts an argument with an optional
    % title (which defaults to ``Example'', which is typeset in a box overlaid
    % on the top border
    \newenvironment{notebookcell}[1][0]{%
      \def\FrameCommand{\notebookcellframe}%
      \def\FirstFrameCommand{\notebookcellframetop}%
      \def\LastFrameCommand{\notebookcellframebottom}%
      \def\MidFrameCommand{\notebookcellframemiddle}%
      \par\vspace{1\baselineskip}%
      \MakeFramed {\FrameRestore}%
      \noindent\tikz\node[inner sep=0em] at ($(A.north west)-(0,0)$) {%
      \begin{minipage}{\cellleftmargin}%
    \hfill%
    {\smaller%
    \tt%
    \color{nbframe-in-prompt}%
    In[#1]:}%
    \hspace{\inputpadding}%
    \hspace{2pt}%
    \hspace{3pt}%
    \end{minipage}%%
      }; \par}%
    {\endMakeFramed}



    
    % Prevent overflowing lines due to hard-to-break entities
    \sloppy 
    % Setup hyperref package
    \hypersetup{
      breaklinks=true,  % so long urls are correctly broken across lines
      colorlinks=true,
      urlcolor=blue,
      linkcolor=darkorange,
      citecolor=darkgreen,
      }
    % Slightly bigger margins than the latex defaults
    
    \geometry{verbose,tmargin=1in,bmargin=1in,lmargin=1in,rmargin=1in}
    
    

    \begin{document}
    
    
    \maketitle
    
    

    
    % Add contents below.

{\par%
\vspace{-1\baselineskip}%
\needspace{4\baselineskip}}%
\begin{notebookcell}[3]%
\begin{addmargin}[\cellleftmargin]{0em}% left, right
{\smaller%
\par%
%
\vspace{-1\smallerfontscale}%
\begin{Verbatim}[commandchars=\\\{\}]
\PY{k+kn}{import} \PY{n+nn}{numpy} \PY{k+kn}{as} \PY{n+nn}{np}
\PY{k+kn}{import} \PY{n+nn}{matplotlib.pyplot} \PY{k+kn}{as} \PY{n+nn}{plt}
\PY{k+kn}{import} \PY{n+nn}{prettyplotlib} \PY{k+kn}{as} \PY{n+nn}{ppl}
\end{Verbatim}
%
\par%
\vspace{-1\smallerfontscale}}%
\end{addmargin}
\end{notebookcell}


    \section{Problem 1, Linear Fit}\label{problem-1-linear-fit}

\subsection{Ignoring the first four
points}\label{ignoring-the-first-four-points}

\begin{itemize}
\itemsep1pt\parskip0pt\parsep0pt
\item
  linear fit to function $y=mx+b$ be done with the function:
\end{itemize}

\begin{equation}
\begin{bmatrix}b\\m\end{bmatrix} = [A^{T}C^{-1}A]^{-1}[A^{T}C^{-1}Y]
\end{equation}

in which

\begin{align}
& A = \begin{bmatrix}
1&x_1\\
2&x_2\\
\vdots&\vdots\\
n&x_n\\
\end{bmatrix}\\
& C = \begin{bmatrix}
\frac{1}{\sigma_1^2}&0&\cdots&0\\
0&\frac{1}{\sigma_2^2}&\cdots&0\\
\vdots&\vdots&\ddots&\vdots\\
0&0&\cdots&\frac{1}{\sigma_n^2}
\end{bmatrix}
\end{align}

    % Add contents below.

{\par%
\vspace{-1\baselineskip}%
\needspace{4\baselineskip}}%
\begin{notebookcell}[4]%
\begin{addmargin}[\cellleftmargin]{0em}% left, right
{\smaller%
\par%
%
\vspace{-1\smallerfontscale}%
\begin{Verbatim}[commandchars=\\\{\}]
\PY{c}{\PYZsh{}\PYZsh{} read data}
\PY{p}{(}\PY{n}{data\PYZus{}id}\PY{p}{,} \PY{n}{x}\PY{p}{,} \PY{n}{y}\PY{p}{,} \PY{n}{dy}\PY{p}{,} \PY{n}{dx}\PY{p}{,} \PY{n}{rho\PYZus{}xy}\PY{p}{)} \PY{o}{=} \PY{n}{np}\PY{o}{.}\PY{n}{loadtxt}\PY{p}{(}\PY{l+s}{\PYZsq{}}\PY{l+s}{sample\PYZus{}data.txt}\PY{l+s}{\PYZsq{}}\PY{p}{,} \PY{n}{comments} \PY{o}{=} \PY{l+s}{\PYZsq{}}\PY{l+s}{\PYZpc{}}\PY{l+s}{\PYZsq{}}\PY{p}{,} \PY{n}{unpack} \PY{o}{=} \PY{n+nb+bp}{True}\PY{p}{)} 
\end{Verbatim}
%
\par%
\vspace{-1\smallerfontscale}}%
\end{addmargin}
\end{notebookcell}


    % Add contents below.

{\par%
\vspace{-1\baselineskip}%
\needspace{4\baselineskip}}%
\begin{notebookcell}[5]%
\begin{addmargin}[\cellleftmargin]{0em}% left, right
{\smaller%
\par%
%
\vspace{-1\smallerfontscale}%
\begin{Verbatim}[commandchars=\\\{\}]
\PY{c}{\PYZsh{}\PYZsh{} linear fitting using the method in Hogg et. al.}
\PY{c}{\PYZsh{}\PYZsh{} \PYZsq{}\PYZus{}1\PYZsq{} indicates the removal of first 4 lines of the orginal data}

\PY{n}{x\PYZus{}1} \PY{o}{=} \PY{n}{x}\PY{p}{[}\PY{l+m+mi}{4}\PY{p}{:}\PY{p}{]}
\PY{n}{y\PYZus{}1} \PY{o}{=} \PY{n}{y}\PY{p}{[}\PY{l+m+mi}{4}\PY{p}{:}\PY{p}{]}
\PY{n}{dy\PYZus{}1} \PY{o}{=} \PY{n}{dy}\PY{p}{[}\PY{l+m+mi}{4}\PY{p}{:}\PY{p}{]}

\PY{c}{\PYZsh{} Different from matlab, numpy array is a different data type from matrix.}
\PY{c}{\PYZsh{} * operator in numpy is similar to .* operator in matlab}
\PY{c}{\PYZsh{} to carry out matrix multiply, the numpy array has to be converted to matrix using function np.mat()}

\PY{n}{Y\PYZus{}vec\PYZus{}1} \PY{o}{=} \PY{n}{np}\PY{o}{.}\PY{n}{mat}\PY{p}{(}\PY{n}{y\PYZus{}1}\PY{p}{)}\PY{o}{.}\PY{n}{T} 
\PY{n}{A\PYZus{}1} \PY{o}{=} \PY{n}{np}\PY{o}{.}\PY{n}{mat}\PY{p}{(}\PY{p}{[}\PY{n}{np}\PY{o}{.}\PY{n}{ones}\PY{p}{(}\PY{n+nb}{len}\PY{p}{(}\PY{n}{x\PYZus{}1}\PY{p}{)}\PY{p}{)}\PY{p}{,} \PY{n}{x\PYZus{}1}\PY{p}{]}\PY{p}{)}\PY{o}{.}\PY{n}{T}
\PY{n}{C\PYZus{}1} \PY{o}{=} \PY{n}{np}\PY{o}{.}\PY{n}{mat}\PY{p}{(}\PY{n}{np}\PY{o}{.}\PY{n}{diagflat}\PY{p}{(}\PY{n}{dy\PYZus{}1}\PY{o}{*}\PY{o}{*}\PY{l+m+mi}{2}\PY{p}{)}\PY{p}{)}


\PY{c}{\PYZsh{}\PYZsh{} calculate b and m}
\PY{p}{(}\PY{n}{b\PYZus{}1}\PY{p}{,} \PY{n}{m\PYZus{}1}\PY{p}{)} \PY{o}{=} \PY{p}{(}\PY{n}{A\PYZus{}1}\PY{o}{.}\PY{n}{T} \PY{o}{*}\PY{n}{C\PYZus{}1}\PY{o}{*}\PY{o}{*}\PY{p}{(}\PY{o}{\PYZhy{}}\PY{l+m+mi}{1}\PY{p}{)} \PY{o}{*} \PY{n}{A\PYZus{}1}\PY{p}{)}\PY{o}{*}\PY{o}{*}\PY{p}{(}\PY{o}{\PYZhy{}}\PY{l+m+mi}{1}\PY{p}{)} \PY{o}{*} \PY{p}{(}\PY{n}{A\PYZus{}1}\PY{o}{.}\PY{n}{T} \PY{o}{*} \PY{n}{C\PYZus{}1}\PY{o}{*}\PY{o}{*}\PY{p}{(}\PY{o}{\PYZhy{}}\PY{l+m+mi}{1}\PY{p}{)} \PY{o}{*} \PY{n}{Y\PYZus{}vec\PYZus{}1}\PY{p}{)}

\PY{n}{b\PYZus{}1} \PY{o}{=} \PY{n}{b\PYZus{}1}\PY{o}{.}\PY{n}{flat}\PY{p}{[}\PY{l+m+mi}{0}\PY{p}{]}
\PY{n}{m\PYZus{}1} \PY{o}{=} \PY{n}{m\PYZus{}1}\PY{o}{.}\PY{n}{flat}\PY{p}{[}\PY{l+m+mi}{0}\PY{p}{]} \PY{c}{\PYZsh{} convert matrix to number}

\PY{k}{print} \PY{l+s}{\PYZsq{}}\PY{l+s}{Fitting result for y = mx + b}\PY{l+s}{\PYZsq{}}
\PY{k}{print} \PY{l+s}{\PYZsq{}}\PY{l+s}{m = \PYZob{}0:.2f\PYZcb{}}\PY{l+s}{\PYZsq{}}\PY{o}{.}\PY{n}{format}\PY{p}{(}\PY{n}{m\PYZus{}1}\PY{p}{)}
\PY{k}{print} \PY{l+s}{\PYZsq{}}\PY{l+s}{b = \PYZob{}0:.2f\PYZcb{}}\PY{l+s}{\PYZsq{}}\PY{o}{.}\PY{n}{format}\PY{p}{(}\PY{n}{b\PYZus{}1}\PY{p}{)}
\end{Verbatim}
%
\par%
\vspace{-1\smallerfontscale}}%
\end{addmargin}
\end{notebookcell}

\par\vspace{1\smallerfontscale}%
    \needspace{4\baselineskip}%
    % Only render the prompt if the cell is pyout.  Note, the outputs prompt 
    % block isn't used since we need to check each indiviual output and only
    % add prompts to the pyout ones.
    %
    %
    \begin{addmargin}[\cellleftmargin]{0em}% left, right
    {\smaller%
    \vspace{-1\smallerfontscale}%
    
    \begin{Verbatim}[commandchars=\\\{\}]
Fitting result for y = mx + b
m = 2.24
b = 34.05
    \end{Verbatim}
}%
    \end{addmargin}%
    \begin{itemize}
\itemsep1pt\parskip0pt\parsep0pt
\item
  The uncertainties of fitting parameter could be calculated with
  following equation

  \begin{equation}
  \begin{bmatrix}
  \sigma_b^2&\sigma_{mb}\\
  \sigma_{bm}&\sigma_m^2
  \end{bmatrix}
  = [A^TC^{-1}A]^{-1}
  \end{equation}
\end{itemize}

    % Add contents below.

{\par%
\vspace{-1\baselineskip}%
\needspace{4\baselineskip}}%
\begin{notebookcell}[96]%
\begin{addmargin}[\cellleftmargin]{0em}% left, right
{\smaller%
\par%
%
\vspace{-1\smallerfontscale}%
\begin{Verbatim}[commandchars=\\\{\}]
\PY{n}{var\PYZus{}b\PYZus{}1}\PY{p}{,} \PY{n}{var\PYZus{}m\PYZus{}1} \PY{o}{=} \PY{n}{np}\PY{o}{.}\PY{n}{diag}\PY{p}{(}\PY{p}{(}\PY{n}{A\PYZus{}1}\PY{o}{.}\PY{n}{T} \PY{o}{*} \PY{n}{C\PYZus{}1}\PY{o}{*}\PY{o}{*}\PY{p}{(}\PY{o}{\PYZhy{}}\PY{l+m+mi}{1}\PY{p}{)} \PY{o}{*} \PY{n}{A\PYZus{}1}\PY{p}{)}\PY{o}{*}\PY{o}{*}\PY{p}{(}\PY{o}{\PYZhy{}}\PY{l+m+mi}{1}\PY{p}{)}\PY{p}{)}
\PY{n}{dm\PYZus{}1} \PY{o}{=} \PY{n}{np}\PY{o}{.}\PY{n}{sqrt}\PY{p}{(}\PY{n}{var\PYZus{}m\PYZus{}1}\PY{p}{)}
\PY{n}{db\PYZus{}1} \PY{o}{=} \PY{n}{np}\PY{o}{.}\PY{n}{sqrt}\PY{p}{(}\PY{n}{var\PYZus{}b\PYZus{}1}\PY{p}{)}

\PY{k}{print} \PY{l+s}{\PYZsq{}}\PY{l+s}{The uncertainties:}\PY{l+s}{\PYZsq{}}
\PY{k}{print} \PY{l+s}{\PYZsq{}}\PY{l+s}{sigma\PYZus{}m = \PYZob{}0:.2f\PYZcb{}}\PY{l+s}{\PYZsq{}}\PY{o}{.}\PY{n}{format}\PY{p}{(}\PY{n}{dm\PYZus{}1}\PY{p}{)}
\PY{k}{print} \PY{l+s}{\PYZsq{}}\PY{l+s}{sigma\PYZus{}b = \PYZob{}0:.2f\PYZcb{}}\PY{l+s}{\PYZsq{}}\PY{o}{.}\PY{n}{format}\PY{p}{(}\PY{n}{db\PYZus{}1}\PY{p}{)}
\end{Verbatim}
%
\par%
\vspace{-1\smallerfontscale}}%
\end{addmargin}
\end{notebookcell}

\par\vspace{1\smallerfontscale}%
    \needspace{4\baselineskip}%
    % Only render the prompt if the cell is pyout.  Note, the outputs prompt 
    % block isn't used since we need to check each indiviual output and only
    % add prompts to the pyout ones.
    %
    %
    \begin{addmargin}[\cellleftmargin]{0em}% left, right
    {\smaller%
    \vspace{-1\smallerfontscale}%
    
    \begin{Verbatim}[commandchars=\\\{\}]
The uncertainties:
sigma\_m = 0.11
sigma\_b = 18.25
    \end{Verbatim}
}%
    \end{addmargin}%
    \begin{itemize}
\itemsep1pt\parskip0pt\parsep0pt
\item
  plot the result
\end{itemize}

    % Add contents below.

{\par%
\vspace{-1\baselineskip}%
\needspace{4\baselineskip}}%
\begin{notebookcell}[7]%
\begin{addmargin}[\cellleftmargin]{0em}% left, right
{\smaller%
\par%
%
\vspace{-1\smallerfontscale}%
\begin{Verbatim}[commandchars=\\\{\}]
\PY{n}{fig} \PY{o}{=} \PY{n}{plt}\PY{o}{.}\PY{n}{figure}\PY{p}{(}\PY{p}{)}
\PY{n}{ax} \PY{o}{=} \PY{n}{fig}\PY{o}{.}\PY{n}{add\PYZus{}subplot}\PY{p}{(}\PY{l+m+mi}{111}\PY{p}{)}
\PY{n}{ax}\PY{o}{.}\PY{n}{errorbar}\PY{p}{(}\PY{n}{x\PYZus{}1}\PY{p}{,} \PY{n}{y\PYZus{}1}\PY{p}{,} \PY{n}{yerr} \PY{o}{=} \PY{n}{dy\PYZus{}1}\PY{p}{,} \PY{n}{fmt} \PY{o}{=} \PY{l+s}{\PYZsq{}}\PY{l+s}{o}\PY{l+s}{\PYZsq{}}\PY{p}{,} \PY{n}{mew} \PY{o}{=} \PY{l+m+mf}{0.6}\PY{p}{)}
\PY{n}{x\PYZus{}sample} \PY{o}{=} \PY{n}{np}\PY{o}{.}\PY{n}{linspace}\PY{p}{(}\PY{n+nb}{min}\PY{p}{(}\PY{n}{x}\PY{p}{)}\PY{p}{,} \PY{n+nb}{max}\PY{p}{(}\PY{n}{x}\PY{p}{)}\PY{p}{,} \PY{l+m+mi}{100}\PY{p}{)}
\PY{n}{ax}\PY{o}{.}\PY{n}{plot}\PY{p}{(}\PY{n}{x\PYZus{}sample}\PY{p}{,} \PY{n}{x\PYZus{}sample} \PY{o}{*} \PY{n}{m\PYZus{}1} \PY{o}{+} \PY{n}{b\PYZus{}1}\PY{p}{)}
\PY{n}{ax}\PY{o}{.}\PY{n}{set\PYZus{}title}\PY{p}{(}\PY{l+s}{\PYZsq{}}\PY{l+s}{Linear Fit}\PY{l+s}{\PYZsq{}}\PY{p}{)}
\PY{n}{ax}\PY{o}{.}\PY{n}{set\PYZus{}xlabel}\PY{p}{(}\PY{l+s}{\PYZsq{}}\PY{l+s}{\PYZdl{}x\PYZdl{}}\PY{l+s}{\PYZsq{}}\PY{p}{)}
\PY{n}{ax}\PY{o}{.}\PY{n}{set\PYZus{}ylabel}\PY{p}{(}\PY{l+s}{\PYZsq{}}\PY{l+s}{\PYZdl{}y\PYZdl{}}\PY{l+s}{\PYZsq{}}\PY{p}{)}
\PY{n}{ax}\PY{o}{.}\PY{n}{text}\PY{p}{(}\PY{l+m+mi}{30}\PY{p}{,} \PY{l+m+mi}{400}\PY{p}{,} \PY{l+s}{\PYZsq{}}\PY{l+s}{\PYZdl{}m = 2.24}\PY{l+s}{\PYZbs{}}\PY{l+s}{pm 0.11\PYZdl{}}\PY{l+s+se}{\PYZbs{}n}\PY{l+s}{\PYZdl{}b = 34 }\PY{l+s}{\PYZbs{}}\PY{l+s}{pm 18\PYZdl{}}\PY{l+s}{\PYZsq{}}\PY{p}{)}
\end{Verbatim}
%
\par%
\vspace{-1\smallerfontscale}}%
\end{addmargin}
\end{notebookcell}

\par\vspace{1\smallerfontscale}%
    \needspace{4\baselineskip}%
    % Only render the prompt if the cell is pyout.  Note, the outputs prompt 
    % block isn't used since we need to check each indiviual output and only
    % add prompts to the pyout ones.
    
        {\par%
        \vspace{-1\smallerfontscale}%
        \noindent%
        \begin{minipage}{\cellleftmargin}%
    \hfill%
    {\smaller%
    \tt%
    \color{nbframe-out-prompt}%
    Out[7]:}%
    \hspace{\inputpadding}%
    \hspace{0em}%
    \hspace{3pt}%
    \end{minipage}%%
        }%
    %
    %
    \begin{addmargin}[\cellleftmargin]{0em}% left, right
    {\smaller%
    \vspace{-1\smallerfontscale}%
    
    
    
    \begin{verbatim}
<matplotlib.text.Text at 0x10f84e990>
    \end{verbatim}

    
}%
    \end{addmargin}%\par\vspace{1\smallerfontscale}%
    \needspace{4\baselineskip}%
    % Only render the prompt if the cell is pyout.  Note, the outputs prompt 
    % block isn't used since we need to check each indiviual output and only
    % add prompts to the pyout ones.
    %
    %
    \begin{addmargin}[\cellleftmargin]{0em}% left, right
    {\smaller%
    \vspace{-1\smallerfontscale}%
    
    \begin{center}
    \adjustimage{max size={0.9\linewidth}{0.9\paperheight}}{Astro518_HW_02_files/Astro518_HW_02_7_1.png}
    \end{center}
    { \hspace*{\fill} \\}
    }%
    \end{addmargin}%
    \subsection{Include the first 4
points}\label{include-the-first-4-points}

    % Add contents below.

{\par%
\vspace{-1\baselineskip}%
\needspace{4\baselineskip}}%
\begin{notebookcell}[8]%
\begin{addmargin}[\cellleftmargin]{0em}% left, right
{\smaller%
\par%
%
\vspace{-1\smallerfontscale}%
\begin{Verbatim}[commandchars=\\\{\}]
\PY{n}{Y\PYZus{}vec} \PY{o}{=} \PY{n}{np}\PY{o}{.}\PY{n}{mat}\PY{p}{(}\PY{n}{y}\PY{p}{)}\PY{o}{.}\PY{n}{T} \PY{c}{\PYZsh{}\PYZsh{} turn the array into matrix}
\PY{n}{A} \PY{o}{=} \PY{n}{np}\PY{o}{.}\PY{n}{mat}\PY{p}{(}\PY{p}{[}\PY{n}{np}\PY{o}{.}\PY{n}{ones}\PY{p}{(}\PY{n+nb}{len}\PY{p}{(}\PY{n}{x}\PY{p}{)}\PY{p}{)}\PY{p}{,} \PY{n}{x}\PY{p}{]}\PY{p}{)}\PY{o}{.}\PY{n}{T}
\PY{n}{C} \PY{o}{=} \PY{n}{np}\PY{o}{.}\PY{n}{mat}\PY{p}{(}\PY{n}{np}\PY{o}{.}\PY{n}{diagflat}\PY{p}{(}\PY{n}{dy}\PY{o}{*}\PY{o}{*}\PY{l+m+mi}{2}\PY{p}{)}\PY{p}{)}

\PY{p}{(}\PY{n}{b}\PY{p}{,} \PY{n}{m}\PY{p}{)} \PY{o}{=} \PY{p}{(}\PY{n}{A}\PY{o}{.}\PY{n}{T} \PY{o}{*}\PY{n}{C}\PY{o}{*}\PY{o}{*}\PY{p}{(}\PY{o}{\PYZhy{}}\PY{l+m+mi}{1}\PY{p}{)} \PY{o}{*} \PY{n}{A}\PY{p}{)}\PY{o}{*}\PY{o}{*}\PY{p}{(}\PY{o}{\PYZhy{}}\PY{l+m+mi}{1}\PY{p}{)} \PY{o}{*} \PY{p}{(}\PY{n}{A}\PY{o}{.}\PY{n}{T} \PY{o}{*} \PY{n}{C}\PY{o}{*}\PY{o}{*}\PY{p}{(}\PY{o}{\PYZhy{}}\PY{l+m+mi}{1}\PY{p}{)} \PY{o}{*} \PY{n}{Y\PYZus{}vec}\PY{p}{)}

\PY{n}{b} \PY{o}{=} \PY{n}{b}\PY{o}{.}\PY{n}{flat}\PY{p}{[}\PY{l+m+mi}{0}\PY{p}{]}
\PY{n}{m} \PY{o}{=} \PY{n}{m}\PY{o}{.}\PY{n}{flat}\PY{p}{[}\PY{l+m+mi}{0}\PY{p}{]}

\PY{k}{print} \PY{l+s}{\PYZsq{}}\PY{l+s}{After including the first 4 points, fitting result for y = mx + b}\PY{l+s}{\PYZsq{}}
\PY{k}{print} \PY{l+s}{\PYZsq{}}\PY{l+s}{m = \PYZob{}0:.2f\PYZcb{}}\PY{l+s}{\PYZsq{}}\PY{o}{.}\PY{n}{format}\PY{p}{(}\PY{n}{m}\PY{p}{)}
\PY{k}{print} \PY{l+s}{\PYZsq{}}\PY{l+s}{b = \PYZob{}0:.2f\PYZcb{}}\PY{l+s}{\PYZsq{}}\PY{o}{.}\PY{n}{format}\PY{p}{(}\PY{n}{b}\PY{p}{)}

\PY{c}{\PYZsh{}\PYZsh{} and the uncertainties}

\PY{n}{var\PYZus{}b}\PY{p}{,} \PY{n}{var\PYZus{}m} \PY{o}{=} \PY{n}{np}\PY{o}{.}\PY{n}{diag}\PY{p}{(}\PY{p}{(}\PY{n}{A}\PY{o}{.}\PY{n}{T} \PY{o}{*} \PY{n}{C}\PY{o}{*}\PY{o}{*}\PY{p}{(}\PY{o}{\PYZhy{}}\PY{l+m+mi}{1}\PY{p}{)} \PY{o}{*} \PY{n}{A}\PY{p}{)}\PY{o}{*}\PY{o}{*}\PY{p}{(}\PY{o}{\PYZhy{}}\PY{l+m+mi}{1}\PY{p}{)}\PY{p}{)}
\PY{n}{dm} \PY{o}{=} \PY{n}{np}\PY{o}{.}\PY{n}{sqrt}\PY{p}{(}\PY{n}{var\PYZus{}m}\PY{p}{)}
\PY{n}{db} \PY{o}{=} \PY{n}{np}\PY{o}{.}\PY{n}{sqrt}\PY{p}{(}\PY{n}{var\PYZus{}b}\PY{p}{)}

\PY{k}{print} \PY{l+s}{\PYZsq{}}\PY{l+s}{And the uncertainties are:}\PY{l+s}{\PYZsq{}}
\PY{k}{print} \PY{l+s}{\PYZsq{}}\PY{l+s}{sigma\PYZus{}m = \PYZob{}0:.2f\PYZcb{}}\PY{l+s}{\PYZsq{}}\PY{o}{.}\PY{n}{format}\PY{p}{(}\PY{n}{dm}\PY{p}{)}
\PY{k}{print} \PY{l+s}{\PYZsq{}}\PY{l+s}{sigma\PYZus{}b = \PYZob{}0:.2f\PYZcb{}}\PY{l+s}{\PYZsq{}}\PY{o}{.}\PY{n}{format}\PY{p}{(}\PY{n}{db}\PY{p}{)}
\end{Verbatim}
%
\par%
\vspace{-1\smallerfontscale}}%
\end{addmargin}
\end{notebookcell}

\par\vspace{1\smallerfontscale}%
    \needspace{4\baselineskip}%
    % Only render the prompt if the cell is pyout.  Note, the outputs prompt 
    % block isn't used since we need to check each indiviual output and only
    % add prompts to the pyout ones.
    %
    %
    \begin{addmargin}[\cellleftmargin]{0em}% left, right
    {\smaller%
    \vspace{-1\smallerfontscale}%
    
    \begin{Verbatim}[commandchars=\\\{\}]
After including the first 4 points, fitting result for y = mx + b
m = 1.08
b = 213.27
And the uncertainties are:
sigma\_m = 0.08
sigma\_b = 14.39
    \end{Verbatim}
}%
    \end{addmargin}%
    % Add contents below.

{\par%
\vspace{-1\baselineskip}%
\needspace{4\baselineskip}}%
\begin{notebookcell}[9]%
\begin{addmargin}[\cellleftmargin]{0em}% left, right
{\smaller%
\par%
%
\vspace{-1\smallerfontscale}%
\begin{Verbatim}[commandchars=\\\{\}]
\PY{n}{fig} \PY{o}{=} \PY{n}{plt}\PY{o}{.}\PY{n}{figure}\PY{p}{(}\PY{p}{)}
\PY{n}{ax} \PY{o}{=} \PY{n}{fig}\PY{o}{.}\PY{n}{add\PYZus{}subplot}\PY{p}{(}\PY{l+m+mi}{111}\PY{p}{)}
\PY{n}{ax}\PY{o}{.}\PY{n}{errorbar}\PY{p}{(}\PY{n}{x}\PY{p}{,} \PY{n}{y}\PY{p}{,} \PY{n}{yerr} \PY{o}{=} \PY{n}{dy}\PY{p}{,} \PY{n}{fmt} \PY{o}{=} \PY{l+s}{\PYZsq{}}\PY{l+s}{o}\PY{l+s}{\PYZsq{}}\PY{p}{,} \PY{n}{mew} \PY{o}{=} \PY{l+m+mf}{0.6}\PY{p}{)}
\PY{n}{ax}\PY{o}{.}\PY{n}{plot}\PY{p}{(}\PY{n}{x\PYZus{}sample}\PY{p}{,} \PY{n}{x\PYZus{}sample} \PY{o}{*} \PY{n}{m} \PY{o}{+} \PY{n}{b}\PY{p}{)}
\PY{n}{ax}\PY{o}{.}\PY{n}{set\PYZus{}title}\PY{p}{(}\PY{l+s}{\PYZsq{}}\PY{l+s}{Linear Fit without dropping points}\PY{l+s}{\PYZsq{}}\PY{p}{)}
\PY{n}{ax}\PY{o}{.}\PY{n}{set\PYZus{}xlabel}\PY{p}{(}\PY{l+s}{\PYZsq{}}\PY{l+s}{\PYZdl{}x\PYZdl{}}\PY{l+s}{\PYZsq{}}\PY{p}{)}
\PY{n}{ax}\PY{o}{.}\PY{n}{set\PYZus{}ylabel}\PY{p}{(}\PY{l+s}{\PYZsq{}}\PY{l+s}{\PYZdl{}y\PYZdl{}}\PY{l+s}{\PYZsq{}}\PY{p}{)}
\PY{n}{ax}\PY{o}{.}\PY{n}{text}\PY{p}{(}\PY{l+m+mi}{30}\PY{p}{,} \PY{l+m+mi}{400}\PY{p}{,} \PY{l+s}{\PYZsq{}}\PY{l+s}{\PYZdl{}m = 1.08}\PY{l+s}{\PYZbs{}}\PY{l+s}{pm 0.08\PYZdl{}}\PY{l+s+se}{\PYZbs{}n}\PY{l+s}{\PYZdl{}b = 213 }\PY{l+s}{\PYZbs{}}\PY{l+s}{pm 14\PYZdl{}}\PY{l+s}{\PYZsq{}}\PY{p}{)}
\end{Verbatim}
%
\par%
\vspace{-1\smallerfontscale}}%
\end{addmargin}
\end{notebookcell}

\par\vspace{1\smallerfontscale}%
    \needspace{4\baselineskip}%
    % Only render the prompt if the cell is pyout.  Note, the outputs prompt 
    % block isn't used since we need to check each indiviual output and only
    % add prompts to the pyout ones.
    
        {\par%
        \vspace{-1\smallerfontscale}%
        \noindent%
        \begin{minipage}{\cellleftmargin}%
    \hfill%
    {\smaller%
    \tt%
    \color{nbframe-out-prompt}%
    Out[9]:}%
    \hspace{\inputpadding}%
    \hspace{0em}%
    \hspace{3pt}%
    \end{minipage}%%
        }%
    %
    %
    \begin{addmargin}[\cellleftmargin]{0em}% left, right
    {\smaller%
    \vspace{-1\smallerfontscale}%
    
    
    
    \begin{verbatim}
<matplotlib.text.Text at 0x10f98de90>
    \end{verbatim}

    
}%
    \end{addmargin}%\par\vspace{1\smallerfontscale}%
    \needspace{4\baselineskip}%
    % Only render the prompt if the cell is pyout.  Note, the outputs prompt 
    % block isn't used since we need to check each indiviual output and only
    % add prompts to the pyout ones.
    %
    %
    \begin{addmargin}[\cellleftmargin]{0em}% left, right
    {\smaller%
    \vspace{-1\smallerfontscale}%
    
    \begin{center}
    \adjustimage{max size={0.9\linewidth}{0.9\paperheight}}{Astro518_HW_02_files/Astro518_HW_02_10_1.png}
    \end{center}
    { \hspace*{\fill} \\}
    }%
    \end{addmargin}%
    \begin{itemize}
\itemsep1pt\parskip0pt\parsep0pt
\item
  Comparison of the results of 1.a and 1.b:
\item
  1.a,

  \begin{align*}
  & m = 2.24\pm 0.11\\
  & b = 34 \pm 18 
  \end{align*}
\item
  1.b

  \begin{align*}
  & m = 1.08\pm 0.08\\
  & b = 213 \pm 14
  \end{align*}
\item
  The difference of the two sets of results are more than $5\sigma$
\item
  1.b has smaller uncertainties because this fit bases on a larger data
  set
\end{itemize}

\subsection{Include a quadratic term}\label{include-a-quadratic-term}

matrix $A$ changes to

\begin{equation}
A_{\mathrm{quad}} = 
\begin{bmatrix}
1&x_1&x_1^2\\
1&x_2&x_2^2\\
\vdots&\vdots&\vdots\\
1&x_n&x_n^2
\end{bmatrix}
\end{equation}

    % Add contents below.

{\par%
\vspace{-1\baselineskip}%
\needspace{4\baselineskip}}%
\begin{notebookcell}[10]%
\begin{addmargin}[\cellleftmargin]{0em}% left, right
{\smaller%
\par%
%
\vspace{-1\smallerfontscale}%
\begin{Verbatim}[commandchars=\\\{\}]
\PY{n}{A\PYZus{}quad} \PY{o}{=} \PY{n}{np}\PY{o}{.}\PY{n}{mat}\PY{p}{(}\PY{p}{[}\PY{n}{np}\PY{o}{.}\PY{n}{ones}\PY{p}{(}\PY{n+nb}{len}\PY{p}{(}\PY{n}{x\PYZus{}1}\PY{p}{)}\PY{p}{)}\PY{p}{,} \PY{n}{x\PYZus{}1}\PY{p}{,} \PY{n}{x\PYZus{}1}\PY{o}{*}\PY{o}{*}\PY{l+m+mi}{2}\PY{p}{]}\PY{p}{)}\PY{o}{.}\PY{n}{T}


\PY{c}{\PYZsh{}\PYZsh{} fit y = a2*x**2 + a1*x + a0}
\PY{p}{(}\PY{n}{a0\PYZus{}quad}\PY{p}{,} \PY{n}{a1\PYZus{}quad}\PY{p}{,} \PY{n}{a2\PYZus{}quad}\PY{p}{)} \PY{o}{=} \PY{p}{(}\PY{n}{A\PYZus{}quad}\PY{o}{.}\PY{n}{T} \PY{o}{*}\PY{n}{C\PYZus{}1}\PY{o}{*}\PY{o}{*}\PY{p}{(}\PY{o}{\PYZhy{}}\PY{l+m+mi}{1}\PY{p}{)} \PY{o}{*} \PY{n}{A\PYZus{}quad}\PY{p}{)}\PY{o}{*}\PY{o}{*}\PY{p}{(}\PY{o}{\PYZhy{}}\PY{l+m+mi}{1}\PY{p}{)} \PY{o}{*} \PY{p}{(}\PY{n}{A\PYZus{}quad}\PY{o}{.}\PY{n}{T} \PY{o}{*} \PY{n}{C\PYZus{}1}\PY{o}{*}\PY{o}{*}\PY{p}{(}\PY{o}{\PYZhy{}}\PY{l+m+mi}{1}\PY{p}{)} \PY{o}{*} \PY{n}{Y\PYZus{}vec\PYZus{}1}\PY{p}{)}

\PY{n}{a0\PYZus{}quad} \PY{o}{=} \PY{n}{a0\PYZus{}quad}\PY{o}{.}\PY{n}{flat}\PY{p}{[}\PY{l+m+mi}{0}\PY{p}{]}
\PY{n}{a1\PYZus{}quad} \PY{o}{=} \PY{n}{a1\PYZus{}quad}\PY{o}{.}\PY{n}{flat}\PY{p}{[}\PY{l+m+mi}{0}\PY{p}{]}
\PY{n}{a2\PYZus{}quad} \PY{o}{=} \PY{n}{a2\PYZus{}quad}\PY{o}{.}\PY{n}{flat}\PY{p}{[}\PY{l+m+mi}{0}\PY{p}{]}

\PY{k}{print} \PY{l+s}{\PYZsq{}}\PY{l+s}{Fitting result for function y = a2*x**2 + a1*x + a0}\PY{l+s}{\PYZsq{}}
\PY{k}{print} \PY{l+s}{\PYZsq{}}\PY{l+s}{a2 = \PYZob{}0:.4f\PYZcb{}}\PY{l+s}{\PYZsq{}}\PY{o}{.}\PY{n}{format}\PY{p}{(}\PY{n}{a2\PYZus{}quad}\PY{p}{)}
\PY{k}{print} \PY{l+s}{\PYZsq{}}\PY{l+s}{a1 = \PYZob{}0:.2f\PYZcb{}}\PY{l+s}{\PYZsq{}}\PY{o}{.}\PY{n}{format}\PY{p}{(}\PY{n}{a1\PYZus{}quad}\PY{p}{)}
\PY{k}{print} \PY{l+s}{\PYZsq{}}\PY{l+s}{a0 = \PYZob{}0:.2f\PYZcb{}}\PY{l+s}{\PYZsq{}}\PY{o}{.}\PY{n}{format}\PY{p}{(}\PY{n}{a0\PYZus{}quad}\PY{p}{)}

\PY{c}{\PYZsh{}\PYZsh{} calculate the uncertainties}
\PY{n}{var\PYZus{}as} \PY{o}{=} \PY{n}{np}\PY{o}{.}\PY{n}{diag}\PY{p}{(}\PY{p}{(}\PY{n}{A\PYZus{}quad}\PY{o}{.}\PY{n}{T} \PY{o}{*} \PY{n}{C\PYZus{}1}\PY{o}{*}\PY{o}{*}\PY{p}{(}\PY{o}{\PYZhy{}}\PY{l+m+mi}{1}\PY{p}{)} \PY{o}{*} \PY{n}{A\PYZus{}quad}\PY{p}{)}\PY{o}{*}\PY{o}{*}\PY{p}{(}\PY{o}{\PYZhy{}}\PY{l+m+mi}{1}\PY{p}{)}\PY{p}{)}
\PY{n}{da0}\PY{p}{,} \PY{n}{da1}\PY{p}{,} \PY{n}{da2} \PY{o}{=} \PY{n}{np}\PY{o}{.}\PY{n}{sqrt}\PY{p}{(}\PY{n}{var\PYZus{}as}\PY{p}{)}

\PY{k}{print} \PY{l+s}{\PYZsq{}}\PY{l+s}{Uncertainties:}\PY{l+s}{\PYZsq{}}
\PY{k}{print} \PY{l+s}{\PYZsq{}}\PY{l+s}{sigma\PYZus{}a2 = \PYZob{}0:.4f\PYZcb{}}\PY{l+s}{\PYZsq{}}\PY{o}{.}\PY{n}{format}\PY{p}{(}\PY{n}{da2}\PY{p}{)}
\PY{k}{print} \PY{l+s}{\PYZsq{}}\PY{l+s}{sigma\PYZus{}a1 = \PYZob{}0:.4f\PYZcb{}}\PY{l+s}{\PYZsq{}}\PY{o}{.}\PY{n}{format}\PY{p}{(}\PY{n}{da1}\PY{p}{)}
\PY{k}{print} \PY{l+s}{\PYZsq{}}\PY{l+s}{sigma\PYZus{}a0 = \PYZob{}0:.4f\PYZcb{}}\PY{l+s}{\PYZsq{}}\PY{o}{.}\PY{n}{format}\PY{p}{(}\PY{n}{da0}\PY{p}{)}
\end{Verbatim}
%
\par%
\vspace{-1\smallerfontscale}}%
\end{addmargin}
\end{notebookcell}

\par\vspace{1\smallerfontscale}%
    \needspace{4\baselineskip}%
    % Only render the prompt if the cell is pyout.  Note, the outputs prompt 
    % block isn't used since we need to check each indiviual output and only
    % add prompts to the pyout ones.
    %
    %
    \begin{addmargin}[\cellleftmargin]{0em}% left, right
    {\smaller%
    \vspace{-1\smallerfontscale}%
    
    \begin{Verbatim}[commandchars=\\\{\}]
Fitting result for function y = a2*x**2 + a1*x + a0
a2 = 0.0023
a1 = 1.60
a0 = 72.89
Uncertainties:
sigma\_a2 = 0.0020
sigma\_a1 = 0.5797
sigma\_a0 = 38.9116
    \end{Verbatim}
}%
    \end{addmargin}%
    % Add contents below.

{\par%
\vspace{-1\baselineskip}%
\needspace{4\baselineskip}}%
\begin{notebookcell}[95]%
\begin{addmargin}[\cellleftmargin]{0em}% left, right
{\smaller%
\par%
%
\vspace{-1\smallerfontscale}%
\begin{Verbatim}[commandchars=\\\{\}]
\PY{n}{fig} \PY{o}{=} \PY{n}{plt}\PY{o}{.}\PY{n}{figure}\PY{p}{(}\PY{p}{)}
\PY{n}{ax} \PY{o}{=} \PY{n}{fig}\PY{o}{.}\PY{n}{add\PYZus{}subplot}\PY{p}{(}\PY{l+m+mi}{111}\PY{p}{)}
\PY{n}{ax}\PY{o}{.}\PY{n}{errorbar}\PY{p}{(}\PY{n}{x\PYZus{}1}\PY{p}{,} \PY{n}{y\PYZus{}1}\PY{p}{,} \PY{n}{yerr} \PY{o}{=} \PY{n}{dy\PYZus{}1}\PY{p}{,} \PY{n}{fmt} \PY{o}{=} \PY{l+s}{\PYZsq{}}\PY{l+s}{o}\PY{l+s}{\PYZsq{}}\PY{p}{,} \PY{n}{mew} \PY{o}{=} \PY{l+m+mf}{0.6}\PY{p}{)}
\PY{n}{ax}\PY{o}{.}\PY{n}{plot}\PY{p}{(}\PY{n}{x\PYZus{}sample}\PY{p}{,} \PY{n}{x\PYZus{}sample}\PY{o}{*}\PY{o}{*}\PY{l+m+mi}{2} \PY{o}{*} \PY{n}{a2\PYZus{}quad} \PY{o}{+} \PY{n}{x\PYZus{}sample}\PY{o}{*}\PY{n}{a1\PYZus{}quad} \PY{o}{+} \PY{n}{a0\PYZus{}quad}\PY{p}{,} 
        \PY{n}{label} \PY{o}{=} \PY{l+s}{\PYZsq{}}\PY{l+s}{With quadratic term,}\PY{l+s+se}{\PYZbs{}n}\PY{l+s}{ \PYZdl{}y=0.002x\PYZca{}2+0.58x+38.9\PYZdl{}}\PY{l+s}{\PYZsq{}} \PY{p}{)}
\PY{n}{ax}\PY{o}{.}\PY{n}{plot}\PY{p}{(}\PY{n}{x\PYZus{}sample}\PY{p}{,} \PY{n}{x\PYZus{}sample} \PY{o}{*} \PY{n}{m\PYZus{}1} \PY{o}{+} \PY{n}{b\PYZus{}1}\PY{p}{,} \PY{n}{label} \PY{o}{=} \PY{l+s}{\PYZsq{}}\PY{l+s}{Linear Fit,}\PY{l+s+se}{\PYZbs{}n}\PY{l+s}{ \PYZdl{}y=2.24x+34\PYZdl{}}\PY{l+s}{\PYZsq{}}\PY{p}{)}
\PY{n}{ax}\PY{o}{.}\PY{n}{set\PYZus{}title}\PY{p}{(}\PY{l+s}{\PYZsq{}}\PY{l+s}{Fit with quadratic term}\PY{l+s}{\PYZsq{}}\PY{p}{)}
\PY{n}{ax}\PY{o}{.}\PY{n}{set\PYZus{}xlabel}\PY{p}{(}\PY{l+s}{\PYZsq{}}\PY{l+s}{\PYZdl{}x\PYZdl{}}\PY{l+s}{\PYZsq{}}\PY{p}{)}
\PY{n}{ax}\PY{o}{.}\PY{n}{set\PYZus{}ylabel}\PY{p}{(}\PY{l+s}{\PYZsq{}}\PY{l+s}{\PYZdl{}y\PYZdl{}}\PY{l+s}{\PYZsq{}}\PY{p}{)}
\PY{n}{ppl}\PY{o}{.}\PY{n}{legend}\PY{p}{(}\PY{n}{ax}\PY{p}{,} \PY{n}{loc} \PY{o}{=} \PY{l+s}{\PYZsq{}}\PY{l+s}{best}\PY{l+s}{\PYZsq{}}\PY{p}{)}
\end{Verbatim}
%
\par%
\vspace{-1\smallerfontscale}}%
\end{addmargin}
\end{notebookcell}

\par\vspace{1\smallerfontscale}%
    \needspace{4\baselineskip}%
    % Only render the prompt if the cell is pyout.  Note, the outputs prompt 
    % block isn't used since we need to check each indiviual output and only
    % add prompts to the pyout ones.
    
        {\par%
        \vspace{-1\smallerfontscale}%
        \noindent%
        \begin{minipage}{\cellleftmargin}%
    \hfill%
    {\smaller%
    \tt%
    \color{nbframe-out-prompt}%
    Out[95]:}%
    \hspace{\inputpadding}%
    \hspace{0em}%
    \hspace{3pt}%
    \end{minipage}%%
        }%
    %
    %
    \begin{addmargin}[\cellleftmargin]{0em}% left, right
    {\smaller%
    \vspace{-1\smallerfontscale}%
    
    
    
    \begin{verbatim}
<matplotlib.legend.Legend at 0x10c41f2d0>
    \end{verbatim}

    
}%
    \end{addmargin}%\par\vspace{1\smallerfontscale}%
    \needspace{4\baselineskip}%
    % Only render the prompt if the cell is pyout.  Note, the outputs prompt 
    % block isn't used since we need to check each indiviual output and only
    % add prompts to the pyout ones.
    %
    %
    \begin{addmargin}[\cellleftmargin]{0em}% left, right
    {\smaller%
    \vspace{-1\smallerfontscale}%
    
    \begin{center}
    \adjustimage{max size={0.9\linewidth}{0.9\paperheight}}{Astro518_HW_02_files/Astro518_HW_02_13_1.png}
    \end{center}
    { \hspace*{\fill} \\}
    }%
    \end{addmargin}%
    \section{Problem 2, Goodness to fit and use
$\chi^2$}\label{problem-2-goodness-to-fit-and-use-chi2}

\subsection{$\chi^2$ calculation}\label{chi2-calculation}

    % Add contents below.

{\par%
\vspace{-1\baselineskip}%
\needspace{4\baselineskip}}%
\begin{notebookcell}[11]%
\begin{addmargin}[\cellleftmargin]{0em}% left, right
{\smaller%
\par%
%
\vspace{-1\smallerfontscale}%
\begin{Verbatim}[commandchars=\\\{\}]
\PY{k}{def} \PY{n+nf}{chisq}\PY{p}{(}\PY{n}{y}\PY{p}{,} \PY{n}{y\PYZus{}mod}\PY{p}{,} \PY{n}{dy}\PY{p}{)}\PY{p}{:}
    \PY{l+s+sd}{\PYZdq{}\PYZdq{}\PYZdq{} }
\PY{l+s+sd}{    function to calculate chi square}
\PY{l+s+sd}{    \PYZdq{}\PYZdq{}\PYZdq{}}
    \PY{k}{return} \PY{n}{np}\PY{o}{.}\PY{n}{sum}\PY{p}{(}\PY{p}{(}\PY{n}{y}\PY{o}{\PYZhy{}}\PY{n}{y\PYZus{}mod}\PY{p}{)}\PY{o}{*}\PY{o}{*}\PY{l+m+mi}{2}\PY{o}{/}\PY{n}{dy}\PY{o}{*}\PY{o}{*}\PY{l+m+mi}{2}\PY{p}{)}

\PY{n}{chisq\PYZus{}1a} \PY{o}{=} \PY{n}{chisq}\PY{p}{(}\PY{n}{y\PYZus{}1}\PY{p}{,} \PY{n}{x\PYZus{}1} \PY{o}{*} \PY{n}{m\PYZus{}1} \PY{o}{+} \PY{n}{b\PYZus{}1}\PY{p}{,} \PY{n}{dy\PYZus{}1}\PY{p}{)}
\PY{n}{chisq\PYZus{}1b} \PY{o}{=} \PY{n}{chisq}\PY{p}{(}\PY{n}{y}\PY{p}{,} \PY{n}{x} \PY{o}{*} \PY{n}{m} \PY{o}{+} \PY{n}{b}\PY{p}{,} \PY{n}{dy}\PY{p}{)}

\PY{n}{chisq\PYZus{}1a0} \PY{o}{=} \PY{n}{chisq\PYZus{}1a}\PY{o}{/}\PY{p}{(}\PY{n+nb}{len}\PY{p}{(}\PY{n}{x\PYZus{}1}\PY{p}{)} \PY{o}{\PYZhy{}} \PY{l+m+mi}{2}\PY{p}{)}
\PY{n}{chisq\PYZus{}1b0} \PY{o}{=} \PY{n}{chisq\PYZus{}1b}\PY{o}{/}\PY{p}{(}\PY{n+nb}{len}\PY{p}{(}\PY{n}{x}\PY{p}{)} \PY{o}{\PYZhy{}} \PY{l+m+mi}{2}\PY{p}{)}

\PY{k}{print} \PY{l+s}{\PYZsq{}}\PY{l+s}{chi square calculation:}\PY{l+s}{\PYZsq{}}
\PY{k}{print} \PY{l+s}{\PYZsq{}}\PY{l+s}{For 1a,}\PY{l+s+se}{\PYZbs{}n}\PY{l+s}{ chisq = \PYZob{}0:.2f\PYZcb{}}\PY{l+s+se}{\PYZbs{}n}\PY{l+s}{ chisq/(N\PYZhy{}2) = \PYZob{}1:.2f\PYZcb{}}\PY{l+s}{\PYZsq{}}\PY{o}{.}\PY{n}{format}\PY{p}{(}\PY{n}{chisq\PYZus{}1a}\PY{p}{,} \PY{n}{chisq\PYZus{}1a0}\PY{p}{)}
\PY{k}{print} \PY{l+s}{\PYZsq{}}\PY{l+s}{For 1b,,}\PY{l+s+se}{\PYZbs{}n}\PY{l+s}{ chisq = \PYZob{}0:.2f\PYZcb{}}\PY{l+s+se}{\PYZbs{}n}\PY{l+s}{ chisq/(N\PYZhy{}2) = \PYZob{}1:.2f\PYZcb{}}\PY{l+s}{\PYZsq{}}\PY{o}{.}\PY{n}{format}\PY{p}{(}\PY{n}{chisq\PYZus{}1b}\PY{p}{,} \PY{n}{chisq\PYZus{}1b0}\PY{p}{)}
\end{Verbatim}
%
\par%
\vspace{-1\smallerfontscale}}%
\end{addmargin}
\end{notebookcell}

\par\vspace{1\smallerfontscale}%
    \needspace{4\baselineskip}%
    % Only render the prompt if the cell is pyout.  Note, the outputs prompt 
    % block isn't used since we need to check each indiviual output and only
    % add prompts to the pyout ones.
    %
    %
    \begin{addmargin}[\cellleftmargin]{0em}% left, right
    {\smaller%
    \vspace{-1\smallerfontscale}%
    
    \begin{Verbatim}[commandchars=\\\{\}]
chi square calculation:
For 1a,
 chisq = 18.68
 chisq/(N-2) = 1.33
For 1b,,
 chisq = 289.96
 chisq/(N-2) = 16.11
    \end{Verbatim}
}%
    \end{addmargin}%
    The results of $\chi^2$ calculation shows that, 1. for 1a,
$\chi^2/(N-2) \sim 1$, the data are well fitted to a linear function. 2.
for 1b, $\chi^2/(N-2) \gg 1$, the data are not well defined by a linear
function.

\subsection{replace $\sigma$ to get a `good'
$\chi$}\label{replace-sigma-to-get-a-good-chi}

suppose

\begin{equation}
\chi^2_0 = \sum\frac{(y_i-y(x_i))^2}{S^2(N-2)} = 1
\end{equation}

we can get

\begin{equation}
S = \sqrt{\sum((y_i-y(x_i))^2/(N-2)}
\end{equation}

    % Add contents below.

{\par%
\vspace{-1\baselineskip}%
\needspace{4\baselineskip}}%
\begin{notebookcell}[12]%
\begin{addmargin}[\cellleftmargin]{0em}% left, right
{\smaller%
\par%
%
\vspace{-1\smallerfontscale}%
\begin{Verbatim}[commandchars=\\\{\}]
\PY{n}{S\PYZus{}1a} \PY{o}{=} \PY{n}{np}\PY{o}{.}\PY{n}{sqrt}\PY{p}{(}\PY{n}{chisq}\PY{p}{(}\PY{n}{y\PYZus{}1}\PY{p}{,} \PY{n}{m\PYZus{}1} \PY{o}{*} \PY{n}{x\PYZus{}1} \PY{o}{+} \PY{n}{b\PYZus{}1}\PY{p}{,} \PY{l+m+mi}{1}\PY{p}{)}\PY{o}{/}\PY{p}{(}\PY{n+nb}{len}\PY{p}{(}\PY{n}{x\PYZus{}1}\PY{p}{)} \PY{o}{\PYZhy{}} \PY{l+m+mi}{2}\PY{p}{)}\PY{p}{)}
\PY{n}{S\PYZus{}1b} \PY{o}{=} \PY{n}{np}\PY{o}{.}\PY{n}{sqrt}\PY{p}{(}\PY{n}{chisq}\PY{p}{(}\PY{n}{y}\PY{p}{,} \PY{n}{m} \PY{o}{*} \PY{n}{x} \PY{o}{+} \PY{n}{b}\PY{p}{,} \PY{l+m+mi}{1}\PY{p}{)}\PY{o}{/}\PY{p}{(}\PY{n+nb}{len}\PY{p}{(}\PY{n}{x}\PY{p}{)} \PY{o}{\PYZhy{}} \PY{l+m+mi}{2}\PY{p}{)}\PY{p}{)}

\PY{k}{print} \PY{l+s}{\PYZsq{}}\PY{l+s}{For 1a, this S value is S = \PYZob{}0:.2f\PYZcb{}}\PY{l+s}{\PYZsq{}}\PY{o}{.}\PY{n}{format}\PY{p}{(}\PY{n}{S\PYZus{}1a}\PY{p}{)}
\PY{k}{print} \PY{l+s}{\PYZsq{}}\PY{l+s}{For 1b, this S value is S = \PYZob{}0:.2f\PYZcb{}}\PY{l+s}{\PYZsq{}}\PY{o}{.}\PY{n}{format}\PY{p}{(}\PY{n}{S\PYZus{}1b}\PY{p}{)}
\end{Verbatim}
%
\par%
\vspace{-1\smallerfontscale}}%
\end{addmargin}
\end{notebookcell}

\par\vspace{1\smallerfontscale}%
    \needspace{4\baselineskip}%
    % Only render the prompt if the cell is pyout.  Note, the outputs prompt 
    % block isn't used since we need to check each indiviual output and only
    % add prompts to the pyout ones.
    %
    %
    \begin{addmargin}[\cellleftmargin]{0em}% left, right
    {\smaller%
    \vspace{-1\smallerfontscale}%
    
    \begin{Verbatim}[commandchars=\\\{\}]
For 1a, this S value is S = 32.16
For 1b, this S value is S = 104.01
    \end{Verbatim}
}%
    \end{addmargin}%
    \begin{itemize}
\itemsep1pt\parskip0pt\parsep0pt
\item
  1.b needs a much larger $S$ value
\end{itemize}

\subsection{Maximum likelihood
calculation}\label{maximum-likelihood-calculation}

    % Add contents below.

{\par%
\vspace{-1\baselineskip}%
\needspace{4\baselineskip}}%
\begin{notebookcell}[13]%
\begin{addmargin}[\cellleftmargin]{0em}% left, right
{\smaller%
\par%
%
\vspace{-1\smallerfontscale}%
\begin{Verbatim}[commandchars=\\\{\}]
\PY{k}{def} \PY{n+nf}{maxLikelihood}\PY{p}{(}\PY{n}{func}\PY{p}{,} \PY{n}{x}\PY{p}{,} \PY{n}{y}\PY{p}{,} \PY{n}{dy}\PY{p}{)}\PY{p}{:}
    \PY{l+s+sd}{\PYZdq{}\PYZdq{}\PYZdq{}}
\PY{l+s+sd}{    maximum likelihood calculation}
\PY{l+s+sd}{    assuming that y have a Gaussian distribution}
\PY{l+s+sd}{    ignoring the x\PYZsq{}s uncertaities}
\PY{l+s+sd}{    \PYZdq{}\PYZdq{}\PYZdq{}}
    \PY{k}{return} \PY{n}{np}\PY{o}{.}\PY{n}{prod}\PY{p}{(}\PY{l+m+mi}{1}\PY{o}{/}\PY{n}{np}\PY{o}{.}\PY{n}{sqrt}\PY{p}{(}\PY{l+m+mi}{2} \PY{o}{*} \PY{n}{np}\PY{o}{.}\PY{n}{pi} \PY{o}{*} \PY{n}{dy}\PY{o}{*}\PY{o}{*}\PY{l+m+mi}{2}\PY{p}{)} \PY{o}{*} \PY{n}{np}\PY{o}{.}\PY{n}{exp}\PY{p}{(}\PY{o}{\PYZhy{}}\PY{p}{(}\PY{n}{y} \PY{o}{\PYZhy{}} \PY{n}{func}\PY{p}{(}\PY{n}{x}\PY{p}{)}\PY{p}{)}\PY{o}{*}\PY{o}{*}\PY{l+m+mi}{2}\PY{o}{/}\PY{p}{(}\PY{l+m+mi}{2} \PY{o}{*} \PY{n}{dy}\PY{o}{*}\PY{o}{*}\PY{l+m+mi}{2}\PY{p}{)}\PY{p}{)}\PY{p}{)}

\PY{k}{def} \PY{n+nf}{linearMLMat}\PY{p}{(}\PY{n}{m\PYZus{}min}\PY{p}{,} \PY{n}{m\PYZus{}max}\PY{p}{,} \PY{n}{b\PYZus{}min}\PY{p}{,} \PY{n}{b\PYZus{}max}\PY{p}{,} \PY{n}{x}\PY{p}{,} \PY{n}{y}\PY{p}{,} \PY{n}{dy}\PY{p}{,} \PY{n}{ngrid} \PY{o}{=} \PY{l+m+mi}{100}\PY{p}{)}\PY{p}{:}
    \PY{l+s+sd}{\PYZdq{}\PYZdq{}\PYZdq{}}
\PY{l+s+sd}{    return a 2d array that contains the ML calculation for each grid point}
\PY{l+s+sd}{    \PYZdq{}\PYZdq{}\PYZdq{}}
    \PY{n}{m\PYZus{}range} \PY{o}{=} \PY{n}{np}\PY{o}{.}\PY{n}{linspace}\PY{p}{(}\PY{n}{m\PYZus{}min}\PY{p}{,} \PY{n}{m\PYZus{}max}\PY{p}{,} \PY{n}{ngrid}\PY{p}{)}
    \PY{n}{b\PYZus{}range} \PY{o}{=} \PY{n}{np}\PY{o}{.}\PY{n}{linspace}\PY{p}{(}\PY{n}{b\PYZus{}min}\PY{p}{,} \PY{n}{b\PYZus{}max}\PY{p}{,} \PY{n}{ngrid}\PY{p}{)}
    
    \PY{n}{mb\PYZus{}mesh} \PY{o}{=} \PY{p}{[}\PY{p}{(}\PY{n}{mi}\PY{p}{,} \PY{n}{bj}\PY{p}{)} \PY{k}{for} \PY{n}{mi} \PY{o+ow}{in} \PY{n}{m\PYZus{}range} \PY{k}{for} \PY{n}{bj} \PY{o+ow}{in} \PY{n}{b\PYZus{}range}\PY{p}{]}\PY{c}{\PYZsh{} a list of the grid coordinates}
    
    \PY{k}{def} \PY{n+nf}{ml\PYZus{}func}\PY{p}{(}\PY{n}{m}\PY{p}{,} \PY{n}{b}\PY{p}{)}\PY{p}{:} 
        \PY{k}{return} \PY{n}{maxLikelihood}\PY{p}{(}\PY{k}{lambda} \PY{n}{xi}\PY{p}{:} \PY{n}{m} \PY{o}{*} \PY{n}{xi} \PY{o}{+} \PY{n}{b}\PY{p}{,} \PY{n}{x}\PY{p}{,} \PY{n}{y}\PY{p}{,} \PY{n}{dy}\PY{p}{)}
    
    \PY{c}{\PYZsh{} use `map` to avoid for\PYZhy{}loop to accelerate the program}
    \PY{k}{return} \PY{n}{m\PYZus{}range}\PY{p}{,} \PY{n}{b\PYZus{}range}\PY{p}{,} \PY{n}{np}\PY{o}{.}\PY{n}{array}\PY{p}{(}\PY{n+nb}{map}\PY{p}{(}\PY{n}{ml\PYZus{}func}\PY{p}{,} \PY{n}{mb\PYZus{}mesh}\PY{p}{)}\PY{p}{)}\PY{o}{.}\PY{n}{reshape}\PY{p}{(}\PY{p}{(}\PY{n+nb}{len}\PY{p}{(}\PY{n}{m\PYZus{}range}\PY{p}{)}\PY{p}{,} \PY{n+nb}{len}\PY{p}{(}\PY{n}{b\PYZus{}range}\PY{p}{)}\PY{p}{)}\PY{p}{)}

\PY{n}{mrange\PYZus{}1a}\PY{p}{,} \PY{n}{brange\PYZus{}1a}\PY{p}{,} \PY{n}{mlmat\PYZus{}1a} \PY{o}{=} \PY{n}{linearMLMat}\PY{p}{(}\PY{n}{m\PYZus{}1} \PY{o}{\PYZhy{}} \PY{l+m+mi}{3} \PY{o}{*} \PY{n}{dm\PYZus{}1}\PY{p}{,} \PY{n}{m\PYZus{}1} \PY{o}{+} \PY{l+m+mi}{3} \PY{o}{*} \PY{n}{dm\PYZus{}1}\PY{p}{,}
                                             \PY{n}{b\PYZus{}1} \PY{o}{\PYZhy{}} \PY{l+m+mi}{3} \PY{o}{*} \PY{n}{db\PYZus{}1}\PY{p}{,} \PY{n}{b\PYZus{}1} \PY{o}{+} \PY{l+m+mi}{3} \PY{o}{*} \PY{n}{db\PYZus{}1}\PY{p}{,} \PY{n}{x\PYZus{}1}\PY{p}{,} \PY{n}{y\PYZus{}1}\PY{p}{,} \PY{n}{dy\PYZus{}1}\PY{p}{)}
\PY{n}{mrange\PYZus{}1b}\PY{p}{,} \PY{n}{brange\PYZus{}1b}\PY{p}{,} \PY{n}{mlmat\PYZus{}1b} \PY{o}{=} \PY{n}{linearMLMat}\PY{p}{(}\PY{n}{m} \PY{o}{\PYZhy{}} \PY{l+m+mi}{3} \PY{o}{*} \PY{n}{dm}\PY{p}{,} \PY{n}{m} \PY{o}{+} \PY{l+m+mi}{3} \PY{o}{*} \PY{n}{dm}\PY{p}{,} \PY{n}{b} \PY{o}{\PYZhy{}} \PY{l+m+mi}{3} \PY{o}{*} \PY{n}{db}\PY{p}{,} 
                                             \PY{n}{b} \PY{o}{+} \PY{l+m+mi}{3} \PY{o}{*} \PY{n}{db}\PY{p}{,} \PY{n}{x}\PY{p}{,} \PY{n}{y}\PY{p}{,} \PY{n}{dy}\PY{p}{)}
\end{Verbatim}
%
\par%
\vspace{-1\smallerfontscale}}%
\end{addmargin}
\end{notebookcell}


    % Add contents below.

{\par%
\vspace{-1\baselineskip}%
\needspace{4\baselineskip}}%
\begin{notebookcell}[13]%
\begin{addmargin}[\cellleftmargin]{0em}% left, right
{\smaller%
\par%
%
\vspace{-1\smallerfontscale}%
\begin{Verbatim}[commandchars=\\\{\}]
\PY{c}{\PYZsh{}\PYZsh{}\PYZsh{} plot the contour/color image}
\PY{n}{fig}\PY{p}{,} \PY{n}{ax} \PY{o}{=} \PY{n}{plt}\PY{o}{.}\PY{n}{subplots}\PY{p}{(}\PY{p}{)}
\PY{n}{cax\PYZus{}1a} \PY{o}{=} \PY{n}{ax}\PY{o}{.}\PY{n}{pcolormesh}\PY{p}{(}\PY{n}{brange\PYZus{}1a}\PY{p}{,} \PY{n}{mrange\PYZus{}1a}\PY{p}{,} \PY{n}{mlmat\PYZus{}1a}\PY{p}{,} \PY{n}{cmap} \PY{o}{=} \PY{l+s}{\PYZsq{}}\PY{l+s}{Greens}\PY{l+s}{\PYZsq{}}\PY{p}{)}
\PY{n}{ax}\PY{o}{.}\PY{n}{contour}\PY{p}{(}\PY{n}{brange\PYZus{}1a}\PY{p}{,} \PY{n}{mrange\PYZus{}1a}\PY{p}{,} \PY{n}{mlmat\PYZus{}1a}\PY{p}{,} \PY{l+m+mi}{5}\PY{p}{,} \PY{n}{linewidths} \PY{o}{=} \PY{l+m+mf}{0.6}\PY{p}{,} \PY{n}{colors} \PY{o}{=} \PY{l+s}{\PYZsq{}}\PY{l+s}{0.4}\PY{l+s}{\PYZsq{}}\PY{p}{)}
\PY{n}{ax}\PY{o}{.}\PY{n}{set\PYZus{}xlabel}\PY{p}{(}\PY{l+s}{\PYZsq{}}\PY{l+s}{\PYZdl{}b\PYZdl{}}\PY{l+s}{\PYZsq{}}\PY{p}{)}
\PY{n}{ax}\PY{o}{.}\PY{n}{set\PYZus{}ylabel}\PY{p}{(}\PY{l+s}{\PYZsq{}}\PY{l+s}{\PYZdl{}m\PYZdl{}}\PY{l+s}{\PYZsq{}}\PY{p}{)}
\PY{n}{ax}\PY{o}{.}\PY{n}{set\PYZus{}title}\PY{p}{(}\PY{l+s}{\PYZsq{}}\PY{l+s}{Contour/Color image for Max Likelihood of 1a}\PY{l+s}{\PYZsq{}}\PY{p}{)}
\PY{n}{ax}\PY{o}{.}\PY{n}{autoscale}\PY{p}{(}\PY{n}{tight} \PY{o}{=} \PY{n+nb+bp}{True}\PY{p}{)}
\PY{n}{fig}\PY{o}{.}\PY{n}{colorbar}\PY{p}{(}\PY{n}{cax\PYZus{}1a}\PY{p}{)}
\PY{n}{fig}\PY{o}{.}\PY{n}{tight\PYZus{}layout}\PY{p}{(}\PY{p}{)}

\PY{n}{figb}\PY{p}{,} \PY{n}{axb} \PY{o}{=} \PY{n}{plt}\PY{o}{.}\PY{n}{subplots}\PY{p}{(}\PY{p}{)}
\PY{n}{cax\PYZus{}1b} \PY{o}{=} \PY{n}{axb}\PY{o}{.}\PY{n}{pcolormesh}\PY{p}{(}\PY{n}{brange\PYZus{}1b}\PY{p}{,} \PY{n}{mrange\PYZus{}1b}\PY{p}{,} \PY{n}{mlmat\PYZus{}1b}\PY{p}{,} \PY{n}{cmap} \PY{o}{=} \PY{l+s}{\PYZsq{}}\PY{l+s}{Blues}\PY{l+s}{\PYZsq{}}\PY{p}{)}
\PY{n}{axb}\PY{o}{.}\PY{n}{contour}\PY{p}{(}\PY{n}{brange\PYZus{}1b}\PY{p}{,} \PY{n}{mrange\PYZus{}1b}\PY{p}{,} \PY{n}{mlmat\PYZus{}1b}\PY{p}{,} \PY{l+m+mi}{5}\PY{p}{,} \PY{n}{linewidths} \PY{o}{=} \PY{l+m+mf}{0.6}\PY{p}{,} \PY{n}{colors} \PY{o}{=} \PY{l+s}{\PYZsq{}}\PY{l+s}{0.4}\PY{l+s}{\PYZsq{}}\PY{p}{)}
\PY{n}{axb}\PY{o}{.}\PY{n}{set\PYZus{}xlabel}\PY{p}{(}\PY{l+s}{\PYZsq{}}\PY{l+s}{\PYZdl{}b\PYZdl{}}\PY{l+s}{\PYZsq{}}\PY{p}{)}
\PY{n}{axb}\PY{o}{.}\PY{n}{set\PYZus{}ylabel}\PY{p}{(}\PY{l+s}{\PYZsq{}}\PY{l+s}{\PYZdl{}m\PYZdl{}}\PY{l+s}{\PYZsq{}}\PY{p}{)}
\PY{n}{axb}\PY{o}{.}\PY{n}{set\PYZus{}title}\PY{p}{(}\PY{l+s}{\PYZsq{}}\PY{l+s}{Contour/Color image for Max Likelihood of 1b}\PY{l+s}{\PYZsq{}}\PY{p}{)}
\PY{n}{axb}\PY{o}{.}\PY{n}{autoscale}\PY{p}{(}\PY{n}{tight} \PY{o}{=} \PY{n+nb+bp}{True}\PY{p}{)}
\PY{n}{figb}\PY{o}{.}\PY{n}{colorbar}\PY{p}{(}\PY{n}{cax\PYZus{}1b}\PY{p}{)}
\PY{n}{figb}\PY{o}{.}\PY{n}{tight\PYZus{}layout}\PY{p}{(}\PY{p}{)}
\end{Verbatim}
%
\par%
\vspace{-1\smallerfontscale}}%
\end{addmargin}
\end{notebookcell}

\par\vspace{1\smallerfontscale}%
    \needspace{4\baselineskip}%
    % Only render the prompt if the cell is pyout.  Note, the outputs prompt 
    % block isn't used since we need to check each indiviual output and only
    % add prompts to the pyout ones.
    %
    %
    \begin{addmargin}[\cellleftmargin]{0em}% left, right
    {\smaller%
    \vspace{-1\smallerfontscale}%
    
    \begin{center}
    \adjustimage{max size={0.9\linewidth}{0.9\paperheight}}{Astro518_HW_02_files/Astro518_HW_02_20_0.png}
    \end{center}
    { \hspace*{\fill} \\}
    }%
    \end{addmargin}%\par\vspace{1\smallerfontscale}%
    \needspace{4\baselineskip}%
    % Only render the prompt if the cell is pyout.  Note, the outputs prompt 
    % block isn't used since we need to check each indiviual output and only
    % add prompts to the pyout ones.
    %
    %
    \begin{addmargin}[\cellleftmargin]{0em}% left, right
    {\smaller%
    \vspace{-1\smallerfontscale}%
    
    \begin{center}
    \adjustimage{max size={0.9\linewidth}{0.9\paperheight}}{Astro518_HW_02_files/Astro518_HW_02_20_1.png}
    \end{center}
    { \hspace*{\fill} \\}
    }%
    \end{addmargin}%
    \section{Problem 3, Bayesian
analysis}\label{problem-3-bayesian-analysis}

\subsection{a, Maximum likelihood with assumption of `bad point'
probability}\label{a-maximum-likelihood-with-assumption-of-bad-point-probability}

\begin{itemize}
\itemsep1pt\parskip0pt\parsep0pt
\item
  Following function \texttt{linearML\_ba\_Mat} calcualtes the maximum
  likelihood grids.
\end{itemize}

    % Add contents below.

{\par%
\vspace{-1\baselineskip}%
\needspace{4\baselineskip}}%
\begin{notebookcell}[14]%
\begin{addmargin}[\cellleftmargin]{0em}% left, right
{\smaller%
\par%
%
\vspace{-1\smallerfontscale}%
\begin{Verbatim}[commandchars=\\\{\}]
\PY{k}{def} \PY{n+nf}{bayesianML}\PY{p}{(}\PY{n}{func}\PY{p}{,} \PY{n}{x}\PY{p}{,} \PY{n}{y}\PY{p}{,} \PY{n}{dy}\PY{p}{)}\PY{p}{:}
    \PY{l+s+sd}{\PYZdq{}\PYZdq{}\PYZdq{}}
\PY{l+s+sd}{    calculate Max likelihood using bayesian analysis}
\PY{l+s+sd}{    \PYZdq{}\PYZdq{}\PYZdq{}}
    \PY{n}{Pb\PYZus{}range} \PY{o}{=} \PY{n}{np}\PY{o}{.}\PY{n}{linspace}\PY{p}{(}\PY{l+m+mi}{0}\PY{p}{,} \PY{l+m+mi}{1}\PY{p}{,} \PY{l+m+mi}{11}\PY{p}{)}
    \PY{n}{Vb\PYZus{}range} \PY{o}{=} \PY{n}{np}\PY{o}{.}\PY{n}{linspace}\PY{p}{(}\PY{l+m+mi}{0}\PY{p}{,} \PY{l+m+mi}{4000}\PY{p}{,} \PY{l+m+mi}{11}\PY{p}{)}
    \PY{n}{Yb\PYZus{}range} \PY{o}{=} \PY{n}{np}\PY{o}{.}\PY{n}{linspace}\PY{p}{(}\PY{l+m+mi}{0}\PY{p}{,} \PY{l+m+mi}{700}\PY{p}{,} \PY{l+m+mi}{11}\PY{p}{)}
    \PY{n}{PVY\PYZus{}list} \PY{o}{=} \PY{p}{[}\PY{p}{(}\PY{n}{pi}\PY{p}{,} \PY{n}{vj}\PY{p}{,} \PY{n}{yk}\PY{p}{)} \PY{k}{for} \PY{n}{pi} \PY{o+ow}{in} \PY{n}{Pb\PYZus{}range} \PY{k}{for} \PY{n}{vj} \PY{o+ow}{in} \PY{n}{Vb\PYZus{}range} \PY{k}{for} \PY{n}{yk} \PY{o+ow}{in} \PY{n}{Yb\PYZus{}range}\PY{p}{]}
    
    \PY{k}{def} \PY{n+nf}{bayesian\PYZus{}func}\PY{p}{(}\PY{n}{pb}\PY{p}{,} \PY{n}{vb}\PY{p}{,} \PY{n}{yb}\PY{p}{)}\PY{p}{:} 
        \PY{n}{np}\PY{o}{.}\PY{n}{prod}\PY{p}{(}\PY{p}{(}\PY{l+m+mi}{1} \PY{o}{\PYZhy{}} \PY{n}{pb}\PY{p}{)}\PY{o}{/}\PY{n}{np}\PY{o}{.}\PY{n}{sqrt}\PY{p}{(}\PY{l+m+mi}{2} \PY{o}{*} \PY{n}{np}\PY{o}{.}\PY{n}{pi} \PY{o}{*} \PY{n}{dy}\PY{o}{*}\PY{o}{*}\PY{l+m+mi}{2}\PY{p}{)} \PY{o}{*} \PY{n}{np}\PY{o}{.}\PY{n}{exp}\PY{p}{(}\PY{o}{\PYZhy{}}\PY{p}{(}\PY{n}{y} \PY{o}{\PYZhy{}} \PY{n}{func}\PY{p}{(}\PY{n}{x}\PY{p}{)}\PY{p}{)}\PY{o}{*}\PY{o}{*}\PY{l+m+mi}{2}\PY{o}{/}\PY{p}{(}\PY{l+m+mi}{2} \PY{o}{*} \PY{n}{dy}\PY{o}{*}\PY{o}{*}\PY{l+m+mi}{2}\PY{p}{)}\PY{p}{)} 
                \PY{o}{+} \PY{n}{pb}\PY{o}{/}\PY{n}{np}\PY{o}{.}\PY{n}{sqrt}\PY{p}{(}\PY{l+m+mi}{2} \PY{o}{*} \PY{n}{np}\PY{o}{.}\PY{n}{pi} \PY{o}{*} \PY{p}{(}\PY{n}{dy}\PY{o}{*}\PY{o}{*}\PY{l+m+mi}{2} \PY{o}{+} \PY{n}{vb}\PY{p}{)}\PY{p}{)} \PY{o}{*} \PY{n}{np}\PY{o}{.}\PY{n}{exp}\PY{p}{(}\PY{o}{\PYZhy{}}\PY{p}{(}\PY{n}{y} \PY{o}{\PYZhy{}} \PY{n}{yb}\PY{p}{)}\PY{o}{*}\PY{o}{*}\PY{l+m+mi}{2}\PY{o}{/}\PY{p}{(}\PY{l+m+mi}{2} \PY{o}{*} \PY{n}{dy}\PY{o}{*}\PY{o}{*}\PY{l+m+mi}{2}\PY{p}{)}\PY{p}{)}\PY{p}{)}
        
    \PY{k}{return} \PY{n}{np}\PY{o}{.}\PY{n}{sum}\PY{p}{(}\PY{n+nb}{map}\PY{p}{(}\PY{n}{bayesian\PYZus{}func}\PY{p}{,} \PY{n}{PVY\PYZus{}list}\PY{p}{)}\PY{p}{)}

\PY{k}{def} \PY{n+nf}{linearML\PYZus{}ba\PYZus{}Mat}\PY{p}{(}\PY{n}{m\PYZus{}min}\PY{p}{,} \PY{n}{m\PYZus{}max}\PY{p}{,} \PY{n}{b\PYZus{}min}\PY{p}{,} \PY{n}{b\PYZus{}max}\PY{p}{,} \PY{n}{x}\PY{p}{,} \PY{n}{y}\PY{p}{,} \PY{n}{dy}\PY{p}{,} \PY{n}{ngrid} \PY{o}{=} \PY{l+m+mi}{20}\PY{p}{)}\PY{p}{:}
    \PY{l+s+sd}{\PYZdq{}\PYZdq{}\PYZdq{}}
\PY{l+s+sd}{    return a 2d array that contains the ML\PYZhy{}Batesian Analysis calculation for each grid point}
\PY{l+s+sd}{    \PYZdq{}\PYZdq{}\PYZdq{}}
    \PY{n}{m\PYZus{}range} \PY{o}{=} \PY{n}{np}\PY{o}{.}\PY{n}{linspace}\PY{p}{(}\PY{n}{m\PYZus{}min}\PY{p}{,} \PY{n}{m\PYZus{}max}\PY{p}{,} \PY{n}{ngrid}\PY{p}{)}
    \PY{n}{b\PYZus{}range} \PY{o}{=} \PY{n}{np}\PY{o}{.}\PY{n}{linspace}\PY{p}{(}\PY{n}{b\PYZus{}min}\PY{p}{,} \PY{n}{b\PYZus{}max}\PY{p}{,} \PY{n}{ngrid}\PY{p}{)}
    \PY{n}{mb\PYZus{}mesh} \PY{o}{=} \PY{p}{[}\PY{p}{(}\PY{n}{mi}\PY{p}{,} \PY{n}{bj}\PY{p}{)} \PY{k}{for} \PY{n}{mi} \PY{o+ow}{in} \PY{n}{m\PYZus{}range} \PY{k}{for} \PY{n}{bj} \PY{o+ow}{in} \PY{n}{b\PYZus{}range}\PY{p}{]}\PY{c}{\PYZsh{} a list of the grid coordinates}
    \PY{n}{ba\PYZus{}ml\PYZus{}func} \PY{o}{=} \PY{k}{lambda} \PY{p}{(}\PY{n}{m}\PY{p}{,} \PY{n}{b}\PY{p}{)}\PY{p}{:} \PY{n}{bayesianML}\PY{p}{(}\PY{k}{lambda} \PY{n}{xi}\PY{p}{:} \PY{n}{m} \PY{o}{*} \PY{n}{xi} \PY{o}{+} \PY{n}{b}\PY{p}{,} \PY{n}{x}\PY{p}{,} \PY{n}{y}\PY{p}{,} \PY{n}{dy}\PY{p}{)}
    \PY{n}{ba\PYZus{}ml\PYZus{}mat} \PY{o}{=} \PY{n}{np}\PY{o}{.}\PY{n}{array}\PY{p}{(}\PY{n+nb}{map}\PY{p}{(}\PY{n}{ba\PYZus{}ml\PYZus{}func}\PY{p}{,} \PY{n}{mb\PYZus{}mesh}\PY{p}{)}\PY{p}{)}\PY{o}{.}\PY{n}{reshape}\PY{p}{(}\PY{p}{(}\PY{n+nb}{len}\PY{p}{(}\PY{n}{m\PYZus{}range}\PY{p}{)}\PY{p}{,} \PY{n+nb}{len}\PY{p}{(}\PY{n}{b\PYZus{}range}\PY{p}{)}\PY{p}{)}\PY{p}{)}
    \PY{k}{return} \PY{n}{m\PYZus{}range}\PY{p}{,} \PY{n}{b\PYZus{}range}\PY{p}{,} \PY{n}{ba\PYZus{}ml\PYZus{}mat}


\PY{n}{m\PYZus{}ba\PYZus{}range\PYZus{}1b}\PY{p}{,} \PY{n}{b\PYZus{}ba\PYZus{}range\PYZus{}1b} \PY{p}{,} \PY{n}{ba\PYZus{}ml\PYZus{}mat\PYZus{}1b} \PY{o}{=} \PY{n}{linearML\PYZus{}ba\PYZus{}Mat}\PY{p}{(}\PY{l+m+mf}{1.6}\PY{p}{,} \PY{l+m+mf}{3.2}\PY{p}{,} \PY{o}{\PYZhy{}}\PY{l+m+mi}{120}\PY{p}{,} \PY{l+m+mi}{120}\PY{p}{,}
                                                              \PY{n}{x}\PY{p}{,} \PY{n}{y}\PY{p}{,} \PY{n}{dy}\PY{p}{,} \PY{n}{ngrid} \PY{o}{=} \PY{l+m+mi}{20}\PY{p}{)}
\end{Verbatim}
%
\par%
\vspace{-1\smallerfontscale}}%
\end{addmargin}
\end{notebookcell}


    \begin{itemize}
\itemsep1pt\parskip0pt\parsep0pt
\item
  Considering the large time consuming of function
  \texttt{linearML\_ba\_Mat}, the parameters' ranges need to be
  constrained step by step. For every step, the result is plotted with
  function \texttt{plotML\_ba} defined as below.
\end{itemize}

    % Add contents below.

{\par%
\vspace{-1\baselineskip}%
\needspace{4\baselineskip}}%
\begin{notebookcell}[15]%
\begin{addmargin}[\cellleftmargin]{0em}% left, right
{\smaller%
\par%
%
\vspace{-1\smallerfontscale}%
\begin{Verbatim}[commandchars=\\\{\}]
\PY{k}{def} \PY{n+nf}{plotML\PYZus{}ba}\PY{p}{(}\PY{n}{m\PYZus{}lim}\PY{p}{,} \PY{n}{b\PYZus{}lim}\PY{p}{,} \PY{n}{x}\PY{p}{,} \PY{n}{y}\PY{p}{,} \PY{n}{dy}\PY{p}{,} \PY{n}{ngrid} \PY{o}{=} \PY{l+m+mi}{20}\PY{p}{)}\PY{p}{:}
    \PY{l+s+sd}{\PYZdq{}\PYZdq{}\PYZdq{}}
\PY{l+s+sd}{    calculate max likelihood matrix with bayesian analysis}
\PY{l+s+sd}{    plot the result and return the coordinate (m, b) that gives the max likelihood}
\PY{l+s+sd}{    \PYZdq{}\PYZdq{}\PYZdq{}}
    \PY{n}{m\PYZus{}range}\PY{p}{,} \PY{n}{b\PYZus{}range}\PY{p}{,} \PY{n}{ba\PYZus{}ml\PYZus{}mat} \PY{o}{=} \PY{n}{linearML\PYZus{}ba\PYZus{}Mat}\PY{p}{(}\PY{n}{m\PYZus{}lim}\PY{p}{[}\PY{l+m+mi}{0}\PY{p}{]}\PY{p}{,} \PY{n}{m\PYZus{}lim}\PY{p}{[}\PY{l+m+mi}{1}\PY{p}{]}\PY{p}{,} \PY{n}{b\PYZus{}lim}\PY{p}{[}\PY{l+m+mi}{0}\PY{p}{]}\PY{p}{,} \PY{n}{b\PYZus{}lim}\PY{p}{[}\PY{l+m+mi}{1}\PY{p}{]}\PY{p}{,}
                                                  \PY{n}{x}\PY{p}{,} \PY{n}{y}\PY{p}{,} \PY{n}{dy}\PY{p}{,} \PY{n}{ngrid} \PY{o}{=} \PY{n}{ngrid}\PY{p}{)}
    \PY{n}{fig}\PY{p}{,} \PY{n}{ax} \PY{o}{=} \PY{n}{plt}\PY{o}{.}\PY{n}{subplots}\PY{p}{(}\PY{p}{)}
    \PY{n}{cax} \PY{o}{=} \PY{n}{ax}\PY{o}{.}\PY{n}{pcolormesh}\PY{p}{(}\PY{n}{b\PYZus{}range}\PY{p}{,} \PY{n}{m\PYZus{}range}\PY{p}{,} \PY{n}{ba\PYZus{}ml\PYZus{}mat}\PY{p}{,} \PY{n}{cmap} \PY{o}{=} \PY{l+s}{\PYZdq{}}\PY{l+s}{Greens}\PY{l+s}{\PYZdq{}}\PY{p}{)}
    \PY{n}{fig}\PY{o}{.}\PY{n}{colorbar}\PY{p}{(}\PY{n}{cax}\PY{p}{)}
    \PY{n}{fig}\PY{o}{.}\PY{n}{tight\PYZus{}layout}\PY{p}{(}\PY{p}{)}
    \PY{n}{ax}\PY{o}{.}\PY{n}{set\PYZus{}title}\PY{p}{(}\PY{l+s}{\PYZsq{}}\PY{l+s}{Max Likelihood with Bayesian Analysis}\PY{l+s}{\PYZsq{}}\PY{p}{)}
    \PY{n}{ax}\PY{o}{.}\PY{n}{set\PYZus{}xlabel}\PY{p}{(}\PY{l+s}{\PYZsq{}}\PY{l+s}{\PYZdl{}b\PYZdl{}}\PY{l+s}{\PYZsq{}}\PY{p}{)}
    \PY{n}{ax}\PY{o}{.}\PY{n}{set\PYZus{}ylabel}\PY{p}{(}\PY{l+s}{\PYZsq{}}\PY{l+s}{\PYZdl{}m\PYZdl{}}\PY{l+s}{\PYZsq{}}\PY{p}{)}
    \PY{n}{ax}\PY{o}{.}\PY{n}{autoscale}\PY{p}{(}\PY{n}{tight} \PY{o}{=} \PY{n+nb+bp}{True}\PY{p}{)}
    \PY{n}{i}\PY{p}{,}\PY{n}{j} \PY{o}{=} \PY{n}{np}\PY{o}{.}\PY{n}{unravel\PYZus{}index}\PY{p}{(}\PY{n}{ba\PYZus{}ml\PYZus{}mat}\PY{o}{.}\PY{n}{argmax}\PY{p}{(}\PY{p}{)}\PY{p}{,} \PY{n}{ba\PYZus{}ml\PYZus{}mat}\PY{o}{.}\PY{n}{shape}\PY{p}{)} \PY{c}{\PYZsh{} get the position of the maximum}
    \PY{n}{ax}\PY{o}{.}\PY{n}{plot}\PY{p}{(}\PY{n}{b\PYZus{}range}\PY{p}{[}\PY{n}{j}\PY{p}{]}\PY{p}{,} \PY{n}{m\PYZus{}range}\PY{p}{[}\PY{n}{i}\PY{p}{]}\PY{p}{,} \PY{l+s}{\PYZsq{}}\PY{l+s}{*}\PY{l+s}{\PYZsq{}}\PY{p}{)}
    \PY{k}{return} \PY{n}{m\PYZus{}range}\PY{p}{[}\PY{n}{i}\PY{p}{]}\PY{p}{,} \PY{n}{b\PYZus{}range}\PY{p}{[}\PY{n}{j}\PY{p}{]}
    
    
\end{Verbatim}
%
\par%
\vspace{-1\smallerfontscale}}%
\end{addmargin}
\end{notebookcell}


    \subsubsection{First step, roughly define the position of the
maximum}\label{first-step-roughly-define-the-position-of-the-maximum}

\begin{itemize}
\itemsep1pt\parskip0pt\parsep0pt
\item
  $m$ from 1.6 to 3.2, $b$ from -120 to 120, 20 number of grid points on
  each dimension
\item
  The resolution for $m$ is 0.08, and for $b$ is 12
\end{itemize}

    % Add contents below.

{\par%
\vspace{-1\baselineskip}%
\needspace{4\baselineskip}}%
\begin{notebookcell}[16]%
\begin{addmargin}[\cellleftmargin]{0em}% left, right
{\smaller%
\par%
%
\vspace{-1\smallerfontscale}%
\begin{Verbatim}[commandchars=\\\{\}]
\PY{n}{m\PYZus{}max}\PY{p}{,} \PY{n}{b\PYZus{}max} \PY{o}{=} \PY{n}{plotML\PYZus{}ba}\PY{p}{(}\PY{p}{[}\PY{l+m+mf}{1.6}\PY{p}{,} \PY{l+m+mf}{3.2}\PY{p}{]}\PY{p}{,} \PY{p}{[}\PY{o}{\PYZhy{}}\PY{l+m+mi}{120}\PY{p}{,} \PY{l+m+mi}{120}\PY{p}{]}\PY{p}{,} \PY{n}{x}\PY{p}{,} \PY{n}{y}\PY{p}{,} \PY{n}{dy}\PY{p}{,} \PY{n}{ngrid} \PY{o}{=} \PY{l+m+mi}{20}\PY{p}{)}
\end{Verbatim}
%
\par%
\vspace{-1\smallerfontscale}}%
\end{addmargin}
\end{notebookcell}

\par\vspace{1\smallerfontscale}%
    \needspace{4\baselineskip}%
    % Only render the prompt if the cell is pyout.  Note, the outputs prompt 
    % block isn't used since we need to check each indiviual output and only
    % add prompts to the pyout ones.
    %
    %
    \begin{addmargin}[\cellleftmargin]{0em}% left, right
    {\smaller%
    \vspace{-1\smallerfontscale}%
    
    \begin{center}
    \adjustimage{max size={0.9\linewidth}{0.9\paperheight}}{Astro518_HW_02_files/Astro518_HW_02_26_0.png}
    \end{center}
    { \hspace*{\fill} \\}
    }%
    \end{addmargin}%
    \begin{itemize}
\itemsep1pt\parskip0pt\parsep0pt
\item
  According to the result of first step, we can find the maximum point
  is roughly in the range of $m\sim(2, 2.6)$, $b\sim (0, 60)$ \#\#\#
  Second step, precisely define the maximum position
\item
  \texttt{m\_min = 2, m\_max = 2.6, b\_min = -0, b\_max = 60, ngrid = 60}
\end{itemize}

    % Add contents below.

{\par%
\vspace{-1\baselineskip}%
\needspace{4\baselineskip}}%
\begin{notebookcell}[97]%
\begin{addmargin}[\cellleftmargin]{0em}% left, right
{\smaller%
\par%
%
\vspace{-1\smallerfontscale}%
\begin{Verbatim}[commandchars=\\\{\}]
\PY{n}{m\PYZus{}max}\PY{p}{,} \PY{n}{b\PYZus{}max} \PY{o}{=} \PY{n}{plotML\PYZus{}ba}\PY{p}{(}\PY{p}{[}\PY{l+m+mi}{2}\PY{p}{,} \PY{l+m+mf}{2.6}\PY{p}{]}\PY{p}{,} \PY{p}{[}\PY{l+m+mi}{0}\PY{p}{,} \PY{l+m+mi}{60}\PY{p}{]}\PY{p}{,} \PY{n}{x}\PY{p}{,} \PY{n}{y}\PY{p}{,} \PY{n}{dy}\PY{p}{,} \PY{n}{ngrid} \PY{o}{=} \PY{l+m+mi}{60}\PY{p}{)}

\PY{k}{print} \PY{l+s}{\PYZsq{}}\PY{l+s}{The maximum likelihood locates at:}\PY{l+s}{\PYZsq{}}
\PY{k}{print} \PY{l+s}{\PYZsq{}}\PY{l+s}{m = \PYZob{}0:.2f\PYZcb{}}\PY{l+s+se}{\PYZbs{}n}\PY{l+s}{ b = \PYZob{}1:.2f\PYZcb{}}\PY{l+s}{\PYZsq{}}\PY{o}{.}\PY{n}{format}\PY{p}{(}\PY{n}{m\PYZus{}max}\PY{p}{,} \PY{n}{b\PYZus{}max}\PY{p}{)}
\end{Verbatim}
%
\par%
\vspace{-1\smallerfontscale}}%
\end{addmargin}
\end{notebookcell}

\par\vspace{1\smallerfontscale}%
    \needspace{4\baselineskip}%
    % Only render the prompt if the cell is pyout.  Note, the outputs prompt 
    % block isn't used since we need to check each indiviual output and only
    % add prompts to the pyout ones.
    %
    %
    \begin{addmargin}[\cellleftmargin]{0em}% left, right
    {\smaller%
    \vspace{-1\smallerfontscale}%
    
    \begin{Verbatim}[commandchars=\\\{\}]
The maximum likelihood locates at:
m = 2.26
 b = 30.51
    \end{Verbatim}
}%
    \end{addmargin}%\par\vspace{1\smallerfontscale}%
    \needspace{4\baselineskip}%
    % Only render the prompt if the cell is pyout.  Note, the outputs prompt 
    % block isn't used since we need to check each indiviual output and only
    % add prompts to the pyout ones.
    %
    %
    \begin{addmargin}[\cellleftmargin]{0em}% left, right
    {\smaller%
    \vspace{-1\smallerfontscale}%
    
    \begin{center}
    \adjustimage{max size={0.9\linewidth}{0.9\paperheight}}{Astro518_HW_02_files/Astro518_HW_02_28_1.png}
    \end{center}
    { \hspace*{\fill} \\}
    }%
    \end{addmargin}%
    \subsection{the comparison}\label{the-comparison}

\begin{itemize}
\itemsep1pt\parskip0pt\parsep0pt
\item
  The result of 3.a is very close to the result of 1.a, and different
  from that of 1.b
\item
  With Bayesian analysis, the 4 points that were ignored in 1.a have
  larger probibilities to be bad points. Thus these points cannot skew
  the fitted line very much
\item
  \textbf{3.a is the best}. In 1.a, we brutally excludes 4 points that
  are not belong the linear relation as we expected. This will introduce
  a bias. In 1.b, the fitting is largely skewed by several points, and
  the linear relation in 1.b is not prominent. Bayesian analysis reduces
  the influence of outlier with prior assumption. The better the
  probability distribution of being a bad point is known, the more
  accurate the fitting result will be.
\end{itemize}

    \section{Problem 4, The uncertainties of fitting parameter,
\textbf{\emph{bootstrap}}}\label{problem-4-the-uncertainties-of-fitting-parameter-bootstrap}

\begin{itemize}
\itemsep1pt\parskip0pt\parsep0pt
\item
  Following function \texttt{bootstrap} caculates the values and
  uncertainties of $m$ and $b$ with bootstrap method.
\end{itemize}

    % Add contents below.

{\par%
\vspace{-1\baselineskip}%
\needspace{4\baselineskip}}%
\begin{notebookcell}[]%
\begin{addmargin}[\cellleftmargin]{0em}% left, right
{\smaller%
\par%
%
\vspace{-1\smallerfontscale}%
\begin{Verbatim}[commandchars=\\\{\}]
\PY{k}{def} \PY{n+nf}{bootstrap}\PY{p}{(}\PY{n}{niter}\PY{p}{,} \PY{n}{x}\PY{p}{,} \PY{n}{y}\PY{p}{,} \PY{n}{dy}\PY{p}{)}\PY{p}{:}
    \PY{l+s+sd}{\PYZdq{}\PYZdq{}\PYZdq{}}
\PY{l+s+sd}{    calculate the fitting result using Bayesian max likelihood}
\PY{l+s+sd}{    input parameter:}
\PY{l+s+sd}{        iteration time}
\PY{l+s+sd}{        x, y, dy}
\PY{l+s+sd}{    output parameter:}
\PY{l+s+sd}{        fitting result list}
\PY{l+s+sd}{    \PYZdq{}\PYZdq{}\PYZdq{}}
    \PY{n}{result\PYZus{}list} \PY{o}{=} \PY{p}{[}\PY{p}{]}
    \PY{k}{for} \PY{n}{i0} \PY{o+ow}{in} \PY{n+nb}{range}\PY{p}{(}\PY{n}{niter}\PY{p}{)}\PY{p}{:}
        \PY{c}{\PYZsh{} numpy provides random module that can generate random integers in a given interval}
        \PY{c}{\PYZsh{} this function uses these random numbers to index the data }
        \PY{n}{index} \PY{o}{=} \PY{n}{np}\PY{o}{.}\PY{n}{random}\PY{o}{.}\PY{n}{random\PYZus{}integers}\PY{p}{(}\PY{l+m+mi}{0}\PY{p}{,} \PY{n+nb}{len}\PY{p}{(}\PY{n}{x}\PY{p}{)} \PY{o}{\PYZhy{}} \PY{l+m+mi}{1}\PY{p}{,} \PY{n+nb}{len}\PY{p}{(}\PY{n}{x}\PY{p}{)} \PY{o}{\PYZhy{}} \PY{l+m+mi}{1}\PY{p}{)}
        \PY{n}{xi0} \PY{o}{=} \PY{n}{x}\PY{p}{[}\PY{n}{index}\PY{p}{]}
        \PY{n}{yi0} \PY{o}{=} \PY{n}{y}\PY{p}{[}\PY{n}{index}\PY{p}{]}
        \PY{n}{dyi0} \PY{o}{=} \PY{n}{dy}\PY{p}{[}\PY{n}{index}\PY{p}{]}
        \PY{n}{m\PYZus{}range}\PY{p}{,} \PY{n}{b\PYZus{}range}\PY{p}{,} \PY{n}{ba\PYZus{}ml\PYZus{}mat} \PY{o}{=} \PY{n}{linearML\PYZus{}ba\PYZus{}Mat}\PY{p}{(}\PY{l+m+mf}{1.8}\PY{p}{,} \PY{l+m+mf}{2.8}\PY{p}{,} \PY{o}{\PYZhy{}}\PY{l+m+mi}{50}\PY{p}{,} \PY{l+m+mi}{100}\PY{p}{,} \PY{n}{xi0}\PY{p}{,} \PY{n}{yi0}\PY{p}{,} \PY{n}{dyi0}\PY{p}{,} \PY{n}{ngrid} \PY{o}{=} \PY{l+m+mi}{20}\PY{p}{)}
        \PY{n}{i}\PY{p}{,}\PY{n}{j} \PY{o}{=} \PY{n}{np}\PY{o}{.}\PY{n}{unravel\PYZus{}index}\PY{p}{(}\PY{n}{ba\PYZus{}ml\PYZus{}mat}\PY{o}{.}\PY{n}{argmax}\PY{p}{(}\PY{p}{)}\PY{p}{,} \PY{n}{ba\PYZus{}ml\PYZus{}mat}\PY{o}{.}\PY{n}{shape}\PY{p}{)}
        \PY{n}{result\PYZus{}list}\PY{o}{.}\PY{n}{append}\PY{p}{(}\PY{p}{(}\PY{n}{m\PYZus{}range}\PY{p}{[}\PY{n}{i}\PY{p}{]}\PY{p}{,} \PY{n}{b\PYZus{}range}\PY{p}{[}\PY{n}{j}\PY{p}{]}\PY{p}{)}\PY{p}{)}
        \PY{k}{print} \PY{l+s}{\PYZsq{}}\PY{l+s}{\PYZob{}0\PYZcb{} step finished}\PY{l+s}{\PYZsq{}}\PY{o}{.}\PY{n}{format}\PY{p}{(}\PY{n}{i0}\PY{p}{)}
        
    
    \PY{k}{return} \PY{n}{result\PYZus{}list}

\PY{n}{mb\PYZus{}bootstrap} \PY{o}{=} \PY{n}{bootstrap}\PY{p}{(}\PY{l+m+mi}{1000}\PY{p}{,} \PY{n}{x}\PY{p}{,} \PY{n}{y}\PY{p}{,} \PY{n}{dy}\PY{p}{)}
\end{Verbatim}
%
\par%
\vspace{-1\smallerfontscale}}%
\end{addmargin}
\end{notebookcell}


    % Add contents below.

{\par%
\vspace{-1\baselineskip}%
\needspace{4\baselineskip}}%
\begin{notebookcell}[90]%
\begin{addmargin}[\cellleftmargin]{0em}% left, right
{\smaller%
\par%
%
\vspace{-1\smallerfontscale}%
\begin{Verbatim}[commandchars=\\\{\}]
\PY{k+kn}{from} \PY{n+nn}{scipy.optimize} \PY{k+kn}{import} \PY{n}{curve\PYZus{}fit}
\PY{n}{m\PYZus{}bstrp} \PY{o}{=} \PY{n+nb}{map}\PY{p}{(}\PY{k}{lambda} \PY{n}{item}\PY{p}{:} \PY{n}{item}\PY{p}{[}\PY{l+m+mi}{0}\PY{p}{]}\PY{p}{,} \PY{n}{mb\PYZus{}bootstrap}\PY{p}{)}
\PY{n}{b\PYZus{}bstrp} \PY{o}{=} \PY{n+nb}{map}\PY{p}{(}\PY{k}{lambda} \PY{n}{item}\PY{p}{:} \PY{n}{item}\PY{p}{[}\PY{l+m+mi}{1}\PY{p}{]}\PY{p}{,} \PY{n}{mb\PYZus{}bootstrap}\PY{p}{)}

\PY{k}{def} \PY{n+nf}{plot\PYZus{}bootstrap}\PY{p}{(}\PY{n}{bstrp}\PY{p}{,} \PY{n}{nbins} \PY{o}{=} \PY{l+m+mi}{20}\PY{p}{,} \PY{n}{p0} \PY{o}{=} \PY{p}{[}\PY{l+m+mf}{1.}\PY{p}{,} \PY{l+m+mf}{0.}\PY{p}{,} \PY{l+m+mf}{1.}\PY{p}{]}\PY{p}{)}\PY{p}{:}
    \PY{c}{\PYZsh{} p0 is the intial values for gaussian fitting}
    \PY{l+s+sd}{\PYZdq{}\PYZdq{}\PYZdq{}}
\PY{l+s+sd}{    plot the bootstrap result as a histogram,}
\PY{l+s+sd}{    fit it with a gaussian curve, and return the fitting result}
\PY{l+s+sd}{    \PYZdq{}\PYZdq{}\PYZdq{}}
    \PY{n}{fig} \PY{o}{=} \PY{n}{plt}\PY{o}{.}\PY{n}{figure}\PY{p}{(}\PY{p}{)}
    \PY{n}{ax} \PY{o}{=} \PY{n}{fig}\PY{o}{.}\PY{n}{add\PYZus{}subplot}\PY{p}{(}\PY{l+m+mi}{111}\PY{p}{)}
    \PY{n}{n}\PY{p}{,} \PY{n}{bins}\PY{p}{,} \PY{n}{patches} \PY{o}{=} \PY{n}{ax}\PY{o}{.}\PY{n}{hist}\PY{p}{(}\PY{n}{bstrp}\PY{p}{,} \PY{n}{bins} \PY{o}{=} \PY{n}{nbins}\PY{p}{,} \PY{n}{alpha} \PY{o}{=} \PY{l+m+mf}{0.9}\PY{p}{)}
    \PY{n}{bin\PYZus{}centers} \PY{o}{=} \PY{p}{(}\PY{n}{bins}\PY{p}{[}\PY{p}{:}\PY{o}{\PYZhy{}}\PY{l+m+mi}{1}\PY{p}{]} \PY{o}{+} \PY{n}{bins}\PY{p}{[}\PY{l+m+mi}{1}\PY{p}{:}\PY{p}{]}\PY{p}{)}\PY{o}{/}\PY{l+m+mf}{2.}
    \PY{n}{ax}\PY{o}{.}\PY{n}{plot}\PY{p}{(}\PY{n}{bin\PYZus{}centers}\PY{p}{,} \PY{n}{n}\PY{p}{,} \PY{l+s}{\PYZsq{}}\PY{l+s}{+}\PY{l+s}{\PYZsq{}}\PY{p}{,} \PY{n}{mew} \PY{o}{=} \PY{l+s}{\PYZsq{}}\PY{l+s}{1}\PY{l+s}{\PYZsq{}}\PY{p}{,} \PY{n}{ms} \PY{o}{=} \PY{l+m+mi}{10}\PY{p}{)}

    \PY{k}{def} \PY{n+nf}{gauss}\PY{p}{(}\PY{n}{x}\PY{p}{,} \PY{o}{*}\PY{n}{p}\PY{p}{)}\PY{p}{:}\PY{c}{\PYZsh{} define a gaussian function for fitting}
        \PY{n}{A}\PY{p}{,} \PY{n}{mu}\PY{p}{,} \PY{n}{sigma} \PY{o}{=} \PY{n}{p}
        \PY{k}{return} \PY{n}{A}\PY{o}{*}\PY{n}{numpy}\PY{o}{.}\PY{n}{exp}\PY{p}{(}\PY{o}{\PYZhy{}}\PY{p}{(}\PY{n}{x}\PY{o}{\PYZhy{}}\PY{n}{mu}\PY{p}{)}\PY{o}{*}\PY{o}{*}\PY{l+m+mi}{2}\PY{o}{/}\PY{p}{(}\PY{l+m+mf}{2.}\PY{o}{*}\PY{n}{sigma}\PY{o}{*}\PY{o}{*}\PY{l+m+mi}{2}\PY{p}{)}\PY{p}{)}

    \PY{n}{para}\PY{p}{,} \PY{n}{var\PYZus{}matrix} \PY{o}{=} \PY{n}{curve\PYZus{}fit}\PY{p}{(}\PY{n}{gauss}\PY{p}{,} \PY{n}{bin\PYZus{}centers}\PY{p}{,} \PY{n}{n}\PY{p}{,} \PY{n}{p0} \PY{o}{=} \PY{n}{p0}\PY{p}{)}
    \PY{n}{x\PYZus{}sample} \PY{o}{=} \PY{n}{np}\PY{o}{.}\PY{n}{linspace}\PY{p}{(}\PY{n}{np}\PY{o}{.}\PY{n}{min}\PY{p}{(}\PY{n}{bins}\PY{p}{)}\PY{p}{,}\PY{n}{np}\PY{o}{.}\PY{n}{max}\PY{p}{(}\PY{n}{bins}\PY{p}{)}\PY{p}{,} \PY{l+m+mi}{200}\PY{p}{)}
    \PY{n}{n\PYZus{}fit} \PY{o}{=} \PY{n}{gauss}\PY{p}{(}\PY{n}{x\PYZus{}sample}\PY{p}{,} \PY{o}{*}\PY{n}{para}\PY{p}{)}

    \PY{n}{ax}\PY{o}{.}\PY{n}{plot}\PY{p}{(}\PY{n}{x\PYZus{}sample}\PY{p}{,} \PY{n}{n\PYZus{}fit}\PY{p}{,} \PY{l+s}{\PYZsq{}}\PY{l+s}{\PYZhy{}}\PY{l+s}{\PYZsq{}}\PY{p}{)} \PY{c}{\PYZsh{} plot the gaussian fitting result}
    \PY{k}{return} \PY{n}{para}\PY{p}{,} \PY{n}{fig}

\PY{n}{m\PYZus{}bstrp\PYZus{}fit}\PY{p}{,} \PY{n}{m\PYZus{}fig} \PY{o}{=} \PY{n}{plot\PYZus{}bootstrap}\PY{p}{(}\PY{n}{m\PYZus{}bstrp}\PY{p}{,} \PY{n}{p0} \PY{o}{=} \PY{p}{[}\PY{l+m+mi}{200}\PY{p}{,} \PY{l+m+mf}{2.2}\PY{p}{,} \PY{l+m+mf}{0.2}\PY{p}{]}\PY{p}{)}
\PY{n}{m\PYZus{}ax} \PY{o}{=} \PY{n}{m\PYZus{}fig}\PY{o}{.}\PY{n}{axes}\PY{p}{[}\PY{l+m+mi}{0}\PY{p}{]}
\PY{n}{m\PYZus{}ax}\PY{o}{.}\PY{n}{set\PYZus{}title}\PY{p}{(}\PY{l+s}{\PYZsq{}}\PY{l+s}{Bootstrap histogram for \PYZdl{}m\PYZdl{}}\PY{l+s}{\PYZsq{}}\PY{p}{)}
\PY{n}{m\PYZus{}ax}\PY{o}{.}\PY{n}{set\PYZus{}xlabel}\PY{p}{(}\PY{l+s}{\PYZsq{}}\PY{l+s}{\PYZdl{}m\PYZdl{}}\PY{l+s}{\PYZsq{}}\PY{p}{)}
\PY{n}{m\PYZus{}ax}\PY{o}{.}\PY{n}{set\PYZus{}ylabel}\PY{p}{(}\PY{l+s}{\PYZsq{}}\PY{l+s}{\PYZdl{}n\PYZdl{}}\PY{l+s}{\PYZsq{}}\PY{p}{)}

\PY{n}{b\PYZus{}bstrp\PYZus{}fit}\PY{p}{,} \PY{n}{b\PYZus{}fig} \PY{o}{=} \PY{n}{plot\PYZus{}bootstrap}\PY{p}{(}\PY{n}{b\PYZus{}bstrp}\PY{p}{,} \PY{n}{p0} \PY{o}{=} \PY{p}{[}\PY{l+m+mi}{200}\PY{p}{,} \PY{l+m+mi}{30}\PY{p}{,} \PY{l+m+mi}{10}\PY{p}{]}\PY{p}{)}
\PY{n}{b\PYZus{}ax} \PY{o}{=} \PY{n}{b\PYZus{}fig}\PY{o}{.}\PY{n}{axes}\PY{p}{[}\PY{l+m+mi}{0}\PY{p}{]}
\PY{n}{b\PYZus{}ax}\PY{o}{.}\PY{n}{set\PYZus{}title}\PY{p}{(}\PY{l+s}{\PYZsq{}}\PY{l+s}{Bootstrap histogram for \PYZdl{}b\PYZdl{}}\PY{l+s}{\PYZsq{}}\PY{p}{)}
\PY{n}{b\PYZus{}ax}\PY{o}{.}\PY{n}{set\PYZus{}xlabel}\PY{p}{(}\PY{l+s}{\PYZsq{}}\PY{l+s}{\PYZdl{}b\PYZdl{}}\PY{l+s}{\PYZsq{}}\PY{p}{)}
\PY{n}{b\PYZus{}ax}\PY{o}{.}\PY{n}{set\PYZus{}ylabel}\PY{p}{(}\PY{l+s}{\PYZsq{}}\PY{l+s}{\PYZdl{}n\PYZdl{}}\PY{l+s}{\PYZsq{}}\PY{p}{)}

\PY{k}{print} \PY{l+s}{\PYZsq{}}\PY{l+s}{According to bootstraping and the fitting result:}\PY{l+s}{\PYZsq{}}
\PY{k}{print} \PY{l+s}{\PYZsq{}}\PY{l+s}{m = \PYZob{}0:.2f\PYZcb{}, sigma\PYZus{}m = \PYZob{}1:.2f\PYZcb{}}\PY{l+s}{\PYZsq{}}\PY{o}{.}\PY{n}{format}\PY{p}{(}\PY{n}{m\PYZus{}bstrp\PYZus{}fit}\PY{p}{[}\PY{l+m+mi}{1}\PY{p}{]}\PY{p}{,} \PY{n}{m\PYZus{}bstrp\PYZus{}fit}\PY{p}{[}\PY{l+m+mi}{2}\PY{p}{]}\PY{p}{)}
\PY{k}{print} \PY{l+s}{\PYZsq{}}\PY{l+s}{b = \PYZob{}0:.2f\PYZcb{}, sigma\PYZus{}b = \PYZob{}1:.2f\PYZcb{}}\PY{l+s}{\PYZsq{}}\PY{o}{.}\PY{n}{format}\PY{p}{(}\PY{n}{b\PYZus{}bstrp\PYZus{}fit}\PY{p}{[}\PY{l+m+mi}{1}\PY{p}{]}\PY{p}{,} \PY{n}{b\PYZus{}bstrp\PYZus{}fit}\PY{p}{[}\PY{l+m+mi}{2}\PY{p}{]}\PY{p}{)}
\end{Verbatim}
%
\par%
\vspace{-1\smallerfontscale}}%
\end{addmargin}
\end{notebookcell}

\par\vspace{1\smallerfontscale}%
    \needspace{4\baselineskip}%
    % Only render the prompt if the cell is pyout.  Note, the outputs prompt 
    % block isn't used since we need to check each indiviual output and only
    % add prompts to the pyout ones.
    %
    %
    \begin{addmargin}[\cellleftmargin]{0em}% left, right
    {\smaller%
    \vspace{-1\smallerfontscale}%
    
    \begin{Verbatim}[commandchars=\\\{\}]
According to bootstraping and the fitting result:
m = 2.25, sigma\_m = 0.07
b = 35.34, sigma\_b = 10.24
    \end{Verbatim}
}%
    \end{addmargin}%\par\vspace{1\smallerfontscale}%
    \needspace{4\baselineskip}%
    % Only render the prompt if the cell is pyout.  Note, the outputs prompt 
    % block isn't used since we need to check each indiviual output and only
    % add prompts to the pyout ones.
    %
    %
    \begin{addmargin}[\cellleftmargin]{0em}% left, right
    {\smaller%
    \vspace{-1\smallerfontscale}%
    
    \begin{center}
    \adjustimage{max size={0.9\linewidth}{0.9\paperheight}}{Astro518_HW_02_files/Astro518_HW_02_32_1.png}
    \end{center}
    { \hspace*{\fill} \\}
    }%
    \end{addmargin}%\par\vspace{1\smallerfontscale}%
    \needspace{4\baselineskip}%
    % Only render the prompt if the cell is pyout.  Note, the outputs prompt 
    % block isn't used since we need to check each indiviual output and only
    % add prompts to the pyout ones.
    %
    %
    \begin{addmargin}[\cellleftmargin]{0em}% left, right
    {\smaller%
    \vspace{-1\smallerfontscale}%
    
    \begin{center}
    \adjustimage{max size={0.9\linewidth}{0.9\paperheight}}{Astro518_HW_02_files/Astro518_HW_02_32_2.png}
    \end{center}
    { \hspace*{\fill} \\}
    }%
    \end{addmargin}%
    % Add contents below.

{\par%
\vspace{-1\baselineskip}%
\needspace{4\baselineskip}}%
\begin{notebookcell}[]%
\begin{addmargin}[\cellleftmargin]{0em}% left, right
{\smaller%
\par%
%
\vspace{-1\smallerfontscale}%
\begin{Verbatim}[commandchars=\\\{\}]

\end{Verbatim}
%
\par%
\vspace{-1\smallerfontscale}}%
\end{addmargin}
\end{notebookcell}



    % Add a bibliography block to the postdoc
    
    
    
    \end{document}
