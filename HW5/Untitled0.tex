
% Default to the notebook output style

    


% Inherit from the specified cell style.




    
\documentclass{article}

    
    
    \usepackage{graphicx} % Used to insert images
    \usepackage{adjustbox} % Used to constrain images to a maximum size 
    \usepackage{color} % Allow colors to be defined
    \usepackage{enumerate} % Needed for markdown enumerations to work
    \usepackage{geometry} % Used to adjust the document margins
    \usepackage{amsmath} % Equations
    \usepackage{amssymb} % Equations
    \usepackage[mathletters]{ucs} % Extended unicode (utf-8) support
    \usepackage[utf8x]{inputenc} % Allow utf-8 characters in the tex document
    \usepackage{fancyvrb} % verbatim replacement that allows latex
    \usepackage{grffile} % extends the file name processing of package graphics 
                         % to support a larger range 
    % The hyperref package gives us a pdf with properly built
    % internal navigation ('pdf bookmarks' for the table of contents,
    % internal cross-reference links, web links for URLs, etc.)
    \usepackage{hyperref}
    \usepackage{longtable} % longtable support required by pandoc >1.10
    \usepackage{booktabs}  % table support for pandoc > 1.12.2
    

    
    
    \definecolor{orange}{cmyk}{0,0.4,0.8,0.2}
    \definecolor{darkorange}{rgb}{.71,0.21,0.01}
    \definecolor{darkgreen}{rgb}{.12,.54,.11}
    \definecolor{myteal}{rgb}{.26, .44, .56}
    \definecolor{gray}{gray}{0.45}
    \definecolor{lightgray}{gray}{.95}
    \definecolor{mediumgray}{gray}{.8}
    \definecolor{inputbackground}{rgb}{.95, .95, .85}
    \definecolor{outputbackground}{rgb}{.95, .95, .95}
    \definecolor{traceback}{rgb}{1, .95, .95}
    % ansi colors
    \definecolor{red}{rgb}{.6,0,0}
    \definecolor{green}{rgb}{0,.65,0}
    \definecolor{brown}{rgb}{0.6,0.6,0}
    \definecolor{blue}{rgb}{0,.145,.698}
    \definecolor{purple}{rgb}{.698,.145,.698}
    \definecolor{cyan}{rgb}{0,.698,.698}
    \definecolor{lightgray}{gray}{0.5}
    
    % bright ansi colors
    \definecolor{darkgray}{gray}{0.25}
    \definecolor{lightred}{rgb}{1.0,0.39,0.28}
    \definecolor{lightgreen}{rgb}{0.48,0.99,0.0}
    \definecolor{lightblue}{rgb}{0.53,0.81,0.92}
    \definecolor{lightpurple}{rgb}{0.87,0.63,0.87}
    \definecolor{lightcyan}{rgb}{0.5,1.0,0.83}
    
    % commands and environments needed by pandoc snippets
    % extracted from the output of `pandoc -s`
    \DefineVerbatimEnvironment{Highlighting}{Verbatim}{commandchars=\\\{\}}
    % Add ',fontsize=\small' for more characters per line
    \newenvironment{Shaded}{}{}
    \newcommand{\KeywordTok}[1]{\textcolor[rgb]{0.00,0.44,0.13}{\textbf{{#1}}}}
    \newcommand{\DataTypeTok}[1]{\textcolor[rgb]{0.56,0.13,0.00}{{#1}}}
    \newcommand{\DecValTok}[1]{\textcolor[rgb]{0.25,0.63,0.44}{{#1}}}
    \newcommand{\BaseNTok}[1]{\textcolor[rgb]{0.25,0.63,0.44}{{#1}}}
    \newcommand{\FloatTok}[1]{\textcolor[rgb]{0.25,0.63,0.44}{{#1}}}
    \newcommand{\CharTok}[1]{\textcolor[rgb]{0.25,0.44,0.63}{{#1}}}
    \newcommand{\StringTok}[1]{\textcolor[rgb]{0.25,0.44,0.63}{{#1}}}
    \newcommand{\CommentTok}[1]{\textcolor[rgb]{0.38,0.63,0.69}{\textit{{#1}}}}
    \newcommand{\OtherTok}[1]{\textcolor[rgb]{0.00,0.44,0.13}{{#1}}}
    \newcommand{\AlertTok}[1]{\textcolor[rgb]{1.00,0.00,0.00}{\textbf{{#1}}}}
    \newcommand{\FunctionTok}[1]{\textcolor[rgb]{0.02,0.16,0.49}{{#1}}}
    \newcommand{\RegionMarkerTok}[1]{{#1}}
    \newcommand{\ErrorTok}[1]{\textcolor[rgb]{1.00,0.00,0.00}{\textbf{{#1}}}}
    \newcommand{\NormalTok}[1]{{#1}}
    
    % Define a nice break command that doesn't care if a line doesn't already
    % exist.
    \def\br{\hspace*{\fill} \\* }
    % Math Jax compatability definitions
    \def\gt{>}
    \def\lt{<}
    % Document parameters
    \title{Untitled0}
    
    
    

    % Pygments definitions
    
\makeatletter
\def\PY@reset{\let\PY@it=\relax \let\PY@bf=\relax%
    \let\PY@ul=\relax \let\PY@tc=\relax%
    \let\PY@bc=\relax \let\PY@ff=\relax}
\def\PY@tok#1{\csname PY@tok@#1\endcsname}
\def\PY@toks#1+{\ifx\relax#1\empty\else%
    \PY@tok{#1}\expandafter\PY@toks\fi}
\def\PY@do#1{\PY@bc{\PY@tc{\PY@ul{%
    \PY@it{\PY@bf{\PY@ff{#1}}}}}}}
\def\PY#1#2{\PY@reset\PY@toks#1+\relax+\PY@do{#2}}

\expandafter\def\csname PY@tok@gd\endcsname{\def\PY@tc##1{\textcolor[rgb]{0.63,0.00,0.00}{##1}}}
\expandafter\def\csname PY@tok@gu\endcsname{\let\PY@bf=\textbf\def\PY@tc##1{\textcolor[rgb]{0.50,0.00,0.50}{##1}}}
\expandafter\def\csname PY@tok@gt\endcsname{\def\PY@tc##1{\textcolor[rgb]{0.00,0.27,0.87}{##1}}}
\expandafter\def\csname PY@tok@gs\endcsname{\let\PY@bf=\textbf}
\expandafter\def\csname PY@tok@gr\endcsname{\def\PY@tc##1{\textcolor[rgb]{1.00,0.00,0.00}{##1}}}
\expandafter\def\csname PY@tok@cm\endcsname{\let\PY@it=\textit\def\PY@tc##1{\textcolor[rgb]{0.25,0.50,0.50}{##1}}}
\expandafter\def\csname PY@tok@vg\endcsname{\def\PY@tc##1{\textcolor[rgb]{0.10,0.09,0.49}{##1}}}
\expandafter\def\csname PY@tok@m\endcsname{\def\PY@tc##1{\textcolor[rgb]{0.40,0.40,0.40}{##1}}}
\expandafter\def\csname PY@tok@mh\endcsname{\def\PY@tc##1{\textcolor[rgb]{0.40,0.40,0.40}{##1}}}
\expandafter\def\csname PY@tok@go\endcsname{\def\PY@tc##1{\textcolor[rgb]{0.53,0.53,0.53}{##1}}}
\expandafter\def\csname PY@tok@ge\endcsname{\let\PY@it=\textit}
\expandafter\def\csname PY@tok@vc\endcsname{\def\PY@tc##1{\textcolor[rgb]{0.10,0.09,0.49}{##1}}}
\expandafter\def\csname PY@tok@il\endcsname{\def\PY@tc##1{\textcolor[rgb]{0.40,0.40,0.40}{##1}}}
\expandafter\def\csname PY@tok@cs\endcsname{\let\PY@it=\textit\def\PY@tc##1{\textcolor[rgb]{0.25,0.50,0.50}{##1}}}
\expandafter\def\csname PY@tok@cp\endcsname{\def\PY@tc##1{\textcolor[rgb]{0.74,0.48,0.00}{##1}}}
\expandafter\def\csname PY@tok@gi\endcsname{\def\PY@tc##1{\textcolor[rgb]{0.00,0.63,0.00}{##1}}}
\expandafter\def\csname PY@tok@gh\endcsname{\let\PY@bf=\textbf\def\PY@tc##1{\textcolor[rgb]{0.00,0.00,0.50}{##1}}}
\expandafter\def\csname PY@tok@ni\endcsname{\let\PY@bf=\textbf\def\PY@tc##1{\textcolor[rgb]{0.60,0.60,0.60}{##1}}}
\expandafter\def\csname PY@tok@nl\endcsname{\def\PY@tc##1{\textcolor[rgb]{0.63,0.63,0.00}{##1}}}
\expandafter\def\csname PY@tok@nn\endcsname{\let\PY@bf=\textbf\def\PY@tc##1{\textcolor[rgb]{0.00,0.00,1.00}{##1}}}
\expandafter\def\csname PY@tok@no\endcsname{\def\PY@tc##1{\textcolor[rgb]{0.53,0.00,0.00}{##1}}}
\expandafter\def\csname PY@tok@na\endcsname{\def\PY@tc##1{\textcolor[rgb]{0.49,0.56,0.16}{##1}}}
\expandafter\def\csname PY@tok@nb\endcsname{\def\PY@tc##1{\textcolor[rgb]{0.00,0.50,0.00}{##1}}}
\expandafter\def\csname PY@tok@nc\endcsname{\let\PY@bf=\textbf\def\PY@tc##1{\textcolor[rgb]{0.00,0.00,1.00}{##1}}}
\expandafter\def\csname PY@tok@nd\endcsname{\def\PY@tc##1{\textcolor[rgb]{0.67,0.13,1.00}{##1}}}
\expandafter\def\csname PY@tok@ne\endcsname{\let\PY@bf=\textbf\def\PY@tc##1{\textcolor[rgb]{0.82,0.25,0.23}{##1}}}
\expandafter\def\csname PY@tok@nf\endcsname{\def\PY@tc##1{\textcolor[rgb]{0.00,0.00,1.00}{##1}}}
\expandafter\def\csname PY@tok@si\endcsname{\let\PY@bf=\textbf\def\PY@tc##1{\textcolor[rgb]{0.73,0.40,0.53}{##1}}}
\expandafter\def\csname PY@tok@s2\endcsname{\def\PY@tc##1{\textcolor[rgb]{0.73,0.13,0.13}{##1}}}
\expandafter\def\csname PY@tok@vi\endcsname{\def\PY@tc##1{\textcolor[rgb]{0.10,0.09,0.49}{##1}}}
\expandafter\def\csname PY@tok@nt\endcsname{\let\PY@bf=\textbf\def\PY@tc##1{\textcolor[rgb]{0.00,0.50,0.00}{##1}}}
\expandafter\def\csname PY@tok@nv\endcsname{\def\PY@tc##1{\textcolor[rgb]{0.10,0.09,0.49}{##1}}}
\expandafter\def\csname PY@tok@s1\endcsname{\def\PY@tc##1{\textcolor[rgb]{0.73,0.13,0.13}{##1}}}
\expandafter\def\csname PY@tok@sh\endcsname{\def\PY@tc##1{\textcolor[rgb]{0.73,0.13,0.13}{##1}}}
\expandafter\def\csname PY@tok@sc\endcsname{\def\PY@tc##1{\textcolor[rgb]{0.73,0.13,0.13}{##1}}}
\expandafter\def\csname PY@tok@sx\endcsname{\def\PY@tc##1{\textcolor[rgb]{0.00,0.50,0.00}{##1}}}
\expandafter\def\csname PY@tok@bp\endcsname{\def\PY@tc##1{\textcolor[rgb]{0.00,0.50,0.00}{##1}}}
\expandafter\def\csname PY@tok@c1\endcsname{\let\PY@it=\textit\def\PY@tc##1{\textcolor[rgb]{0.25,0.50,0.50}{##1}}}
\expandafter\def\csname PY@tok@kc\endcsname{\let\PY@bf=\textbf\def\PY@tc##1{\textcolor[rgb]{0.00,0.50,0.00}{##1}}}
\expandafter\def\csname PY@tok@c\endcsname{\let\PY@it=\textit\def\PY@tc##1{\textcolor[rgb]{0.25,0.50,0.50}{##1}}}
\expandafter\def\csname PY@tok@mf\endcsname{\def\PY@tc##1{\textcolor[rgb]{0.40,0.40,0.40}{##1}}}
\expandafter\def\csname PY@tok@err\endcsname{\def\PY@bc##1{\setlength{\fboxsep}{0pt}\fcolorbox[rgb]{1.00,0.00,0.00}{1,1,1}{\strut ##1}}}
\expandafter\def\csname PY@tok@kd\endcsname{\let\PY@bf=\textbf\def\PY@tc##1{\textcolor[rgb]{0.00,0.50,0.00}{##1}}}
\expandafter\def\csname PY@tok@ss\endcsname{\def\PY@tc##1{\textcolor[rgb]{0.10,0.09,0.49}{##1}}}
\expandafter\def\csname PY@tok@sr\endcsname{\def\PY@tc##1{\textcolor[rgb]{0.73,0.40,0.53}{##1}}}
\expandafter\def\csname PY@tok@mo\endcsname{\def\PY@tc##1{\textcolor[rgb]{0.40,0.40,0.40}{##1}}}
\expandafter\def\csname PY@tok@kn\endcsname{\let\PY@bf=\textbf\def\PY@tc##1{\textcolor[rgb]{0.00,0.50,0.00}{##1}}}
\expandafter\def\csname PY@tok@mi\endcsname{\def\PY@tc##1{\textcolor[rgb]{0.40,0.40,0.40}{##1}}}
\expandafter\def\csname PY@tok@gp\endcsname{\let\PY@bf=\textbf\def\PY@tc##1{\textcolor[rgb]{0.00,0.00,0.50}{##1}}}
\expandafter\def\csname PY@tok@o\endcsname{\def\PY@tc##1{\textcolor[rgb]{0.40,0.40,0.40}{##1}}}
\expandafter\def\csname PY@tok@kr\endcsname{\let\PY@bf=\textbf\def\PY@tc##1{\textcolor[rgb]{0.00,0.50,0.00}{##1}}}
\expandafter\def\csname PY@tok@s\endcsname{\def\PY@tc##1{\textcolor[rgb]{0.73,0.13,0.13}{##1}}}
\expandafter\def\csname PY@tok@kp\endcsname{\def\PY@tc##1{\textcolor[rgb]{0.00,0.50,0.00}{##1}}}
\expandafter\def\csname PY@tok@w\endcsname{\def\PY@tc##1{\textcolor[rgb]{0.73,0.73,0.73}{##1}}}
\expandafter\def\csname PY@tok@kt\endcsname{\def\PY@tc##1{\textcolor[rgb]{0.69,0.00,0.25}{##1}}}
\expandafter\def\csname PY@tok@ow\endcsname{\let\PY@bf=\textbf\def\PY@tc##1{\textcolor[rgb]{0.67,0.13,1.00}{##1}}}
\expandafter\def\csname PY@tok@sb\endcsname{\def\PY@tc##1{\textcolor[rgb]{0.73,0.13,0.13}{##1}}}
\expandafter\def\csname PY@tok@k\endcsname{\let\PY@bf=\textbf\def\PY@tc##1{\textcolor[rgb]{0.00,0.50,0.00}{##1}}}
\expandafter\def\csname PY@tok@se\endcsname{\let\PY@bf=\textbf\def\PY@tc##1{\textcolor[rgb]{0.73,0.40,0.13}{##1}}}
\expandafter\def\csname PY@tok@sd\endcsname{\let\PY@it=\textit\def\PY@tc##1{\textcolor[rgb]{0.73,0.13,0.13}{##1}}}

\def\PYZbs{\char`\\}
\def\PYZus{\char`\_}
\def\PYZob{\char`\{}
\def\PYZcb{\char`\}}
\def\PYZca{\char`\^}
\def\PYZam{\char`\&}
\def\PYZlt{\char`\<}
\def\PYZgt{\char`\>}
\def\PYZsh{\char`\#}
\def\PYZpc{\char`\%}
\def\PYZdl{\char`\$}
\def\PYZhy{\char`\-}
\def\PYZsq{\char`\'}
\def\PYZdq{\char`\"}
\def\PYZti{\char`\~}
% for compatibility with earlier versions
\def\PYZat{@}
\def\PYZlb{[}
\def\PYZrb{]}
\makeatother


    % Exact colors from NB
    \definecolor{incolor}{rgb}{0.0, 0.0, 0.5}
    \definecolor{outcolor}{rgb}{0.545, 0.0, 0.0}



    
    % Prevent overflowing lines due to hard-to-break entities
    \sloppy 
    % Setup hyperref package
    \hypersetup{
      breaklinks=true,  % so long urls are correctly broken across lines
      colorlinks=true,
      urlcolor=blue,
      linkcolor=darkorange,
      citecolor=darkgreen,
      }
    % Slightly bigger margins than the latex defaults
    
    \geometry{verbose,tmargin=1in,bmargin=1in,lmargin=1in,rmargin=1in}
    
    

    \begin{document}
    
    
    \maketitle
    
    

    
    \begin{Verbatim}[commandchars=\\\{\}]
{\color{incolor}In [{\color{incolor}30}]:} \PY{k+kn}{import} \PY{n+nn}{numpy} \PY{k+kn}{as} \PY{n+nn}{np}
         \PY{k+kn}{import} \PY{n+nn}{pandas} \PY{k+kn}{as} \PY{n+nn}{pd}
         \PY{k+kn}{import} \PY{n+nn}{matplotlib.pyplot} \PY{k+kn}{as} \PY{n+nn}{plt}
         \PY{k+kn}{from} \PY{n+nn}{\PYZus{}\PYZus{}future\PYZus{}\PYZus{}} \PY{k+kn}{import} \PY{n}{division}
\end{Verbatim}

    \section{Problem 1}\label{problem-1}

According to Snell's Law:

\begin{equation}
\sin\alpha_0 = n\sin\alpha
\end{equation}

the refraction angle:

\begin{equation}
\Delta\alpha \approx -\tan\alpha_0\cdot\frac{\Delta n}{n^2}\propto\Delta n
\end{equation}

According to Cauchy Formula,

\begin{align}
\Delta n(\lambda = 0.6) = 276.0\times 10^6
\Delta n(\lambda = 0.4) = 280.2\times 10^6
\Delta n(\lambda = 0.8) = 274.5\times 10^6
\end{align}

Thus, the refraction angle:

\begin{equation}
\Delta = \Delta \alpha(\lambda = 0.4) - \Delta \alpha(\lambda = 0.8) = (\frac{280.2}{276.0}-\frac{274.5}{276.0})\mathrm{arcmin} = 0.053\mathrm{arcmin}
\end{equation}

    \section{Problem 2}\label{problem-2}

    \begin{Verbatim}[commandchars=\\\{\}]
{\color{incolor}In [{\color{incolor}60}]:} \PY{k}{def} \PY{n+nf}{Gauss}\PY{p}{(}\PY{n}{x}\PY{p}{,} \PY{n}{FWHM}\PY{p}{)}\PY{p}{:}
             \PY{n}{sigma} \PY{o}{=} \PY{l+m+mf}{1.}\PY{o}{/}\PY{l+m+mf}{2.3548} \PY{o}{*} \PY{n}{FWHM}
             \PY{k}{return} \PY{n}{np}\PY{o}{.}\PY{n}{exp}\PY{p}{(}\PY{o}{\PYZhy{}}\PY{n}{x}\PY{o}{*}\PY{o}{*}\PY{l+m+mi}{2}\PY{o}{/}\PY{p}{(}\PY{l+m+mi}{2}\PY{o}{*}\PY{n}{sigma}\PY{o}{*}\PY{o}{*}\PY{l+m+mi}{2}\PY{p}{)}\PY{p}{)}
         
         \PY{n}{x} \PY{o}{=} \PY{n}{np}\PY{o}{.}\PY{n}{linspace}\PY{p}{(}\PY{o}{\PYZhy{}}\PY{l+m+mi}{10}\PY{p}{,} \PY{l+m+mi}{10}\PY{p}{,} \PY{l+m+mi}{2000}\PY{p}{)} \PY{c}{\PYZsh{}length in arcsec, oversample}
         \PY{n}{y} \PY{o}{=} \PY{n}{Gauss}\PY{p}{(}\PY{n}{x}\PY{p}{,} \PY{l+m+mf}{1.}\PY{p}{)}
         
         \PY{k}{def} \PY{n+nf}{centroidErr}\PY{p}{(}\PY{n}{pixelSize}\PY{p}{)}\PY{p}{:}
             \PY{n}{pixelCenter} \PY{o}{=} \PY{n}{np}\PY{o}{.}\PY{n}{sort}\PY{p}{(}\PY{n}{np}\PY{o}{.}\PY{n}{concatenate}\PY{p}{(}\PY{p}{(}\PY{n}{np}\PY{o}{.}\PY{n}{arange}\PY{p}{(}\PY{o}{\PYZhy{}}\PY{l+m+mf}{0.75}\PY{o}{*}\PY{n}{pixelSize}\PY{p}{,} \PY{o}{\PYZhy{}}\PY{l+m+mi}{10}\PY{o}{+}\PY{n}{pixelSize}\PY{p}{,}\PY{o}{\PYZhy{}}\PY{n}{pixelSize}\PY{p}{)}\PY{p}{,} 
                                       \PY{n}{np}\PY{o}{.}\PY{n}{arange}\PY{p}{(}\PY{l+m+mf}{0.25}\PY{o}{*}\PY{n}{pixelSize}\PY{p}{,} \PY{l+m+mi}{10} \PY{o}{\PYZhy{}}\PY{n}{pixelSize}\PY{p}{,} \PY{n}{pixelSize}\PY{p}{)}\PY{p}{)}\PY{p}{)}\PY{p}{)}
             \PY{n}{pixelFlux} \PY{o}{=} \PY{p}{[}\PY{p}{]}
             \PY{k}{for} \PY{n}{pxCenter} \PY{o+ow}{in} \PY{n}{pixelCenter}\PY{p}{:}
                 \PY{c}{\PYZsh{} flux for one pixel can be devided into three part}
                 \PY{n}{flux1} \PY{o}{=} \PY{n}{np}\PY{o}{.}\PY{n}{sum}\PY{p}{(}\PY{l+m+mf}{0.9} \PY{o}{*} \PY{n}{y}\PY{p}{[}\PY{p}{(}\PY{n}{x} \PY{o}{\PYZgt{}} \PY{n}{pxCenter} \PY{o}{\PYZhy{}} \PY{l+m+mf}{0.5} \PY{o}{*} \PY{n}{pixelSize}\PY{p}{)}\PY{o}{\PYZam{}}\PY{p}{(}\PY{n}{x} \PY{o}{\PYZlt{}} \PY{n}{pxCenter} \PY{o}{\PYZhy{}} \PY{l+m+mf}{0.45} \PY{o}{*} \PY{n}{pixelSize}\PY{p}{)}\PY{p}{]}\PY{p}{)}
                 \PY{n}{flux2} \PY{o}{=} \PY{n}{np}\PY{o}{.}\PY{n}{sum}\PY{p}{(}\PY{n}{y}\PY{p}{[}\PY{p}{(}\PY{n}{x} \PY{o}{\PYZgt{}} \PY{n}{pxCenter} \PY{o}{\PYZhy{}} \PY{l+m+mf}{0.45} \PY{o}{*} \PY{n}{pixelSize}\PY{p}{)}\PY{o}{\PYZam{}}\PY{p}{(}\PY{n}{x} \PY{o}{\PYZlt{}} \PY{n}{pxCenter} \PY{o}{+} \PY{l+m+mf}{0.45} \PY{o}{*} \PY{n}{pixelSize}\PY{p}{)}\PY{p}{]}\PY{p}{)}
                 \PY{n}{flux3} \PY{o}{=} \PY{n}{np}\PY{o}{.}\PY{n}{sum}\PY{p}{(}\PY{l+m+mf}{0.9} \PY{o}{*} \PY{n}{y}\PY{p}{[}\PY{p}{(}\PY{n}{x} \PY{o}{\PYZgt{}} \PY{n}{pxCenter} \PY{o}{+} \PY{l+m+mf}{0.45} \PY{o}{*} \PY{n}{pixelSize}\PY{p}{)}\PY{o}{\PYZam{}}\PY{p}{(}\PY{n}{x} \PY{o}{\PYZlt{}} \PY{n}{pxCenter} \PY{o}{+} \PY{l+m+mf}{0.45} \PY{o}{*} \PY{n}{pixelSize}\PY{p}{)}\PY{p}{]}\PY{p}{)}
                 \PY{n}{pixelFlux}\PY{o}{.}\PY{n}{append}\PY{p}{(}\PY{n}{flux1} \PY{o}{+} \PY{n}{flux2} \PY{o}{+} \PY{n}{flux3}\PY{p}{)}
             \PY{n}{pixelFlux} \PY{o}{=} \PY{n}{np}\PY{o}{.}\PY{n}{array}\PY{p}{(}\PY{n}{pixelFlux}\PY{p}{)}
             \PY{n}{cMeasured} \PY{o}{=} \PY{n}{np}\PY{o}{.}\PY{n}{sum}\PY{p}{(}\PY{n}{pixelCenter} \PY{o}{*} \PY{n}{pixelFlux}\PY{p}{)}\PY{o}{/}\PY{n}{np}\PY{o}{.}\PY{n}{sum}\PY{p}{(}\PY{n}{pixelFlux}\PY{p}{)}
             \PY{k}{return} \PY{n+nb}{abs}\PY{p}{(}\PY{n}{cMeasured}\PY{p}{)}
         
         \PY{n}{psList} \PY{o}{=} \PY{n}{np}\PY{o}{.}\PY{n}{linspace}\PY{p}{(}\PY{l+m+mf}{0.1}\PY{p}{,} \PY{l+m+mi}{1}\PY{p}{,} \PY{l+m+mi}{100}\PY{p}{)}
         \PY{n}{errList} \PY{o}{=} \PY{n+nb}{map}\PY{p}{(}\PY{n}{centroidErr}\PY{p}{,} \PY{n}{psList}\PY{p}{)}
         \PY{n}{plt}\PY{o}{.}\PY{n}{plot}\PY{p}{(}\PY{n}{psList}\PY{p}{,}\PY{n}{errList}\PY{p}{)}
         \PY{n}{plt}\PY{o}{.}\PY{n}{axhline}\PY{p}{(}\PY{l+m+mf}{0.01}\PY{p}{,} \PY{n}{color} \PY{o}{=} \PY{l+s}{\PYZsq{}}\PY{l+s}{r}\PY{l+s}{\PYZsq{}}\PY{p}{)}
         \PY{n}{plt}\PY{o}{.}\PY{n}{xlabel}\PY{p}{(}\PY{l+s}{\PYZsq{}}\PY{l+s}{Pixel size (arcsec)}\PY{l+s}{\PYZsq{}}\PY{p}{)}
         \PY{n}{plt}\PY{o}{.}\PY{n}{ylabel}\PY{p}{(}\PY{l+s}{\PYZsq{}}\PY{l+s}{Centroid Error (arcsec)}\PY{l+s}{\PYZsq{}}\PY{p}{)}
\end{Verbatim}

            \begin{Verbatim}[commandchars=\\\{\}]
{\color{outcolor}Out[{\color{outcolor}60}]:} <matplotlib.text.Text at 0x10c7fc250>
\end{Verbatim}
        
    \begin{center}
    \adjustimage{max size={0.9\linewidth}{0.9\paperheight}}{Untitled0_files/Untitled0_3_1.png}
    \end{center}
    { \hspace*{\fill} \\}
    
    Above script calculates the centroid measurement error as a function of
pixel size. According to the plot, the sample size should be smaller
than 0.4 arcsec so that the centroid can be measured as precise as 0.01
arcsec

    \section{Problem 3}\label{problem-3}

table assembled as below

    \begin{Verbatim}[commandchars=\\\{\}]
{\color{incolor}In [{\color{incolor}75}]:} \PY{n}{df0} \PY{o}{=} \PY{n}{pd}\PY{o}{.}\PY{n}{read\PYZus{}csv}\PY{p}{(}\PY{l+s}{\PYZsq{}}\PY{l+s}{table0.csv}\PY{l+s}{\PYZsq{}}\PY{p}{,} \PY{n}{names} \PY{o}{=} \PY{p}{[}\PY{l+s}{\PYZsq{}}\PY{l+s}{name}\PY{l+s}{\PYZsq{}}\PY{p}{,}\PY{l+s}{\PYZsq{}}\PY{l+s}{type}\PY{l+s}{\PYZsq{}}\PY{p}{,}\PY{l+s}{\PYZsq{}}\PY{l+s}{Jmag}\PY{l+s}{\PYZsq{}}\PY{p}{,}\PY{l+s}{\PYZsq{}}\PY{l+s}{Kmag}\PY{l+s}{\PYZsq{}}\PY{p}{]}\PY{p}{,} \PY{n}{usecols} \PY{o}{=} \PY{p}{(}\PY{l+m+mi}{0}\PY{p}{,}\PY{l+m+mi}{1}\PY{p}{,}\PY{l+m+mi}{2}\PY{p}{,}\PY{l+m+mi}{3}\PY{p}{)}\PY{p}{)}
         \PY{n}{df1} \PY{o}{=} \PY{n}{pd}\PY{o}{.}\PY{n}{read\PYZus{}csv}\PY{p}{(}\PY{l+s}{\PYZsq{}}\PY{l+s}{table1.csv}\PY{l+s}{\PYZsq{}}\PY{p}{,} \PY{n}{names} \PY{o}{=} \PY{p}{[}\PY{l+s}{\PYZsq{}}\PY{l+s}{name}\PY{l+s}{\PYZsq{}}\PY{p}{,} \PY{l+s}{\PYZsq{}}\PY{l+s}{Jmag}\PY{l+s}{\PYZsq{}}\PY{p}{,} \PY{l+s}{\PYZsq{}}\PY{l+s}{Kmag}\PY{l+s}{\PYZsq{}}\PY{p}{]}\PY{p}{,} \PY{n}{usecols} \PY{o}{=} \PY{p}{(}\PY{l+m+mi}{0}\PY{p}{,}\PY{l+m+mi}{3}\PY{p}{,}\PY{l+m+mi}{5}\PY{p}{)}\PY{p}{)}
         \PY{n}{df0}\PY{p}{[}\PY{l+s}{\PYZsq{}}\PY{l+s}{name}\PY{l+s}{\PYZsq{}}\PY{p}{]} \PY{o}{=} \PY{p}{[}\PY{n}{string}\PY{o}{.}\PY{n}{strip}\PY{p}{(}\PY{p}{)} \PY{k}{for} \PY{n}{string} \PY{o+ow}{in} \PY{n}{df0}\PY{p}{[}\PY{l+s}{\PYZsq{}}\PY{l+s}{name}\PY{l+s}{\PYZsq{}}\PY{p}{]}\PY{o}{.}\PY{n}{values}\PY{p}{]}
         \PY{n}{df1}\PY{p}{[}\PY{l+s}{\PYZsq{}}\PY{l+s}{name}\PY{l+s}{\PYZsq{}}\PY{p}{]} \PY{o}{=} \PY{p}{[}\PY{n}{string}\PY{o}{.}\PY{n}{strip}\PY{p}{(}\PY{p}{)} \PY{k}{for} \PY{n}{string} \PY{o+ow}{in} \PY{n}{df1}\PY{p}{[}\PY{l+s}{\PYZsq{}}\PY{l+s}{name}\PY{l+s}{\PYZsq{}}\PY{p}{]}\PY{o}{.}\PY{n}{values}\PY{p}{]}
         \PY{n}{total} \PY{o}{=}\PY{n}{pd}\PY{o}{.}\PY{n}{merge}\PY{p}{(}\PY{n}{df0}\PY{p}{,} \PY{n}{df1}\PY{p}{,} \PY{n}{on} \PY{o}{=} \PY{l+s}{\PYZsq{}}\PY{l+s}{name}\PY{l+s}{\PYZsq{}}\PY{p}{)}
         \PY{n}{total}\PY{p}{[}\PY{l+s}{\PYZsq{}}\PY{l+s}{J\PYZhy{}K(KM)}\PY{l+s}{\PYZsq{}}\PY{p}{]} \PY{o}{=} \PY{n}{total}\PY{p}{[}\PY{l+s}{\PYZsq{}}\PY{l+s}{Jmag\PYZus{}y}\PY{l+s}{\PYZsq{}}\PY{p}{]} \PY{o}{\PYZhy{}} \PY{n}{total}\PY{p}{[}\PY{l+s}{\PYZsq{}}\PY{l+s}{Kmag\PYZus{}y}\PY{l+s}{\PYZsq{}}\PY{p}{]}
         \PY{n}{total}
\end{Verbatim}

            \begin{Verbatim}[commandchars=\\\{\}]
{\color{outcolor}Out[{\color{outcolor}75}]:}        name       type  Jmag\_x  Kmag\_x  Jmag\_y  Kmag\_y  J-K(KM)
         0   HR 1256     K0III     2.63    1.97   2.575  1.9750   0.6000
         1   HR 1286  K1II-III     3.36    2.49   3.177  2.4000   0.7770
         2   HR 1791     B7III     1.97    2.05   1.868  1.9560  -0.0880
         3   HR 1907    K0IIIb     2.35    1.69   2.282  1.6840   0.5980
         4   HR 1963     K1III     2.89    2.11   2.797  2.0700   0.7270
         5   HR 2077     K0III     2.09    1.46   1.970  1.4000   0.5700
         6   HR 2427     K3Iab     2.79    2.05   2.731  2.0520   0.6790
         7   HR 2560  G5III-IV     2.89    2.33   2.810  2.3160   0.4940
         8   HR 3003     K4III     2.28    1.33   2.190  1.3000   0.8900
         9   HR 4335     K1III     1.16    0.44   1.095  0.4240   0.6710
         10  HR 4377     K3III     1.18    0.31   1.092  0.2740   0.8180
         11  HR 4608    G8IIIa     2.48    1.90   2.435  1.9877   0.4473
         12  HR 4737     K1III     2.55    1.90   2.473  1.8710   0.6020
         13  HR 4983     F9.5V     3.24    2.90   3.166  2.8730   0.2930
         14  HR 5107       A3V     3.20    3.11   3.099  3.0720   0.0270
         15  HR 5854    K2IIIb     0.76    0.06   0.713  0.0790   0.6340
         16  HR 5947     K2III     2.09    1.30   2.010  1.2840   0.7260
         17  HR 6623      G5IV     2.18    1.77   2.127  1.7390   0.3880
         18  HR 6698     G9III     1.68    1.12   1.721  1.1830   0.5380
         19  HR 6705     K5III    -0.39   -1.34  -0.430 -1.3100   0.8800
         20  HR 6707      F2II     3.55    3.23   3.500  3.2200   0.2800
         21  HR 7236      B9Vn     3.64    3.67   3.559  3.6000  -0.0410
         22  HR 7525      K3II     0.30   -0.59   0.208 -0.6400   0.8480
         23  HR 7557       A7V     0.39    0.26   0.327  0.2050   0.1220
         24  HR 7615     K0III     2.28    1.67   2.205  1.6530   0.5520
         25  HR 7949     K0III     0.77    0.11   0.683  0.0800   0.6030
         26  HR 8143     B9Iab     3.95    3.79   3.850  3.7400   0.1100
         27  HR 8632     K2III     2.36    1.51   2.240  1.4600   0.7800
         28  HR 8905      F8IV     3.37    3.02   3.308  2.9730   0.3350
\end{Verbatim}
        
    \begin{Verbatim}[commandchars=\\\{\}]
{\color{incolor}In [{\color{incolor}79}]:} \PY{c}{\PYZsh{}Linear Fit}
         \PY{c}{\PYZsh{}For J}
         \PY{n}{JK} \PY{o}{=} \PY{n}{np}\PY{o}{.}\PY{n}{linspace}\PY{p}{(}\PY{o}{\PYZhy{}}\PY{l+m+mf}{0.2}\PY{p}{,} \PY{l+m+mf}{1.0}\PY{p}{,} \PY{l+m+mi}{100}\PY{p}{)}
         \PY{n}{mJ}\PY{p}{,} \PY{n}{bJ} \PY{o}{=} \PY{n}{np}\PY{o}{.}\PY{n}{polyfit}\PY{p}{(} \PY{n}{total}\PY{p}{[}\PY{l+s}{\PYZsq{}}\PY{l+s}{J\PYZhy{}K(KM)}\PY{l+s}{\PYZsq{}}\PY{p}{]}\PY{p}{,} \PY{n}{total}\PY{p}{[}\PY{l+s}{\PYZsq{}}\PY{l+s}{Jmag\PYZus{}x}\PY{l+s}{\PYZsq{}}\PY{p}{]}\PY{o}{\PYZhy{}}\PY{n}{total}\PY{p}{[}\PY{l+s}{\PYZsq{}}\PY{l+s}{Jmag\PYZus{}y}\PY{l+s}{\PYZsq{}}\PY{p}{]}\PY{p}{,} \PY{l+m+mi}{1}\PY{p}{)}
         \PY{n}{plt}\PY{o}{.}\PY{n}{plot}\PY{p}{(}\PY{n}{total}\PY{p}{[}\PY{l+s}{\PYZsq{}}\PY{l+s}{J\PYZhy{}K(KM)}\PY{l+s}{\PYZsq{}}\PY{p}{]}\PY{p}{,} \PY{n}{total}\PY{p}{[}\PY{l+s}{\PYZsq{}}\PY{l+s}{Jmag\PYZus{}x}\PY{l+s}{\PYZsq{}}\PY{p}{]} \PY{o}{\PYZhy{}} \PY{n}{total}\PY{p}{[}\PY{l+s}{\PYZsq{}}\PY{l+s}{Jmag\PYZus{}y}\PY{l+s}{\PYZsq{}}\PY{p}{]}\PY{p}{,} \PY{l+s}{\PYZsq{}}\PY{l+s}{.}\PY{l+s}{\PYZsq{}}\PY{p}{)}
         \PY{n}{plt}\PY{o}{.}\PY{n}{plot}\PY{p}{(}\PY{n}{JK}\PY{p}{,} \PY{n}{mJ}\PY{o}{*}\PY{n}{JK} \PY{o}{+} \PY{n}{bJ}\PY{p}{)}
         \PY{n}{plt}\PY{o}{.}\PY{n}{xlabel}\PY{p}{(}\PY{l+s}{\PYZsq{}}\PY{l+s}{J \PYZhy{} K color (KM)}\PY{l+s}{\PYZsq{}}\PY{p}{)}
         \PY{n}{plt}\PY{o}{.}\PY{n}{ylabel}\PY{p}{(}\PY{l+s}{\PYZsq{}}\PY{l+s}{\PYZdl{}J\PYZus{}}\PY{l+s}{\PYZbs{}}\PY{l+s}{mathrm\PYZob{}J\PYZcb{} \PYZhy{} J\PYZus{}}\PY{l+s}{\PYZbs{}}\PY{l+s}{mathrm\PYZob{}KM\PYZcb{}\PYZdl{}}\PY{l+s}{\PYZsq{}}\PY{p}{)}
         \PY{n}{plt}\PY{o}{.}\PY{n}{text}\PY{p}{(}\PY{l+m+mi}{0}\PY{p}{,} \PY{l+m+mi}{0}\PY{p}{,} \PY{l+s}{\PYZsq{}}\PY{l+s}{J = \PYZob{}0:.3f\PYZcb{} * ((J\PYZhy{}K)(KM)) + J(KM) + \PYZob{}1:.3f\PYZcb{}}\PY{l+s}{\PYZsq{}}\PY{o}{.}\PY{n}{format}\PY{p}{(}\PY{n}{mJ}\PY{p}{,} \PY{n}{bJ}\PY{p}{)}\PY{p}{)}
\end{Verbatim}

            \begin{Verbatim}[commandchars=\\\{\}]
{\color{outcolor}Out[{\color{outcolor}79}]:} <matplotlib.text.Text at 0x10c84d210>
\end{Verbatim}
        
    \begin{center}
    \adjustimage{max size={0.9\linewidth}{0.9\paperheight}}{Untitled0_files/Untitled0_7_1.png}
    \end{center}
    { \hspace*{\fill} \\}
    
    \begin{Verbatim}[commandchars=\\\{\}]
{\color{incolor}In [{\color{incolor}81}]:} \PY{n}{mK}\PY{p}{,} \PY{n}{bK} \PY{o}{=} \PY{n}{np}\PY{o}{.}\PY{n}{polyfit}\PY{p}{(} \PY{n}{total}\PY{p}{[}\PY{l+s}{\PYZsq{}}\PY{l+s}{J\PYZhy{}K(KM)}\PY{l+s}{\PYZsq{}}\PY{p}{]}\PY{p}{,} \PY{n}{total}\PY{p}{[}\PY{l+s}{\PYZsq{}}\PY{l+s}{Kmag\PYZus{}x}\PY{l+s}{\PYZsq{}}\PY{p}{]}\PY{o}{\PYZhy{}}\PY{n}{total}\PY{p}{[}\PY{l+s}{\PYZsq{}}\PY{l+s}{Kmag\PYZus{}y}\PY{l+s}{\PYZsq{}}\PY{p}{]}\PY{p}{,} \PY{l+m+mi}{1}\PY{p}{)}
         \PY{n}{plt}\PY{o}{.}\PY{n}{plot}\PY{p}{(}\PY{n}{total}\PY{p}{[}\PY{l+s}{\PYZsq{}}\PY{l+s}{J\PYZhy{}K(KM)}\PY{l+s}{\PYZsq{}}\PY{p}{]}\PY{p}{,} \PY{n}{total}\PY{p}{[}\PY{l+s}{\PYZsq{}}\PY{l+s}{Kmag\PYZus{}x}\PY{l+s}{\PYZsq{}}\PY{p}{]} \PY{o}{\PYZhy{}} \PY{n}{total}\PY{p}{[}\PY{l+s}{\PYZsq{}}\PY{l+s}{Kmag\PYZus{}y}\PY{l+s}{\PYZsq{}}\PY{p}{]}\PY{p}{,} \PY{l+s}{\PYZsq{}}\PY{l+s}{.}\PY{l+s}{\PYZsq{}}\PY{p}{)}
         \PY{n}{plt}\PY{o}{.}\PY{n}{plot}\PY{p}{(}\PY{n}{JK}\PY{p}{,} \PY{n}{mK}\PY{o}{*}\PY{n}{JK} \PY{o}{+} \PY{n}{bK}\PY{p}{)}
         \PY{n}{plt}\PY{o}{.}\PY{n}{xlabel}\PY{p}{(}\PY{l+s}{\PYZsq{}}\PY{l+s}{J \PYZhy{} K color (KM)}\PY{l+s}{\PYZsq{}}\PY{p}{)}
         \PY{n}{plt}\PY{o}{.}\PY{n}{ylabel}\PY{p}{(}\PY{l+s}{\PYZsq{}}\PY{l+s}{\PYZdl{}K\PYZus{}}\PY{l+s}{\PYZbs{}}\PY{l+s}{mathrm\PYZob{}J\PYZcb{} \PYZhy{} K\PYZus{}}\PY{l+s}{\PYZbs{}}\PY{l+s}{mathrm\PYZob{}KM\PYZcb{}\PYZdl{}}\PY{l+s}{\PYZsq{}}\PY{p}{)}
         \PY{n}{plt}\PY{o}{.}\PY{n}{text}\PY{p}{(}\PY{l+m+mi}{0}\PY{p}{,} \PY{o}{\PYZhy{}}\PY{l+m+mf}{0.05}\PY{p}{,} \PY{l+s}{\PYZsq{}}\PY{l+s}{K = \PYZob{}0:.3f\PYZcb{} * ((J\PYZhy{}K)(KM)) + K(KM) + \PYZob{}1:.3f\PYZcb{}}\PY{l+s}{\PYZsq{}}\PY{o}{.}\PY{n}{format}\PY{p}{(}\PY{n}{mK}\PY{p}{,} \PY{n}{bK}\PY{p}{)}\PY{p}{)}
\end{Verbatim}

            \begin{Verbatim}[commandchars=\\\{\}]
{\color{outcolor}Out[{\color{outcolor}81}]:} <matplotlib.text.Text at 0x10fd8bf10>
\end{Verbatim}
        
    \begin{center}
    \adjustimage{max size={0.9\linewidth}{0.9\paperheight}}{Untitled0_files/Untitled0_8_1.png}
    \end{center}
    { \hspace*{\fill} \\}
    
    With linear fit of J and K magnitude difference with KM's J-K color,
Johnson's magnitude can be expressed as above.

    \section{Problem 4}\label{problem-4}

\begin{enumerate}
\def\labelenumi{\alph{enumi})}
\itemsep1pt\parskip0pt\parsep0pt
\item
  According to the plot in Problem 3, the precision can be around 0.1
  magntiude.
\item
  For vega, J-K color equals 0. According to the plot above, Vega has a
  J magnitude of -0.05 and K magnitude of -0.02. Vega changes because it
  is a fasting rotate star and varies a lot.
\end{enumerate}

    \section{Problem 5}\label{problem-5}

\begin{itemize}
\itemsep1pt\parskip0pt\parsep0pt
\item
  determined by integrating the source spectrum over the filter
  passband(if $f = a\lambda^{-4}d\lambda$):
\end{itemize}

\begin{equation}
\begin{split}
I &= a\int_{\lambda_1}^{lambda_2} \lambda^{-4}d\lambda\\
&=\frac{a}{3}\left(\frac{1}{(\lambda - \Delta/2)^3}-\frac{1}{(\lambda_2+\Delta/2)^3}\right)\\
&=\frac{a}{3\lambda^3}\left[1+3\left(\frac{\Delta}{2\lambda}\right) + \frac{3\times4}{2!}\left(\frac{\Delta}{2\lambda}\right)^2+ \frac{3\times4\times5}{3!}\left(\frac{\Delta}{2\lambda}\right)^3+\cdots\right] - \frac{a}{3\lambda^3}\left[1-3\left(\frac{\Delta}{2\lambda}\right) + \frac{3\times4}{2!}\left(\frac{\Delta}{2\lambda}\right)^2-\frac{3\times4\times5}{3!}\left(\frac{\Delta}{2\lambda}\right)^3+\cdots\right]\\
&=a\frac{\Delta}{\lambda^4}\left[1+\frac{5}{6}\left(\frac{\Delta}{\lambda}\right)^2\right] + O\left(\frac{\Delta^4}{\lambda^4}\right)
\end{split}
\end{equation}

    \section{Problem 6}\label{problem-6}

B-V color is 0.68, thus

\begin{itemize}
\itemsep1pt\parskip0pt\parsep0pt
\item
  $T_\mathrm{eff} = 5560K$
\item
  absolute V band magnitude is $M_\mathrm{V} = 5.1$
\item
  Distance:
\end{itemize}

\begin{equation}
D = 10^{\frac{m_V - M_V + 5}{5}} = 10^{(7.02 - 5.1 +5)/5} = 24.2 \mathrm{pc}
\end{equation}

\begin{itemize}
\itemsep1pt\parskip0pt\parsep0pt
\item
  Bolometric magnitude: $M_\mathrm{bol} = M_\mathrm{V} + BC = 4.89$
\item
  Luminosity:
\end{itemize}

\begin{equation}
L = 10^{(M_{\mathrm{bol},\odot} - M_\mathrm{bol})/2.5} L_\odot= 10^{(4.75 - 4.89)/2.5}L_\odot = 0.88L_\odot
\end{equation}

    \section{Problem 7}\label{problem-7}

use two half wave plate would rotate the position angle with
$180^\circ$, that is two say it would be the same as there is no half
wave plate

    \begin{Verbatim}[commandchars=\\\{\}]
{\color{incolor}In [{\color{incolor}}]:} 
\end{Verbatim}


    % Add a bibliography block to the postdoc
    
    
    
    \end{document}
