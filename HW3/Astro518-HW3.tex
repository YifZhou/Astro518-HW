
% Default to the notebook output style

    


% Inherit from the specified cell style.






    
\documentclass{article}

    
        
    
    \usepackage{graphicx} % Used to insert images
    \usepackage{adjustbox} % Used to constrain images to a maximum size 
    \usepackage{color} % Allow colors to be defined
    \usepackage{enumerate} % Needed for markdown enumerations to work
    \usepackage{geometry} % Used to adjust the document margins
    \usepackage{amsmath} % Equations
    \usepackage{amssymb} % Equations
    \usepackage[mathletters]{ucs} % Extended unicode (utf-8) support
    \usepackage[utf8x]{inputenc} % Allow utf-8 characters in the tex document
    \usepackage{fancyvrb} % verbatim replacement that allows latex
    \usepackage{grffile} % extends the file name processing of package graphics 
                         % to support a larger range 
    % The hyperref package gives us a pdf with properly built
    % internal navigation ('pdf bookmarks' for the table of contents,
    % internal cross-reference links, web links for URLs, etc.)
    \usepackage{hyperref}
    \usepackage{longtable} % longtable support required by pandoc >1.10
    \usepackage{booktabs}  % table support for pandoc > 1.12.2
    

    \usepackage{tikz} % Needed to box output/input
    \usepackage{scrextend} % Used to indent output
    \usepackage{needspace} % Make prompts follow contents
    \usepackage{framed} % Used to draw output that spans multiple pages


    
    
    \definecolor{orange}{cmyk}{0,0.4,0.8,0.2}
    \definecolor{darkorange}{rgb}{.71,0.21,0.01}
    \definecolor{darkgreen}{rgb}{.12,.54,.11}
    \definecolor{myteal}{rgb}{.26, .44, .56}
    \definecolor{gray}{gray}{0.45}
    \definecolor{lightgray}{gray}{.95}
    \definecolor{mediumgray}{gray}{.8}
    \definecolor{inputbackground}{rgb}{.95, .95, .85}
    \definecolor{outputbackground}{rgb}{.95, .95, .95}
    \definecolor{traceback}{rgb}{1, .95, .95}
    % ansi colors
    \definecolor{red}{rgb}{.6,0,0}
    \definecolor{green}{rgb}{0,.65,0}
    \definecolor{brown}{rgb}{0.6,0.6,0}
    \definecolor{blue}{rgb}{0,.145,.698}
    \definecolor{purple}{rgb}{.698,.145,.698}
    \definecolor{cyan}{rgb}{0,.698,.698}
    \definecolor{lightgray}{gray}{0.5}
    
    % bright ansi colors
    \definecolor{darkgray}{gray}{0.25}
    \definecolor{lightred}{rgb}{1.0,0.39,0.28}
    \definecolor{lightgreen}{rgb}{0.48,0.99,0.0}
    \definecolor{lightblue}{rgb}{0.53,0.81,0.92}
    \definecolor{lightpurple}{rgb}{0.87,0.63,0.87}
    \definecolor{lightcyan}{rgb}{0.5,1.0,0.83}
    
    % commands and environments needed by pandoc snippets
    % extracted from the output of `pandoc -s`
    \DefineVerbatimEnvironment{Highlighting}{Verbatim}{commandchars=\\\{\}}
    % Add ',fontsize=\small' for more characters per line
    \newenvironment{Shaded}{}{}
    \newcommand{\KeywordTok}[1]{\textcolor[rgb]{0.00,0.44,0.13}{\textbf{{#1}}}}
    \newcommand{\DataTypeTok}[1]{\textcolor[rgb]{0.56,0.13,0.00}{{#1}}}
    \newcommand{\DecValTok}[1]{\textcolor[rgb]{0.25,0.63,0.44}{{#1}}}
    \newcommand{\BaseNTok}[1]{\textcolor[rgb]{0.25,0.63,0.44}{{#1}}}
    \newcommand{\FloatTok}[1]{\textcolor[rgb]{0.25,0.63,0.44}{{#1}}}
    \newcommand{\CharTok}[1]{\textcolor[rgb]{0.25,0.44,0.63}{{#1}}}
    \newcommand{\StringTok}[1]{\textcolor[rgb]{0.25,0.44,0.63}{{#1}}}
    \newcommand{\CommentTok}[1]{\textcolor[rgb]{0.38,0.63,0.69}{\textit{{#1}}}}
    \newcommand{\OtherTok}[1]{\textcolor[rgb]{0.00,0.44,0.13}{{#1}}}
    \newcommand{\AlertTok}[1]{\textcolor[rgb]{1.00,0.00,0.00}{\textbf{{#1}}}}
    \newcommand{\FunctionTok}[1]{\textcolor[rgb]{0.02,0.16,0.49}{{#1}}}
    \newcommand{\RegionMarkerTok}[1]{{#1}}
    \newcommand{\ErrorTok}[1]{\textcolor[rgb]{1.00,0.00,0.00}{\textbf{{#1}}}}
    \newcommand{\NormalTok}[1]{{#1}}
    
    % Define a nice break command that doesn't care if a line doesn't already
    % exist.
    \def\br{\hspace*{\fill} \\* }
    % Math Jax compatability definitions
    \def\gt{>}
    \def\lt{<}
    % Document parameters
    \title{Astro518-HW3}
    
    
    

    % Pygments definitions
    
\makeatletter
\def\PY@reset{\let\PY@it=\relax \let\PY@bf=\relax%
    \let\PY@ul=\relax \let\PY@tc=\relax%
    \let\PY@bc=\relax \let\PY@ff=\relax}
\def\PY@tok#1{\csname PY@tok@#1\endcsname}
\def\PY@toks#1+{\ifx\relax#1\empty\else%
    \PY@tok{#1}\expandafter\PY@toks\fi}
\def\PY@do#1{\PY@bc{\PY@tc{\PY@ul{%
    \PY@it{\PY@bf{\PY@ff{#1}}}}}}}
\def\PY#1#2{\PY@reset\PY@toks#1+\relax+\PY@do{#2}}

\expandafter\def\csname PY@tok@gd\endcsname{\def\PY@tc##1{\textcolor[rgb]{0.63,0.00,0.00}{##1}}}
\expandafter\def\csname PY@tok@gu\endcsname{\let\PY@bf=\textbf\def\PY@tc##1{\textcolor[rgb]{0.50,0.00,0.50}{##1}}}
\expandafter\def\csname PY@tok@gt\endcsname{\def\PY@tc##1{\textcolor[rgb]{0.00,0.27,0.87}{##1}}}
\expandafter\def\csname PY@tok@gs\endcsname{\let\PY@bf=\textbf}
\expandafter\def\csname PY@tok@gr\endcsname{\def\PY@tc##1{\textcolor[rgb]{1.00,0.00,0.00}{##1}}}
\expandafter\def\csname PY@tok@cm\endcsname{\let\PY@it=\textit\def\PY@tc##1{\textcolor[rgb]{0.25,0.50,0.50}{##1}}}
\expandafter\def\csname PY@tok@vg\endcsname{\def\PY@tc##1{\textcolor[rgb]{0.10,0.09,0.49}{##1}}}
\expandafter\def\csname PY@tok@m\endcsname{\def\PY@tc##1{\textcolor[rgb]{0.40,0.40,0.40}{##1}}}
\expandafter\def\csname PY@tok@mh\endcsname{\def\PY@tc##1{\textcolor[rgb]{0.40,0.40,0.40}{##1}}}
\expandafter\def\csname PY@tok@go\endcsname{\def\PY@tc##1{\textcolor[rgb]{0.53,0.53,0.53}{##1}}}
\expandafter\def\csname PY@tok@ge\endcsname{\let\PY@it=\textit}
\expandafter\def\csname PY@tok@vc\endcsname{\def\PY@tc##1{\textcolor[rgb]{0.10,0.09,0.49}{##1}}}
\expandafter\def\csname PY@tok@il\endcsname{\def\PY@tc##1{\textcolor[rgb]{0.40,0.40,0.40}{##1}}}
\expandafter\def\csname PY@tok@cs\endcsname{\let\PY@it=\textit\def\PY@tc##1{\textcolor[rgb]{0.25,0.50,0.50}{##1}}}
\expandafter\def\csname PY@tok@cp\endcsname{\def\PY@tc##1{\textcolor[rgb]{0.74,0.48,0.00}{##1}}}
\expandafter\def\csname PY@tok@gi\endcsname{\def\PY@tc##1{\textcolor[rgb]{0.00,0.63,0.00}{##1}}}
\expandafter\def\csname PY@tok@gh\endcsname{\let\PY@bf=\textbf\def\PY@tc##1{\textcolor[rgb]{0.00,0.00,0.50}{##1}}}
\expandafter\def\csname PY@tok@ni\endcsname{\let\PY@bf=\textbf\def\PY@tc##1{\textcolor[rgb]{0.60,0.60,0.60}{##1}}}
\expandafter\def\csname PY@tok@nl\endcsname{\def\PY@tc##1{\textcolor[rgb]{0.63,0.63,0.00}{##1}}}
\expandafter\def\csname PY@tok@nn\endcsname{\let\PY@bf=\textbf\def\PY@tc##1{\textcolor[rgb]{0.00,0.00,1.00}{##1}}}
\expandafter\def\csname PY@tok@no\endcsname{\def\PY@tc##1{\textcolor[rgb]{0.53,0.00,0.00}{##1}}}
\expandafter\def\csname PY@tok@na\endcsname{\def\PY@tc##1{\textcolor[rgb]{0.49,0.56,0.16}{##1}}}
\expandafter\def\csname PY@tok@nb\endcsname{\def\PY@tc##1{\textcolor[rgb]{0.00,0.50,0.00}{##1}}}
\expandafter\def\csname PY@tok@nc\endcsname{\let\PY@bf=\textbf\def\PY@tc##1{\textcolor[rgb]{0.00,0.00,1.00}{##1}}}
\expandafter\def\csname PY@tok@nd\endcsname{\def\PY@tc##1{\textcolor[rgb]{0.67,0.13,1.00}{##1}}}
\expandafter\def\csname PY@tok@ne\endcsname{\let\PY@bf=\textbf\def\PY@tc##1{\textcolor[rgb]{0.82,0.25,0.23}{##1}}}
\expandafter\def\csname PY@tok@nf\endcsname{\def\PY@tc##1{\textcolor[rgb]{0.00,0.00,1.00}{##1}}}
\expandafter\def\csname PY@tok@si\endcsname{\let\PY@bf=\textbf\def\PY@tc##1{\textcolor[rgb]{0.73,0.40,0.53}{##1}}}
\expandafter\def\csname PY@tok@s2\endcsname{\def\PY@tc##1{\textcolor[rgb]{0.73,0.13,0.13}{##1}}}
\expandafter\def\csname PY@tok@vi\endcsname{\def\PY@tc##1{\textcolor[rgb]{0.10,0.09,0.49}{##1}}}
\expandafter\def\csname PY@tok@nt\endcsname{\let\PY@bf=\textbf\def\PY@tc##1{\textcolor[rgb]{0.00,0.50,0.00}{##1}}}
\expandafter\def\csname PY@tok@nv\endcsname{\def\PY@tc##1{\textcolor[rgb]{0.10,0.09,0.49}{##1}}}
\expandafter\def\csname PY@tok@s1\endcsname{\def\PY@tc##1{\textcolor[rgb]{0.73,0.13,0.13}{##1}}}
\expandafter\def\csname PY@tok@sh\endcsname{\def\PY@tc##1{\textcolor[rgb]{0.73,0.13,0.13}{##1}}}
\expandafter\def\csname PY@tok@sc\endcsname{\def\PY@tc##1{\textcolor[rgb]{0.73,0.13,0.13}{##1}}}
\expandafter\def\csname PY@tok@sx\endcsname{\def\PY@tc##1{\textcolor[rgb]{0.00,0.50,0.00}{##1}}}
\expandafter\def\csname PY@tok@bp\endcsname{\def\PY@tc##1{\textcolor[rgb]{0.00,0.50,0.00}{##1}}}
\expandafter\def\csname PY@tok@c1\endcsname{\let\PY@it=\textit\def\PY@tc##1{\textcolor[rgb]{0.25,0.50,0.50}{##1}}}
\expandafter\def\csname PY@tok@kc\endcsname{\let\PY@bf=\textbf\def\PY@tc##1{\textcolor[rgb]{0.00,0.50,0.00}{##1}}}
\expandafter\def\csname PY@tok@c\endcsname{\let\PY@it=\textit\def\PY@tc##1{\textcolor[rgb]{0.25,0.50,0.50}{##1}}}
\expandafter\def\csname PY@tok@mf\endcsname{\def\PY@tc##1{\textcolor[rgb]{0.40,0.40,0.40}{##1}}}
\expandafter\def\csname PY@tok@err\endcsname{\def\PY@bc##1{\setlength{\fboxsep}{0pt}\fcolorbox[rgb]{1.00,0.00,0.00}{1,1,1}{\strut ##1}}}
\expandafter\def\csname PY@tok@kd\endcsname{\let\PY@bf=\textbf\def\PY@tc##1{\textcolor[rgb]{0.00,0.50,0.00}{##1}}}
\expandafter\def\csname PY@tok@ss\endcsname{\def\PY@tc##1{\textcolor[rgb]{0.10,0.09,0.49}{##1}}}
\expandafter\def\csname PY@tok@sr\endcsname{\def\PY@tc##1{\textcolor[rgb]{0.73,0.40,0.53}{##1}}}
\expandafter\def\csname PY@tok@mo\endcsname{\def\PY@tc##1{\textcolor[rgb]{0.40,0.40,0.40}{##1}}}
\expandafter\def\csname PY@tok@kn\endcsname{\let\PY@bf=\textbf\def\PY@tc##1{\textcolor[rgb]{0.00,0.50,0.00}{##1}}}
\expandafter\def\csname PY@tok@mi\endcsname{\def\PY@tc##1{\textcolor[rgb]{0.40,0.40,0.40}{##1}}}
\expandafter\def\csname PY@tok@gp\endcsname{\let\PY@bf=\textbf\def\PY@tc##1{\textcolor[rgb]{0.00,0.00,0.50}{##1}}}
\expandafter\def\csname PY@tok@o\endcsname{\def\PY@tc##1{\textcolor[rgb]{0.40,0.40,0.40}{##1}}}
\expandafter\def\csname PY@tok@kr\endcsname{\let\PY@bf=\textbf\def\PY@tc##1{\textcolor[rgb]{0.00,0.50,0.00}{##1}}}
\expandafter\def\csname PY@tok@s\endcsname{\def\PY@tc##1{\textcolor[rgb]{0.73,0.13,0.13}{##1}}}
\expandafter\def\csname PY@tok@kp\endcsname{\def\PY@tc##1{\textcolor[rgb]{0.00,0.50,0.00}{##1}}}
\expandafter\def\csname PY@tok@w\endcsname{\def\PY@tc##1{\textcolor[rgb]{0.73,0.73,0.73}{##1}}}
\expandafter\def\csname PY@tok@kt\endcsname{\def\PY@tc##1{\textcolor[rgb]{0.69,0.00,0.25}{##1}}}
\expandafter\def\csname PY@tok@ow\endcsname{\let\PY@bf=\textbf\def\PY@tc##1{\textcolor[rgb]{0.67,0.13,1.00}{##1}}}
\expandafter\def\csname PY@tok@sb\endcsname{\def\PY@tc##1{\textcolor[rgb]{0.73,0.13,0.13}{##1}}}
\expandafter\def\csname PY@tok@k\endcsname{\let\PY@bf=\textbf\def\PY@tc##1{\textcolor[rgb]{0.00,0.50,0.00}{##1}}}
\expandafter\def\csname PY@tok@se\endcsname{\let\PY@bf=\textbf\def\PY@tc##1{\textcolor[rgb]{0.73,0.40,0.13}{##1}}}
\expandafter\def\csname PY@tok@sd\endcsname{\let\PY@it=\textit\def\PY@tc##1{\textcolor[rgb]{0.73,0.13,0.13}{##1}}}

\def\PYZbs{\char`\\}
\def\PYZus{\char`\_}
\def\PYZob{\char`\{}
\def\PYZcb{\char`\}}
\def\PYZca{\char`\^}
\def\PYZam{\char`\&}
\def\PYZlt{\char`\<}
\def\PYZgt{\char`\>}
\def\PYZsh{\char`\#}
\def\PYZpc{\char`\%}
\def\PYZdl{\char`\$}
\def\PYZhy{\char`\-}
\def\PYZsq{\char`\'}
\def\PYZdq{\char`\"}
\def\PYZti{\char`\~}
% for compatibility with earlier versions
\def\PYZat{@}
\def\PYZlb{[}
\def\PYZrb{]}
\makeatother


    % NB prompt colors
    \definecolor{nbframe-border}{rgb}{0.867,0.867,0.867}
    \definecolor{nbframe-bg}{rgb}{0.969,0.969,0.969}
    \definecolor{nbframe-in-prompt}{rgb}{0.0,0.0,0.502}
    \definecolor{nbframe-out-prompt}{rgb}{0.545,0.0,0.0}

    % NB prompt lengths
    \newlength{\inputpadding}
    \setlength{\inputpadding}{0.5em}
    \newlength{\cellleftmargin}
    \setlength{\cellleftmargin}{0.15\linewidth}
    \newlength{\borderthickness}
    \setlength{\borderthickness}{0.4pt}
    \newlength{\smallerfontscale}
    \setlength{\smallerfontscale}{9.5pt}

    % NB prompt font size
    \def\smaller{\fontsize{\smallerfontscale}{\smallerfontscale}\selectfont}

    % Define a background layer, in which the nb prompt shape is drawn
    \pgfdeclarelayer{background}
    \pgfsetlayers{background,main}
    \usetikzlibrary{calc}

    % define styles for the normal border and the torn border
    \tikzset{
      normal border/.style={draw=nbframe-border, fill=nbframe-bg,
        rectangle, rounded corners=2.5pt, line width=\borderthickness},
      torn border/.style={draw=white, fill=white, line width=\borderthickness}}

    % Macro to draw the shape behind the text, when it fits completly in the
    % page
    \def\notebookcellframe#1{%
    \tikz{%
      \node[inner sep=\inputpadding] (A) {#1};% Draw the text of the node
      \begin{pgfonlayer}{background}% Draw the shape behind
      \fill[normal border]%
            (A.south east) -- ($(A.south west)+(\cellleftmargin,0)$) -- 
            ($(A.north west)+(\cellleftmargin,0)$) -- (A.north east) -- cycle;
      \end{pgfonlayer}}}%

    % Macro to draw the shape, when the text will continue in next page
    \def\notebookcellframetop#1{%
    \tikz{%
      \node[inner sep=\inputpadding] (A) {#1};    % Draw the text of the node
      \begin{pgfonlayer}{background}    
      \fill[normal border]              % Draw the ``complete shape'' behind
            (A.south east) -- ($(A.south west)+(\cellleftmargin,0)$) -- 
            ($(A.north west)+(\cellleftmargin,0)$) -- (A.north east) -- cycle;
      \fill[torn border]                % Add the torn lower border
            ($(A.south east)-(0,.1)$) -- ($(A.south west)+(\cellleftmargin,-.1)$) -- 
            ($(A.south west)+(\cellleftmargin,.1)$) -- ($(A.south east)+(0,.1)$) -- cycle;
      \end{pgfonlayer}}}

    % Macro to draw the shape, when the text continues from previous page
    \def\notebookcellframebottom#1{%
    \tikz{%
      \node[inner sep=\inputpadding] (A) {#1};   % Draw the text of the node
      \begin{pgfonlayer}{background}   
      \fill[normal border]             % Draw the ``complete shape'' behind
            (A.south east) -- ($(A.south west)+(\cellleftmargin,0)$) -- 
            ($(A.north west)+(\cellleftmargin,0)$) -- (A.north east) -- cycle;
      \fill[torn border]               % Add the torn upper border
            ($(A.north east)-(0,.1)$) -- ($(A.north west)+(\cellleftmargin,-.1)$) -- 
            ($(A.north west)+(\cellleftmargin,.1)$) -- ($(A.north east)+(0,.1)$) -- cycle;
      \end{pgfonlayer}}}

    % Macro to draw the shape, when both the text continues from previous page
    % and it will continue in next page
    \def\notebookcellframemiddle#1{%
    \tikz{%
      \node[inner sep=\inputpadding] (A) {#1};   % Draw the text of the node
      \begin{pgfonlayer}{background}   
      \fill[normal border]             % Draw the ``complete shape'' behind
            (A.south east) -- ($(A.south west)+(\cellleftmargin,0)$) -- 
            ($(A.north west)+(\cellleftmargin,0)$) -- (A.north east) -- cycle;
      \fill[torn border]               % Add the torn lower border
            ($(A.south east)-(0,.1)$) -- ($(A.south west)+(\cellleftmargin,-.1)$) -- 
            ($(A.south west)+(\cellleftmargin,.1)$) -- ($(A.south east)+(0,.1)$) -- cycle;
      \fill[torn border]               % Add the torn upper border
            ($(A.north east)-(0,.1)$) -- ($(A.north west)+(\cellleftmargin,-.1)$) -- 
            ($(A.north west)+(\cellleftmargin,.1)$) -- ($(A.north east)+(0,.1)$) -- cycle;
      \end{pgfonlayer}}}

    % Define the environment which puts the frame
    % In this case, the environment also accepts an argument with an optional
    % title (which defaults to ``Example'', which is typeset in a box overlaid
    % on the top border
    \newenvironment{notebookcell}[1][0]{%
      \def\FrameCommand{\notebookcellframe}%
      \def\FirstFrameCommand{\notebookcellframetop}%
      \def\LastFrameCommand{\notebookcellframebottom}%
      \def\MidFrameCommand{\notebookcellframemiddle}%
      \par\vspace{1\baselineskip}%
      \MakeFramed {\FrameRestore}%
      \noindent\tikz\node[inner sep=0em] at ($(A.north west)-(0,0)$) {%
      \begin{minipage}{\cellleftmargin}%
    \hfill%
    {\smaller%
    \tt%
    \color{nbframe-in-prompt}%
    In[#1]:}%
    \hspace{\inputpadding}%
    \hspace{2pt}%
    \hspace{3pt}%
    \end{minipage}%%
      }; \par}%
    {\endMakeFramed}



    
    % Prevent overflowing lines due to hard-to-break entities
    \sloppy 
    % Setup hyperref package
    \hypersetup{
      breaklinks=true,  % so long urls are correctly broken across lines
      colorlinks=true,
      urlcolor=blue,
      linkcolor=darkorange,
      citecolor=darkgreen,
      }
    % Slightly bigger margins than the latex defaults
    
    \geometry{verbose,tmargin=1in,bmargin=1in,lmargin=1in,rmargin=1in}
    
    

    \begin{document}
    
    
    \maketitle
    
    

    
    \section{Astro518 Homework 3}\label{astro518-homework-3}

\subsection{Markov Chain Monte Carlo}\label{markov-chain-monte-carlo}

    % Add contents below.

{\par%
\vspace{-1\baselineskip}%
\needspace{4\baselineskip}}%
\begin{notebookcell}[1]%
\begin{addmargin}[\cellleftmargin]{0em}% left, right
{\smaller%
\par%
%
\vspace{-1\smallerfontscale}%
\begin{Verbatim}[commandchars=\\\{\}]
\PY{k+kn}{from} \PY{n+nn}{random} \PY{k+kn}{import} \PY{n}{randint}\PY{p}{,} \PY{n}{uniform}
\PY{k+kn}{import} \PY{n+nn}{matplotlib.pyplot} \PY{k+kn}{as} \PY{n+nn}{plt}
\end{Verbatim}
%
\par%
\vspace{-1\smallerfontscale}}%
\end{addmargin}
\end{notebookcell}


    % Add contents below.

{\par%
\vspace{-1\baselineskip}%
\needspace{4\baselineskip}}%
\begin{notebookcell}[2]%
\begin{addmargin}[\cellleftmargin]{0em}% left, right
{\smaller%
\par%
%
\vspace{-1\smallerfontscale}%
\begin{Verbatim}[commandchars=\\\{\}]
\PY{p}{(}\PY{n}{data\PYZus{}id}\PY{p}{,} \PY{n}{x}\PY{p}{,} \PY{n}{y}\PY{p}{,} \PY{n}{dy}\PY{p}{,} \PY{n}{dx}\PY{p}{,} \PY{n}{rho\PYZus{}xy}\PY{p}{)} \PY{o}{=} \PY{n}{np}\PY{o}{.}\PY{n}{loadtxt}\PY{p}{(}\PY{l+s}{\PYZsq{}}\PY{l+s}{sample\PYZus{}data.txt}\PY{l+s}{\PYZsq{}}\PY{p}{,} \PY{n}{comments} \PY{o}{=} \PY{l+s}{\PYZsq{}}\PY{l+s}{\PYZpc{}}\PY{l+s}{\PYZsq{}}\PY{p}{,} \PY{n}{unpack} \PY{o}{=} \PY{n+nb+bp}{True}\PY{p}{)} \PY{c}{\PYZsh{}\PYZsh{} readin data}
\end{Verbatim}
%
\par%
\vspace{-1\smallerfontscale}}%
\end{addmargin}
\end{notebookcell}


    \subsection{Write the code and Run}\label{write-the-code-and-run}

\texttt{ML\_MCMC.py} program calls for 7 command parameters, which are
intial values for: 1. m 2. b 3. Pb 4. Vb 5. Yb and the step length and
step number. The `base length' for every step is coded in the script,
which are 0.01 for m, 10 for b, 0.1 for Pb, 400 for Vb and 70 for Yb.
The real step in every Monte Carlo trial is the command line step
multiplying the baselength.

For the first run, the m and b are set to values which make sure they
are far away from the convergent values to see how many step needed for
burning in. Here m0 equals to 0 and b0 equals to 200.

    % Add contents below.

{\par%
\vspace{-1\baselineskip}%
\needspace{4\baselineskip}}%
\begin{notebookcell}[150]%
\begin{addmargin}[\cellleftmargin]{0em}% left, right
{\smaller%
\par%
%
\vspace{-1\smallerfontscale}%
\begin{Verbatim}[commandchars=\\\{\}]
\PY{o}{\PYZpc{}\PYZpc{}}\PY{k}{time}
\PY{o}{\PYZpc{}}\PY{k}{run} \PY{o}{\PYZhy{}}\PY{n}{i} \PY{n}{ML\PYZus{}MCMC}\PY{o}{.}\PY{n}{py} \PY{l+m+mi}{0} \PY{l+m+mi}{200} \PY{l+m+mf}{0.5} \PY{l+m+mi}{2000} \PY{l+m+mi}{300} \PY{l+m+mf}{0.5} \PY{l+m+mi}{1000000} 
\PY{n}{np}\PY{o}{.}\PY{n}{savetxt}\PY{p}{(}\PY{l+s}{\PYZdq{}}\PY{l+s}{MCMC\PYZus{}First\PYZus{}run.dat}\PY{l+s}{\PYZdq{}}\PY{p}{,} \PY{n+nb}{zip}\PY{p}{(}\PY{n}{m\PYZus{}list}\PY{p}{,} \PY{n}{b\PYZus{}list}\PY{p}{,} \PY{n}{Pb\PYZus{}list}\PY{p}{,} \PY{n}{Vb\PYZus{}list}\PY{p}{,} \PY{n}{Yb\PYZus{}list}\PY{p}{)}\PY{p}{,} \PY{n}{fmt} \PY{o}{=} \PY{l+s}{\PYZsq{}}\PY{l+s+si}{\PYZpc{}.8f}\PY{l+s}{\PYZsq{}}\PY{p}{)}
\end{Verbatim}
%
\par%
\vspace{-1\smallerfontscale}}%
\end{addmargin}
\end{notebookcell}

\par\vspace{1\smallerfontscale}%
    \needspace{4\baselineskip}%
    % Only render the prompt if the cell is pyout.  Note, the outputs prompt 
    % block isn't used since we need to check each indiviual output and only
    % add prompts to the pyout ones.
    %
    %
    \begin{addmargin}[\cellleftmargin]{0em}% left, right
    {\smaller%
    \vspace{-1\smallerfontscale}%
    
    \begin{Verbatim}[commandchars=\\\{\}]
1000000
CPU times: user 1min 10s, sys: 265 ms, total: 1min 11s
Wall time: 1min 10s
    \end{Verbatim}
}%
    \end{addmargin}%
    % Add contents below.

{\par%
\vspace{-1\baselineskip}%
\needspace{4\baselineskip}}%
\begin{notebookcell}[155]%
\begin{addmargin}[\cellleftmargin]{0em}% left, right
{\smaller%
\par%
%
\vspace{-1\smallerfontscale}%
\begin{Verbatim}[commandchars=\\\{\}]
\PY{c}{\PYZsh{}\PYZsh{} How many points are needed to converge}
\PY{n}{fig} \PY{o}{=} \PY{n}{plt}\PY{o}{.}\PY{n}{figure}\PY{p}{(}\PY{p}{)}
\PY{n}{m\PYZus{}ax} \PY{o}{=} \PY{n}{fig}\PY{o}{.}\PY{n}{add\PYZus{}subplot}\PY{p}{(}\PY{l+m+mi}{211}\PY{p}{)}
\PY{n}{m\PYZus{}ax}\PY{o}{.}\PY{n}{plot}\PY{p}{(}\PY{n}{m\PYZus{}list}\PY{p}{)}
\PY{n}{m\PYZus{}ax}\PY{o}{.}\PY{n}{set\PYZus{}xlabel}\PY{p}{(}\PY{l+s}{\PYZsq{}}\PY{l+s}{Number of Step}\PY{l+s}{\PYZsq{}}\PY{p}{)}
\PY{n}{m\PYZus{}ax}\PY{o}{.}\PY{n}{set\PYZus{}ylabel}\PY{p}{(}\PY{l+s}{\PYZsq{}}\PY{l+s}{\PYZdl{}m\PYZdl{}}\PY{l+s}{\PYZsq{}}\PY{p}{)}

\PY{n}{b\PYZus{}ax} \PY{o}{=} \PY{n}{fig}\PY{o}{.}\PY{n}{add\PYZus{}subplot}\PY{p}{(}\PY{l+m+mi}{212}\PY{p}{)}
\PY{n}{b\PYZus{}ax}\PY{o}{.}\PY{n}{plot}\PY{p}{(}\PY{n}{b\PYZus{}list}\PY{p}{)}
\PY{n}{b\PYZus{}ax}\PY{o}{.}\PY{n}{set\PYZus{}xlabel}\PY{p}{(}\PY{l+s}{\PYZsq{}}\PY{l+s}{Number of Step}\PY{l+s}{\PYZsq{}}\PY{p}{)}
\PY{n}{b\PYZus{}ax}\PY{o}{.}\PY{n}{set\PYZus{}ylabel}\PY{p}{(}\PY{l+s}{\PYZsq{}}\PY{l+s}{\PYZdl{}b\PYZdl{}}\PY{l+s}{\PYZsq{}}\PY{p}{)}

\PY{n}{fig}\PY{o}{.}\PY{n}{tight\PYZus{}layout}\PY{p}{(}\PY{p}{)}
\end{Verbatim}
%
\par%
\vspace{-1\smallerfontscale}}%
\end{addmargin}
\end{notebookcell}

\par\vspace{1\smallerfontscale}%
    \needspace{4\baselineskip}%
    % Only render the prompt if the cell is pyout.  Note, the outputs prompt 
    % block isn't used since we need to check each indiviual output and only
    % add prompts to the pyout ones.
    %
    %
    \begin{addmargin}[\cellleftmargin]{0em}% left, right
    {\smaller%
    \vspace{-1\smallerfontscale}%
    
    \begin{center}
    \adjustimage{max size={0.9\linewidth}{0.9\paperheight}}{Astro518-HW3_files/Astro518-HW3_5_0.png}
    \end{center}
    { \hspace*{\fill} \\}
    }%
    \end{addmargin}%
    % Add contents below.

{\par%
\vspace{-1\baselineskip}%
\needspace{4\baselineskip}}%
\begin{notebookcell}[151]%
\begin{addmargin}[\cellleftmargin]{0em}% left, right
{\smaller%
\par%
%
\vspace{-1\smallerfontscale}%
\begin{Verbatim}[commandchars=\\\{\}]
\PY{o}{\PYZpc{}}\PY{k}{run} \PY{n}{plot\PYZus{}hist}\PY{o}{.}\PY{n}{py}
\end{Verbatim}
%
\par%
\vspace{-1\smallerfontscale}}%
\end{addmargin}
\end{notebookcell}


    % Add contents below.

{\par%
\vspace{-1\baselineskip}%
\needspace{4\baselineskip}}%
\begin{notebookcell}[172]%
\begin{addmargin}[\cellleftmargin]{0em}% left, right
{\smaller%
\par%
%
\vspace{-1\smallerfontscale}%
\begin{Verbatim}[commandchars=\\\{\}]
\PY{n}{m\PYZus{}fig}\PY{p}{,} \PY{n}{m\PYZus{}gaussian} \PY{o}{=} \PY{n}{plotHist}\PY{p}{(}\PY{n}{m\PYZus{}list}\PY{p}{[}\PY{l+m+mi}{80000}\PY{p}{:}\PY{p}{]}\PY{p}{,} \PY{p}{[}\PY{l+m+mi}{20000}\PY{p}{,} \PY{l+m+mf}{2.2}\PY{p}{,} \PY{l+m+mf}{0.5}\PY{p}{]}\PY{p}{,} \PY{l+m+mi}{50}\PY{p}{)}
\PY{n}{m\PYZus{}ax} \PY{o}{=} \PY{n}{m\PYZus{}fig}\PY{o}{.}\PY{n}{axes}\PY{p}{[}\PY{l+m+mi}{0}\PY{p}{]}
\PY{n}{m\PYZus{}ax}\PY{o}{.}\PY{n}{set\PYZus{}xlabel}\PY{p}{(}\PY{l+s}{\PYZsq{}}\PY{l+s}{\PYZdl{}m\PYZdl{}}\PY{l+s}{\PYZsq{}}\PY{p}{)}
\PY{n}{m\PYZus{}ax}\PY{o}{.}\PY{n}{set\PYZus{}ylabel}\PY{p}{(}\PY{l+s}{\PYZsq{}}\PY{l+s}{\PYZdl{}n\PYZdl{}}\PY{l+s}{\PYZsq{}}\PY{p}{)}
\PY{n}{mu\PYZus{}m} \PY{o}{=} \PY{n}{m\PYZus{}gaussian}\PY{p}{[}\PY{l+m+mi}{1}\PY{p}{]}
\PY{n}{sigma\PYZus{}m} \PY{o}{=} \PY{n}{m\PYZus{}gaussian}\PY{p}{[}\PY{l+m+mi}{2}\PY{p}{]}
\PY{n}{m\PYZus{}ax}\PY{o}{.}\PY{n}{text}\PY{p}{(}\PY{l+m+mf}{2.0}\PY{p}{,} \PY{l+m+mi}{7000}\PY{p}{,} \PY{l+s}{\PYZsq{}}\PY{l+s}{\PYZdl{}}\PY{l+s}{\PYZbs{}}\PY{l+s}{mu\PYZdl{} = \PYZob{}0:.2f\PYZcb{}}\PY{l+s+se}{\PYZbs{}n}\PY{l+s}{\PYZdl{}}\PY{l+s}{\PYZbs{}}\PY{l+s}{sigma\PYZdl{} = \PYZob{}1:.2f\PYZcb{}}\PY{l+s}{\PYZsq{}}\PY{o}{.}\PY{n}{format}\PY{p}{(}\PY{n}{mu\PYZus{}m}\PY{p}{,} \PY{n}{sigma\PYZus{}m}\PY{p}{)}\PY{p}{)}

\PY{n}{b\PYZus{}fig}\PY{p}{,} \PY{n}{b\PYZus{}gaussian} \PY{o}{=} \PY{n}{plotHist}\PY{p}{(}\PY{n}{b\PYZus{}list}\PY{p}{[}\PY{l+m+mi}{80000}\PY{p}{:}\PY{p}{]}\PY{p}{,} \PY{p}{[}\PY{l+m+mi}{20000}\PY{p}{,} \PY{l+m+mi}{30}\PY{p}{,} \PY{l+m+mi}{20}\PY{p}{]}\PY{p}{,} \PY{l+m+mi}{50}\PY{p}{)}
\PY{n}{b\PYZus{}ax} \PY{o}{=} \PY{n}{b\PYZus{}fig}\PY{o}{.}\PY{n}{axes}\PY{p}{[}\PY{l+m+mi}{0}\PY{p}{]}
\PY{n}{b\PYZus{}ax}\PY{o}{.}\PY{n}{set\PYZus{}xlabel}\PY{p}{(}\PY{l+s}{\PYZsq{}}\PY{l+s}{\PYZdl{}b\PYZdl{}}\PY{l+s}{\PYZsq{}}\PY{p}{)}
\PY{n}{b\PYZus{}ax}\PY{o}{.}\PY{n}{set\PYZus{}ylabel}\PY{p}{(}\PY{l+s}{\PYZsq{}}\PY{l+s}{\PYZdl{}n\PYZdl{}}\PY{l+s}{\PYZsq{}}\PY{p}{)}
\PY{n}{mu\PYZus{}b} \PY{o}{=} \PY{n}{b\PYZus{}gaussian}\PY{p}{[}\PY{l+m+mi}{1}\PY{p}{]}
\PY{n}{sigma\PYZus{}b} \PY{o}{=} \PY{n}{b\PYZus{}gaussian}\PY{p}{[}\PY{l+m+mi}{2}\PY{p}{]}
\PY{n}{b\PYZus{}ax}\PY{o}{.}\PY{n}{text}\PY{p}{(}\PY{o}{\PYZhy{}}\PY{l+m+mi}{20}\PY{p}{,} \PY{l+m+mi}{7000}\PY{p}{,} \PY{l+s}{\PYZsq{}}\PY{l+s}{\PYZdl{}}\PY{l+s}{\PYZbs{}}\PY{l+s}{mu\PYZdl{} = \PYZob{}0:.2f\PYZcb{}}\PY{l+s+se}{\PYZbs{}n}\PY{l+s}{\PYZdl{}}\PY{l+s}{\PYZbs{}}\PY{l+s}{sigma\PYZdl{} = \PYZob{}1:.2f\PYZcb{}}\PY{l+s}{\PYZsq{}}\PY{o}{.}\PY{n}{format}\PY{p}{(}\PY{n}{mu\PYZus{}b}\PY{p}{,} \PY{n}{sigma\PYZus{}b}\PY{p}{)}\PY{p}{)}

\PY{c}{\PYZsh{}\PYZsh{} a 2D histogram}
\PY{n}{fig} \PY{o}{=} \PY{n}{plt}\PY{o}{.}\PY{n}{figure}\PY{p}{(}\PY{p}{)}
\PY{n}{ax} \PY{o}{=} \PY{n}{fig}\PY{o}{.}\PY{n}{add\PYZus{}subplot}\PY{p}{(}\PY{l+m+mi}{111}\PY{p}{)}
\PY{n}{bars} \PY{o}{=} \PY{n}{ax}\PY{o}{.}\PY{n}{hist2d}\PY{p}{(}\PY{n}{m\PYZus{}list}\PY{p}{[}\PY{l+m+mi}{80000}\PY{p}{:}\PY{p}{]}\PY{p}{,} \PY{n}{b\PYZus{}list}\PY{p}{[}\PY{l+m+mi}{80000}\PY{p}{:}\PY{p}{]}\PY{p}{,} \PY{n}{bins} \PY{o}{=} \PY{l+m+mi}{50}\PY{p}{)}
\PY{n}{ax}\PY{o}{.}\PY{n}{set\PYZus{}xlabel}\PY{p}{(}\PY{l+s}{\PYZsq{}}\PY{l+s}{\PYZdl{}m\PYZdl{}}\PY{l+s}{\PYZsq{}}\PY{p}{)}
\PY{n}{ax}\PY{o}{.}\PY{n}{set\PYZus{}ylabel}\PY{p}{(}\PY{l+s}{\PYZsq{}}\PY{l+s}{\PYZdl{}b\PYZdl{}}\PY{l+s}{\PYZsq{}}\PY{p}{)}
\end{Verbatim}
%
\par%
\vspace{-1\smallerfontscale}}%
\end{addmargin}
\end{notebookcell}

\par\vspace{1\smallerfontscale}%
    \needspace{4\baselineskip}%
    % Only render the prompt if the cell is pyout.  Note, the outputs prompt 
    % block isn't used since we need to check each indiviual output and only
    % add prompts to the pyout ones.
    
        {\par%
        \vspace{-1\smallerfontscale}%
        \noindent%
        \begin{minipage}{\cellleftmargin}%
    \hfill%
    {\smaller%
    \tt%
    \color{nbframe-out-prompt}%
    Out[172]:}%
    \hspace{\inputpadding}%
    \hspace{0em}%
    \hspace{3pt}%
    \end{minipage}%%
        }%
    %
    %
    \begin{addmargin}[\cellleftmargin]{0em}% left, right
    {\smaller%
    \vspace{-1\smallerfontscale}%
    
    
    
    \begin{verbatim}
<matplotlib.text.Text at 0x1216f0190>
    \end{verbatim}

    
}%
    \end{addmargin}%\par\vspace{1\smallerfontscale}%
    \needspace{4\baselineskip}%
    % Only render the prompt if the cell is pyout.  Note, the outputs prompt 
    % block isn't used since we need to check each indiviual output and only
    % add prompts to the pyout ones.
    %
    %
    \begin{addmargin}[\cellleftmargin]{0em}% left, right
    {\smaller%
    \vspace{-1\smallerfontscale}%
    
    \begin{center}
    \adjustimage{max size={0.9\linewidth}{0.9\paperheight}}{Astro518-HW3_files/Astro518-HW3_7_1.png}
    \end{center}
    { \hspace*{\fill} \\}
    }%
    \end{addmargin}%\par\vspace{1\smallerfontscale}%
    \needspace{4\baselineskip}%
    % Only render the prompt if the cell is pyout.  Note, the outputs prompt 
    % block isn't used since we need to check each indiviual output and only
    % add prompts to the pyout ones.
    %
    %
    \begin{addmargin}[\cellleftmargin]{0em}% left, right
    {\smaller%
    \vspace{-1\smallerfontscale}%
    
    \begin{center}
    \adjustimage{max size={0.9\linewidth}{0.9\paperheight}}{Astro518-HW3_files/Astro518-HW3_7_2.png}
    \end{center}
    { \hspace*{\fill} \\}
    }%
    \end{addmargin}%\par\vspace{1\smallerfontscale}%
    \needspace{4\baselineskip}%
    % Only render the prompt if the cell is pyout.  Note, the outputs prompt 
    % block isn't used since we need to check each indiviual output and only
    % add prompts to the pyout ones.
    %
    %
    \begin{addmargin}[\cellleftmargin]{0em}% left, right
    {\smaller%
    \vspace{-1\smallerfontscale}%
    
    \begin{center}
    \adjustimage{max size={0.9\linewidth}{0.9\paperheight}}{Astro518-HW3_files/Astro518-HW3_7_3.png}
    \end{center}
    { \hspace*{\fill} \\}
    }%
    \end{addmargin}%
    % Add contents below.

{\par%
\vspace{-1\baselineskip}%
\needspace{4\baselineskip}}%
\begin{notebookcell}[171]%
\begin{addmargin}[\cellleftmargin]{0em}% left, right
{\smaller%
\par%
%
\vspace{-1\smallerfontscale}%
\begin{Verbatim}[commandchars=\\\{\}]
\PY{c}{\PYZsh{}\PYZsh{} Hist gram of Pb, Vb, Yb}
\PY{n}{fig} \PY{o}{=} \PY{n}{plt}\PY{o}{.}\PY{n}{figure}\PY{p}{(}\PY{n}{figsize} \PY{o}{=} \PY{p}{(}\PY{l+m+mi}{6}\PY{p}{,} \PY{l+m+mi}{9}\PY{p}{)}\PY{p}{)}
\PY{n}{ax\PYZus{}pb} \PY{o}{=} \PY{n}{fig}\PY{o}{.}\PY{n}{add\PYZus{}subplot}\PY{p}{(}\PY{l+m+mi}{311}\PY{p}{)}
\PY{n}{bars} \PY{o}{=} \PY{n}{ax\PYZus{}pb}\PY{o}{.}\PY{n}{hist}\PY{p}{(}\PY{n}{Pb\PYZus{}list}\PY{p}{[}\PY{l+m+mi}{80000}\PY{p}{:}\PY{p}{]}\PY{p}{,} \PY{n}{bins} \PY{o}{=} \PY{l+m+mi}{50}\PY{p}{)}
\PY{n}{ax\PYZus{}pb}\PY{o}{.}\PY{n}{set\PYZus{}xlabel}\PY{p}{(}\PY{l+s}{\PYZsq{}}\PY{l+s}{Pb}\PY{l+s}{\PYZsq{}}\PY{p}{)}
\PY{n}{ax\PYZus{}pb}\PY{o}{.}\PY{n}{set\PYZus{}ylabel}\PY{p}{(}\PY{l+s}{\PYZsq{}}\PY{l+s}{\PYZdl{}n\PYZdl{}}\PY{l+s}{\PYZsq{}}\PY{p}{)}

\PY{n}{ax\PYZus{}Vb} \PY{o}{=} \PY{n}{fig}\PY{o}{.}\PY{n}{add\PYZus{}subplot}\PY{p}{(}\PY{l+m+mi}{312}\PY{p}{)}
\PY{n}{bars} \PY{o}{=} \PY{n}{ax\PYZus{}Vb}\PY{o}{.}\PY{n}{hist}\PY{p}{(}\PY{n}{Vb\PYZus{}list}\PY{p}{[}\PY{l+m+mi}{80000}\PY{p}{:}\PY{p}{]}\PY{p}{,} \PY{n}{bins} \PY{o}{=} \PY{l+m+mi}{50}\PY{p}{)}
\PY{n}{ax\PYZus{}Vb}\PY{o}{.}\PY{n}{set\PYZus{}xlabel}\PY{p}{(}\PY{l+s}{\PYZsq{}}\PY{l+s}{Vb}\PY{l+s}{\PYZsq{}}\PY{p}{)}
\PY{n}{ax\PYZus{}Vb}\PY{o}{.}\PY{n}{set\PYZus{}ylabel}\PY{p}{(}\PY{l+s}{\PYZsq{}}\PY{l+s}{\PYZdl{}n\PYZdl{}}\PY{l+s}{\PYZsq{}}\PY{p}{)}
\PY{n}{ax\PYZus{}Yb} \PY{o}{=} \PY{n}{fig}\PY{o}{.}\PY{n}{add\PYZus{}subplot}\PY{p}{(}\PY{l+m+mi}{313}\PY{p}{)}
\PY{n}{bars} \PY{o}{=} \PY{n}{ax\PYZus{}Yb}\PY{o}{.}\PY{n}{hist}\PY{p}{(}\PY{n}{Yb\PYZus{}list}\PY{p}{[}\PY{l+m+mi}{80000}\PY{p}{:}\PY{p}{]}\PY{p}{,} \PY{n}{bins} \PY{o}{=} \PY{l+m+mi}{50}\PY{p}{)}
\PY{n}{ax\PYZus{}Yb}\PY{o}{.}\PY{n}{set\PYZus{}xlabel}\PY{p}{(}\PY{l+s}{\PYZsq{}}\PY{l+s}{Yb}\PY{l+s}{\PYZsq{}}\PY{p}{)}
\PY{n}{ax\PYZus{}Yb}\PY{o}{.}\PY{n}{set\PYZus{}ylabel}\PY{p}{(}\PY{l+s}{\PYZsq{}}\PY{l+s}{\PYZdl{}n\PYZdl{}}\PY{l+s}{\PYZsq{}}\PY{p}{)}
\PY{n}{fig}\PY{o}{.}\PY{n}{tight\PYZus{}layout}\PY{p}{(}\PY{p}{)}
\end{Verbatim}
%
\par%
\vspace{-1\smallerfontscale}}%
\end{addmargin}
\end{notebookcell}

\par\vspace{1\smallerfontscale}%
    \needspace{4\baselineskip}%
    % Only render the prompt if the cell is pyout.  Note, the outputs prompt 
    % block isn't used since we need to check each indiviual output and only
    % add prompts to the pyout ones.
    %
    %
    \begin{addmargin}[\cellleftmargin]{0em}% left, right
    {\smaller%
    \vspace{-1\smallerfontscale}%
    
    \begin{center}
    \adjustimage{max size={0.9\linewidth}{0.9\paperheight}}{Astro518-HW3_files/Astro518-HW3_8_0.png}
    \end{center}
    { \hspace*{\fill} \\}
    }%
    \end{addmargin}%
    % Add contents below.

{\par%
\vspace{-1\baselineskip}%
\needspace{4\baselineskip}}%
\begin{notebookcell}[176]%
\begin{addmargin}[\cellleftmargin]{0em}% left, right
{\smaller%
\par%
%
\vspace{-1\smallerfontscale}%
\begin{Verbatim}[commandchars=\\\{\}]
\PY{k}{def} \PY{n+nf}{confidInterval}\PY{p}{(}\PY{n}{data\PYZus{}list}\PY{p}{,} \PY{n}{interval\PYZus{}range} \PY{o}{=} \PY{l+m+mf}{0.95}\PY{p}{)}\PY{p}{:}
    \PY{l+s+sd}{\PYZdq{}\PYZdq{}\PYZdq{}}
\PY{l+s+sd}{    calculate interval limit with given interval range}
\PY{l+s+sd}{    \PYZdq{}\PYZdq{}\PYZdq{}}
    \PY{n}{n\PYZus{}data} \PY{o}{=} \PY{n+nb}{len}\PY{p}{(}\PY{n}{data\PYZus{}list}\PY{p}{)}
    \PY{n}{data\PYZus{}list}\PY{o}{.}\PY{n}{sort}\PY{p}{(}\PY{p}{)}
    \PY{n}{lower\PYZus{}id} \PY{o}{=} \PY{n+nb}{int}\PY{p}{(}\PY{n}{n\PYZus{}data} \PY{o}{*} \PY{p}{(}\PY{l+m+mi}{1} \PY{o}{\PYZhy{}} \PY{n}{interval\PYZus{}range}\PY{p}{)}\PY{o}{/}\PY{l+m+mi}{2}\PY{p}{)}
    \PY{n}{upper\PYZus{}id} \PY{o}{=} \PY{n}{n\PYZus{}data} \PY{o}{\PYZhy{}} \PY{n}{lower\PYZus{}id}
    \PY{k}{return} \PY{n}{data\PYZus{}list}\PY{p}{[}\PY{n}{lower\PYZus{}id}\PY{p}{]}\PY{p}{,} \PY{n}{data\PYZus{}list}\PY{p}{[}\PY{n}{upper\PYZus{}id}\PY{p}{]}

\PY{c}{\PYZsh{}\PYZsh{} for m}
\PY{n}{m\PYZus{}low}\PY{p}{,} \PY{n}{m\PYZus{}high} \PY{o}{=} \PY{n}{confidInterval}\PY{p}{(}\PY{n}{m\PYZus{}list}\PY{p}{[}\PY{l+m+mi}{80000}\PY{p}{:}\PY{p}{]}\PY{p}{)}
\PY{k}{print} \PY{l+s}{\PYZsq{}}\PY{l+s}{for m}\PY{l+s}{\PYZsq{}}
\PY{k}{print} \PY{l+s}{\PYZsq{}}\PY{l+s}{95}\PY{l+s+si}{\PYZpc{} c}\PY{l+s}{onfident interval is \PYZob{}0:.3f\PYZcb{} to \PYZob{}1:.3f\PYZcb{}}\PY{l+s}{\PYZsq{}}\PY{o}{.}\PY{n}{format}\PY{p}{(}\PY{n}{m\PYZus{}low}\PY{p}{,} \PY{n}{m\PYZus{}high}\PY{p}{)}
\PY{k}{print} \PY{l+s}{\PYZsq{}}\PY{l+s}{while 2 sigma interval is \PYZob{}0:.3f\PYZcb{} to \PYZob{}1:.3f\PYZcb{}}\PY{l+s+se}{\PYZbs{}n}\PY{l+s}{\PYZsq{}}\PY{o}{.}\PY{n}{format}\PY{p}{(}\PY{n}{mu\PYZus{}m}\PY{o}{\PYZhy{}} \PY{l+m+mi}{2} \PY{o}{*} \PY{n}{sigma\PYZus{}m}\PY{p}{,} \PY{n}{mu\PYZus{}m} \PY{o}{+} \PY{l+m+mi}{2} \PY{o}{*} \PY{n}{sigma\PYZus{}m}\PY{p}{)}
\PY{n}{b\PYZus{}low}\PY{p}{,} \PY{n}{b\PYZus{}high} \PY{o}{=} \PY{n}{confidInterval}\PY{p}{(}\PY{n}{b\PYZus{}list}\PY{p}{[}\PY{l+m+mi}{80000}\PY{p}{:}\PY{p}{]}\PY{p}{)}
\PY{k}{print} \PY{l+s}{\PYZsq{}}\PY{l+s}{for b}\PY{l+s}{\PYZsq{}}
\PY{k}{print} \PY{l+s}{\PYZsq{}}\PY{l+s}{95}\PY{l+s+si}{\PYZpc{} c}\PY{l+s}{onfident interval is \PYZob{}0:.3f\PYZcb{} to \PYZob{}1:.3f\PYZcb{}}\PY{l+s}{\PYZsq{}}\PY{o}{.}\PY{n}{format}\PY{p}{(}\PY{n}{b\PYZus{}low}\PY{p}{,} \PY{n}{b\PYZus{}high}\PY{p}{)}
\PY{k}{print} \PY{l+s}{\PYZsq{}}\PY{l+s}{while 2 sigma interval is \PYZob{}0:.3f\PYZcb{} to \PYZob{}1:.3f\PYZcb{}}\PY{l+s+se}{\PYZbs{}n}\PY{l+s}{\PYZsq{}}\PY{o}{.}\PY{n}{format}\PY{p}{(}\PY{n}{mu\PYZus{}b}\PY{o}{\PYZhy{}} \PY{l+m+mi}{2} \PY{o}{*} \PY{n}{sigma\PYZus{}b}\PY{p}{,} \PY{n}{mu\PYZus{}b} \PY{o}{+} \PY{l+m+mi}{2} \PY{o}{*} \PY{n}{sigma\PYZus{}b}\PY{p}{)}
\end{Verbatim}
%
\par%
\vspace{-1\smallerfontscale}}%
\end{addmargin}
\end{notebookcell}

\par\vspace{1\smallerfontscale}%
    \needspace{4\baselineskip}%
    % Only render the prompt if the cell is pyout.  Note, the outputs prompt 
    % block isn't used since we need to check each indiviual output and only
    % add prompts to the pyout ones.
    %
    %
    \begin{addmargin}[\cellleftmargin]{0em}% left, right
    {\smaller%
    \vspace{-1\smallerfontscale}%
    
    \begin{Verbatim}[commandchars=\\\{\}]
for m
95\% confident interval is 2.055 to 2.490
while 2 sigma interval is 2.048 to 2.441

for b
95\% confident interval is -6.508 to 66.249
while 2 sigma interval is 0.093 to 67.898
    \end{Verbatim}
}%
    \end{addmargin}%
    \begin{itemize}
\itemsep1pt\parskip0pt\parsep0pt
\item
  The advantage of using 95\% trust interval is that we do not need the
  distribution to be gaussian. For
\end{itemize}

    \subsection{Malmquist Bias}\label{malmquist-bias}

    \subsubsection{Generate supernovae and calculate mean
distance}\label{generate-supernovae-and-calculate-mean-distance}

\begin{itemize}
\itemsep1pt\parskip0pt\parsep0pt
\item
  To simplify the script, the following function uniformly generate
  supernovae in a box, and remove supernovae who have larger distance to
  the center than 2000 Mpc
\item
  Analytically

  \begin{equation}
  \begin{split}
  \bar{d} & = \frac{\int_0^{d_0} r\cdot n\cdot r^2 \sin\theta drd\theta d\phi}{\int_0^{d_0}n\cdot r^2 \sin\theta drd\theta d\phi}\\
  & = \frac{\int_0^{d_0} r^3dr}{\int_0^{d_0} r^2dr}\\
  & = \frac{3d_0}{4}
  \end{split}
  \end{equation}
\end{itemize}

    % Add contents below.

{\par%
\vspace{-1\baselineskip}%
\needspace{4\baselineskip}}%
\begin{notebookcell}[1]%
\begin{addmargin}[\cellleftmargin]{0em}% left, right
{\smaller%
\par%
%
\vspace{-1\smallerfontscale}%
\begin{Verbatim}[commandchars=\\\{\}]
\PY{k}{def} \PY{n+nf}{SNgenesis}\PY{p}{(}\PY{n}{N}\PY{p}{,} \PY{n}{r} \PY{o}{=} \PY{l+m+mi}{2000}\PY{p}{)}\PY{p}{:}
    \PY{l+s+sd}{\PYZdq{}\PYZdq{}\PYZdq{}}
\PY{l+s+sd}{    generate supernovae uniformly distributed within a given distance}
\PY{l+s+sd}{    \PYZdq{}\PYZdq{}\PYZdq{}}
    \PY{n}{N} \PY{o}{=} \PY{n}{np}\PY{o}{.}\PY{n}{int}\PY{p}{(}\PY{n}{N} \PY{o}{*} \PY{l+m+mf}{6.}\PY{o}{/}\PY{n}{np}\PY{o}{.}\PY{n}{pi}\PY{p}{)} \PY{c}{\PYZsh{}generate more SN because we have to drop some}
    \PY{n}{x} \PY{o}{=} \PY{n}{np}\PY{o}{.}\PY{n}{random}\PY{o}{.}\PY{n}{uniform}\PY{p}{(}\PY{o}{\PYZhy{}}\PY{n}{r}\PY{p}{,} \PY{n}{r}\PY{p}{,} \PY{n}{size} \PY{o}{=} \PY{n}{N}\PY{p}{)}
    \PY{n}{y} \PY{o}{=} \PY{n}{np}\PY{o}{.}\PY{n}{random}\PY{o}{.}\PY{n}{uniform}\PY{p}{(}\PY{o}{\PYZhy{}}\PY{n}{r}\PY{p}{,} \PY{n}{r}\PY{p}{,} \PY{n}{size} \PY{o}{=} \PY{n}{N}\PY{p}{)}
    \PY{n}{z} \PY{o}{=} \PY{n}{np}\PY{o}{.}\PY{n}{random}\PY{o}{.}\PY{n}{uniform}\PY{p}{(}\PY{o}{\PYZhy{}}\PY{n}{r}\PY{p}{,} \PY{n}{r}\PY{p}{,} \PY{n}{size} \PY{o}{=} \PY{n}{N}\PY{p}{)}
    \PY{n}{dis} \PY{o}{=} \PY{n}{np}\PY{o}{.}\PY{n}{sqrt}\PY{p}{(}\PY{n}{x}\PY{o}{*}\PY{o}{*}\PY{l+m+mi}{2} \PY{o}{+} \PY{n}{y}\PY{o}{*}\PY{o}{*}\PY{l+m+mi}{2} \PY{o}{+} \PY{n}{z}\PY{o}{*}\PY{o}{*}\PY{l+m+mi}{2}\PY{p}{)}
    \PY{k}{return} \PY{n}{dis}\PY{p}{[}\PY{p}{[}\PY{n}{dis} \PY{o}{\PYZlt{}} \PY{l+m+mi}{2000}\PY{p}{]}\PY{p}{]}

\PY{n}{sn\PYZus{}dis} \PY{o}{=} \PY{n}{SNgenesis}\PY{p}{(}\PY{l+m+mi}{100000}\PY{p}{)}
\PY{k}{print} \PY{l+s}{\PYZsq{}}\PY{l+s}{number of SN generated:}\PY{l+s}{\PYZsq{}}\PY{p}{,} \PY{n+nb}{len}\PY{p}{(}\PY{n}{sn\PYZus{}dis}\PY{p}{)}
\PY{k}{print} \PY{l+s}{\PYZsq{}}\PY{l+s}{mean distance: \PYZob{}0:.2f\PYZcb{}}\PY{l+s}{\PYZsq{}}\PY{o}{.}\PY{n}{format}\PY{p}{(}\PY{n}{np}\PY{o}{.}\PY{n}{mean}\PY{p}{(}\PY{n}{sn\PYZus{}dis}\PY{p}{)}\PY{p}{)}
\end{Verbatim}
%
\par%
\vspace{-1\smallerfontscale}}%
\end{addmargin}
\end{notebookcell}

\par\vspace{1\smallerfontscale}%
    \needspace{4\baselineskip}%
    % Only render the prompt if the cell is pyout.  Note, the outputs prompt 
    % block isn't used since we need to check each indiviual output and only
    % add prompts to the pyout ones.
    %
    %
    \begin{addmargin}[\cellleftmargin]{0em}% left, right
    {\smaller%
    \vspace{-1\smallerfontscale}%
    
    \begin{Verbatim}[commandchars=\\\{\}]
number of SN generated: 100090
mean distance: 1499.94
    \end{Verbatim}
}%
    \end{addmargin}%
    \begin{itemize}
\itemsep1pt\parskip0pt\parsep0pt
\item
  distance calculated wiht Monte Carlo method agrees well with
  analytical result
\end{itemize}

    % Add contents below.

{\par%
\vspace{-1\baselineskip}%
\needspace{4\baselineskip}}%
\begin{notebookcell}[10]%
\begin{addmargin}[\cellleftmargin]{0em}% left, right
{\smaller%
\par%
%
\vspace{-1\smallerfontscale}%
\begin{Verbatim}[commandchars=\\\{\}]
\PY{k}{def} \PY{n+nf}{simObservation}\PY{p}{(}\PY{n}{samplesize} \PY{o}{=} \PY{l+m+mi}{10000}\PY{p}{)}\PY{p}{:}
    \PY{l+s+sd}{\PYZdq{}\PYZdq{}\PYZdq{}}
\PY{l+s+sd}{    carry out the simulated observation}
\PY{l+s+sd}{    return the apparent magnitude, and velocity that measured}
\PY{l+s+sd}{    \PYZdq{}\PYZdq{}\PYZdq{}}
    
    \PY{n}{dis} \PY{o}{=} \PY{n}{SNgenesis}\PY{p}{(}\PY{n}{samplesize}\PY{p}{)}
    \PY{n}{velocity} \PY{o}{=} \PY{n}{dis} \PY{o}{*} \PY{l+m+mi}{72} \PY{c}{\PYZsh{} Hubble\PYZsq{}s law}
    \PY{n}{Mag} \PY{o}{=} \PY{n}{np}\PY{o}{.}\PY{n}{random}\PY{o}{.}\PY{n}{normal}\PY{p}{(}\PY{l+m+mi}{0}\PY{p}{,} \PY{l+m+mi}{1}\PY{p}{,} \PY{n}{size} \PY{o}{=} \PY{n+nb}{len}\PY{p}{(}\PY{n}{dis}\PY{p}{)}\PY{p}{)} \PY{o}{\PYZhy{}} \PY{l+m+mi}{19}
    \PY{n}{mag} \PY{o}{=} \PY{n}{Mag} \PY{o}{+} \PY{l+m+mi}{5} \PY{o}{*} \PY{n}{np}\PY{o}{.}\PY{n}{log10}\PY{p}{(}\PY{n}{dis} \PY{o}{*} \PY{l+m+mf}{1e6}\PY{o}{/}\PY{l+m+mi}{10}\PY{p}{)}

    \PY{k}{return} \PY{n}{mag}\PY{p}{[}\PY{n}{mag} \PY{o}{\PYZlt{}} \PY{l+m+mi}{20}\PY{p}{]}\PY{p}{,} \PY{n}{velocity}\PY{p}{[}\PY{n}{mag} \PY{o}{\PYZlt{}} \PY{l+m+mi}{20}\PY{p}{]} \PY{c}{\PYZsh{} only return supernovae that have apparent magnitudes less than 20}

\PY{n}{mag\PYZus{}ob}\PY{p}{,} \PY{n}{velocity\PYZus{}ob} \PY{o}{=} \PY{n}{simObservation}\PY{p}{(}\PY{l+m+mi}{10000}\PY{p}{)}
\PY{n}{dis\PYZus{}ob} \PY{o}{=} \PY{l+m+mi}{10} \PY{o}{*} \PY{l+m+mi}{10}\PY{o}{*}\PY{o}{*}\PY{p}{(}\PY{p}{(}\PY{n}{mag\PYZus{}ob} \PY{o}{\PYZhy{}} \PY{p}{(}\PY{o}{\PYZhy{}}\PY{l+m+mi}{19}\PY{p}{)}\PY{p}{)}\PY{o}{/}\PY{l+m+mf}{5.}\PY{p}{)} \PY{o}{/}\PY{l+m+mf}{1e6} \PY{c}{\PYZsh{}Mpc    }

\PY{k}{print} \PY{l+s}{\PYZsq{}}\PY{l+s}{Observed Hubble constant:}\PY{l+s}{\PYZsq{}}\PY{p}{,} \PY{n}{np}\PY{o}{.}\PY{n}{mean}\PY{p}{(}\PY{n}{velocity\PYZus{}ob}\PY{o}{/}\PY{n}{dis\PYZus{}ob}\PY{p}{)}
\end{Verbatim}
%
\par%
\vspace{-1\smallerfontscale}}%
\end{addmargin}
\end{notebookcell}

\par\vspace{1\smallerfontscale}%
    \needspace{4\baselineskip}%
    % Only render the prompt if the cell is pyout.  Note, the outputs prompt 
    % block isn't used since we need to check each indiviual output and only
    % add prompts to the pyout ones.
    %
    %
    \begin{addmargin}[\cellleftmargin]{0em}% left, right
    {\smaller%
    \vspace{-1\smallerfontscale}%
    
    \begin{Verbatim}[commandchars=\\\{\}]
Observed Hubble constant: 140.004811072
    \end{Verbatim}
}%
    \end{addmargin}%
    % Add contents below.

{\par%
\vspace{-1\baselineskip}%
\needspace{4\baselineskip}}%
\begin{notebookcell}[14]%
\begin{addmargin}[\cellleftmargin]{0em}% left, right
{\smaller%
\par%
%
\vspace{-1\smallerfontscale}%
\begin{Verbatim}[commandchars=\\\{\}]
\PY{n}{plt}\PY{o}{.}\PY{n}{plot}\PY{p}{(}\PY{n}{dis\PYZus{}ob}\PY{p}{,} \PY{n}{velocity\PYZus{}ob}\PY{p}{,} \PY{l+s}{\PYZsq{}}\PY{l+s}{o}\PY{l+s}{\PYZsq{}}\PY{p}{)}
\PY{n}{ax} \PY{o}{=} \PY{n}{plt}\PY{o}{.}\PY{n}{gca}\PY{p}{(}\PY{p}{)}
\PY{n}{ax}\PY{o}{.}\PY{n}{set\PYZus{}xlabel}\PY{p}{(}\PY{l+s}{\PYZsq{}}\PY{l+s}{Observed distance(pc)}\PY{l+s}{\PYZsq{}}\PY{p}{)}
\PY{n}{ax}\PY{o}{.}\PY{n}{set\PYZus{}ylabel}\PY{p}{(}\PY{l+s}{\PYZsq{}}\PY{l+s}{Observed velocity(\PYZdl{}}\PY{l+s}{\PYZbs{}}\PY{l+s}{mathrm\PYZob{}km}\PY{l+s}{\PYZbs{}}\PY{l+s}{cdot s\PYZca{}\PYZob{}\PYZhy{}1\PYZcb{}\PYZcb{}\PYZdl{})}\PY{l+s}{\PYZsq{}}\PY{p}{)}
\PY{n}{ax}\PY{o}{.}\PY{n}{text}\PY{p}{(}\PY{l+m+mi}{100}\PY{p}{,} \PY{l+m+mi}{140000}\PY{p}{,} \PY{l+s}{r\PYZdq{}}\PY{l+s}{\PYZdl{}H\PYZus{}0 = }\PY{l+s}{\PYZbs{}}\PY{l+s}{bar\PYZob{}}\PY{l+s}{\PYZbs{}}\PY{l+s}{left(}\PY{l+s}{\PYZbs{}}\PY{l+s}{frac\PYZob{}v}\PY{l+s}{\PYZsq{}}\PY{l+s}{\PYZcb{}\PYZob{}d}\PY{l+s}{\PYZsq{}}\PY{l+s}{\PYZcb{}}\PY{l+s}{\PYZbs{}}\PY{l+s}{right)\PYZcb{} = 140.0\PYZdl{}}\PY{l+s}{\PYZdq{}}\PY{p}{,}\PY{n}{fontsize} \PY{o}{=} \PY{l+m+mi}{16}\PY{p}{)}
\end{Verbatim}
%
\par%
\vspace{-1\smallerfontscale}}%
\end{addmargin}
\end{notebookcell}

\par\vspace{1\smallerfontscale}%
    \needspace{4\baselineskip}%
    % Only render the prompt if the cell is pyout.  Note, the outputs prompt 
    % block isn't used since we need to check each indiviual output and only
    % add prompts to the pyout ones.
    
        {\par%
        \vspace{-1\smallerfontscale}%
        \noindent%
        \begin{minipage}{\cellleftmargin}%
    \hfill%
    {\smaller%
    \tt%
    \color{nbframe-out-prompt}%
    Out[14]:}%
    \hspace{\inputpadding}%
    \hspace{0em}%
    \hspace{3pt}%
    \end{minipage}%%
        }%
    %
    %
    \begin{addmargin}[\cellleftmargin]{0em}% left, right
    {\smaller%
    \vspace{-1\smallerfontscale}%
    
    
    
    \begin{verbatim}
<matplotlib.text.Text at 0x114868c90>
    \end{verbatim}

    
}%
    \end{addmargin}%\par\vspace{1\smallerfontscale}%
    \needspace{4\baselineskip}%
    % Only render the prompt if the cell is pyout.  Note, the outputs prompt 
    % block isn't used since we need to check each indiviual output and only
    % add prompts to the pyout ones.
    %
    %
    \begin{addmargin}[\cellleftmargin]{0em}% left, right
    {\smaller%
    \vspace{-1\smallerfontscale}%
    
    \begin{center}
    \adjustimage{max size={0.9\linewidth}{0.9\paperheight}}{Astro518-HW3_files/Astro518-HW3_16_1.png}
    \end{center}
    { \hspace*{\fill} \\}
    }%
    \end{addmargin}%
    In order to eliminate bias, we can use Bayesian analysis. We can set a
prior distribution for the absolute magnitude of supernova, and
calculate the maximum likelihood directly using grid or using Markov
Chain Monte Carlo.

    % Add contents below.

{\par%
\vspace{-1\baselineskip}%
\needspace{4\baselineskip}}%
\begin{notebookcell}[]%
\begin{addmargin}[\cellleftmargin]{0em}% left, right
{\smaller%
\par%
%
\vspace{-1\smallerfontscale}%
\begin{Verbatim}[commandchars=\\\{\}]

\end{Verbatim}
%
\par%
\vspace{-1\smallerfontscale}}%
\end{addmargin}
\end{notebookcell}



    % Add a bibliography block to the postdoc
    
    
    
    \end{document}
