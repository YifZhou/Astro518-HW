\documentclass[preprint, 12pt]{aastex}
\synctex=1
\usepackage{graphicx}
\shorttitle{Instrumentation Term Paper}
\shortauthors{Yifan Zhou}
\slugcomment{Astro 518 Term Paper}
\doublespace

\begin{document}
\title{Instrumentation Term Paper}
\author{Yifan Zhou}
\affil{University of Arizona}
\email{yifzhou@email.arizona.edu}

\section{Introduction}
Infrared Array Camera is the near infrared imaging instrument equipped
on Spitzer Space Telescope. NIRCam is the imaging camera on James Webb
Space Telescope (JWST). I listed the general design and performance
comparison of these two instrument to illustrate the big leap of
NIRCam comparing to IRAC.

Although exoplanet observation is not the key factor in defining
the IRAC desgin, astronomers made great progress in detecting and
characterizing exoplanet with the help of IRAC. I listed several key
discoveries in exoplanet with IRAC to demonstrate what aspect of IRAC
is made use of in exoplanet studies. With the comparison of optical
performance, we can see the potential of NIRCam in exoplanet
observations. 
%\section{Scientific Requirements}

\section{Instrumentation Design}
\subsection{Optical Design}

The optical design of Spitzer IRAC and JWST NIRCam are demonstrated in
figure \ref{fig:optics}.\par

\subsubsection{IRAC}
The optics path of IRAC is relatively simple comparing to that of
NIRCam. Two pickoff mirrors that are slightly displaced and tilted
separate the light into two beams. Each reflected beam then passes
doublet lenses to be reimaged at the focal plane. After the doublet
lenses, two Ge substrate beam splitters separate the lower beam into
Channel 1 (reflected) and Channel 3 (transmitted) and the upper beam
into Channel 2 (reflected) and Channel 4 (transmitted). Finally 4
Channels of light are received by 4 detectors accordingly. This design
enables IRAC to take image in 4 different bands simultaneously.\par

\subsubsection{NIRCam}
NIRCam consists of two modules, each imaging a $2.16' \times 2.16'$
field of view.  The modules are built on two optical benches mounted
back-to-back. As a result, the two fields of views imaged by the
modules are adjacent.  The modules are functionally identical, with
identical optical and focal plane components. They are mirror images
of each other except for coronagraphic pupil masks.

The optical design of each module is shown in figure
\ref{fig:optics}. There are more devices on the optics path of NIRCam to enable
different observation mode. After reflection of pickoff mirror and
first fold mirror, the light are then split by a dichroic
beam-splitter into long wave beam and short wave beam. Before the two
beams reach the detector, two filter wheels are equipped on the paths
so that multi-band imaging can be accomplished. After the pick-off
mirror, there is a coronagraph elements to provide NIRCam the ability
to make coronagraphic measurements.

Several optical elements are added
or specially designed for optical accuracy improvement. The pickoff
mirror is actuated by a three degree-of- freedom focus and alignment
mechanism that provides fine positioning in tip, tilt, and piston,
allowing NIRCam to accommodate small pointing and focus changes that
may manifest themselves once the Observatory is on-orbit. Internal
calibration sources including a flat field source and a coranographic
source are used to aid with on-orbit calibration and characterization. 


\begin{figure}
  \centering
  \plotone{IRAC-optics}
  \plotone{NIRCam-optics}
  \caption{Optical Design of Spitzer IRAC (above) and JWST NIRCam (below).}
  \label{fig:optics}
\end{figure}

\subsection{Focal Plane Array}

\subsubsection{IRAC}
IRAC has 4 detector arrays to recieve photons in 4 channels. The 4
arrays are all 256$\times$256 pixels in size and have the same
physical pixel size of 30 $\mu$m. The channel 1 and channel 2 arrays
are made of InSb while the longer wavelength channels 3 and 4 are
Si:As detectors.


\subsubsection{NIRCam}

NIRCam is based on 2048$\times$2048 HgCdTe photodiode chips, with
physical pixel size of 18 $\mu$m $\times$ 18 $\mu$m pixels. The chips
used for the short and long wavelength channels are based on different
types of HgCdTe. However, in what concerns their functionality and
control they are basically identical.  The sensitive area of a chip is
2040x2040 pixels, due to a 4 pixel wide border of reference pixels
along all edges. Reference pixels are used for tracking bias drifts
during an exposure.  The finer pixel scale of the short-wavelength
channels requires a Focal Plane Array, i.e. a 2x2 mosaic of chips
butted with a small gap (5”) between them. The resulting gaps in the
images have to be filled by dithering moves of the telescope. The long
wavelength channels use a single chip.  The fields observed by the two
modules are adjacent with about 50" separation. Also this larger gap
in the images has to be filled by dithering moves.


\subsection{Operation Mode}
\subsubsection{IRAC}

The four 256$\times$256 Focal Plane Arrays can be read out with different
operation mode.They are full-array readout mode, stellar photometry
mode and subarray mode.\par

In full-array readout mode, there were four selectable frame times: 2,
12, 30, and 100 s. To allow sensitive observations without losing dynamic
range, there was a high dynamic range (HDR) option. 

Stellar photometry mode was available for observations of objects much
brighter in channels 1 and 2 than in 3 and 4 (typically stars). This
mode took short exposures in channels 1 and 2, and long exposures in
channels 3 and 4.  The sensitivities of each frame are identical to
those in full array mode.

For very bright sources, a subarray mode was available. In this mode,
only a small $32 \times 32$ pixel portion of the array was read out,
so the field of view was only $38' \times 38'$. Mapping was not
allowed in subarray mode.  In subarray readout mode, there were three
selectable frame times: 0.02, 0.1, and 0.4 sec. This mode allows
observer to catch fast variability of the targets.


\subsubsection{NIRCam}
Comparing to IRAC, NIRCam equipped with much more complicated devices,
which allows different observation to be made with.

The primary purpose of the NIRCam instrument is to take image
measurement. For imaging observation, the major difference from IRAC
is that both short wave and long wave channels of NIRCam are furnished
with a Filter Wheel that allows for observations through more than 12
different spectral bands. There are totally 30 filters on NIRCam,
whose wavelength covers from 0.7 to 4.8 $\mu$m. 

Another unique feature of NIRCam is the ability to make Coronagraphic
measurements. In coronagraphic mode, a coronagraphic occulting mask
that can be chosen from a selection of $\mathrm{sinc}^{2}$, top-hat, and Gaussian
designs, optimized for light at either 2.0 or 4.6 $\mu m$ block the
light from the bright sources. This mode is essential in high contrast
imaging observation, e.g. direct imaging of exoplanets.

NIRCam is also available for spectroscopic observation. NIRCam has a
grism in the long wavelength channel. The grism offers the capability to
carry out slitless spectroscopy in the wavelength range 2.4 to 5 $\mu$m,
with spectroscopic resolution R $\sim 2000$.


\section{Instrument Performance}
As a detector equipped on the next generation infrared space
telescope, NIRCam definitely surpasses IRAC in different ways. Here I
compare several key factors that most valued by exoplanet observation
between IRAC and NIRCam to illustrate the great potential of NIRCam of
expanding our horizon on exoplanet. There factors are spatial
resolution, spectral coverage, sensitivity, dynamic range and frame
rate.\par

\subsection{Spatial Resolution}
High spatial resolution is essential in direct imaging
observation. Limited by small primary mirror size of Spitzer Space
Telescope (0.85 m), it is not IRAC's strength to spatially resolved
objects. The FWHM of point spread function (PSF) is $1''.66, 1''.72,
1''.88 \mbox{and} 1''.92$ at 3.6, 4.5, 5.8 and 8.0 $\mu$m channels,
which is hugh for direct imaiging study. The large pixel scale
($~1''.2$) resulting in under sampled PSF makes the situation even
worse. As a result, IRAC is hardly ever used for imaging of wide
orbit planetary mass companions.\par

The 6.5 m sized primary mirror of JWST provide NIRCam with significant
spatial resolve ability. NIRCam's PSF has FWHM of $0''.063$, and is
nearly 20 times sharper than that of IRAC. The pixel scale that is
optimized at 2 $\mu$m ensured well sampled PSFs at that wavelength. At
smaller wavelength, the PSF would be a little undersampled. However
the sample rate is still better than IRAC.

\subsection{Wavelength Coverage}
IRAC covers four wavelength channel at 3.6, 4.5, 5.8 and 8.0 $\mu$m. The
wavelength coverage of NIRCam which ranges from 1 to 5 $\mu$m lacks
long wave length ability comparing to IRAC. However, more than 30
filters supplied with NIRCam would provide spectral energy
distribution with fine structures.

\subsection{Sensitivity}

Low thermal emission of exoplanets call for high sensitivity. Table
\ref{tab:irac-sen} and figure \ref{fig:nircam-sen} illustrate the
sensitivities of IRAC and NIRCam. In figure \ref{fig:nircam-sen},
sensitivities are expressed as the flux required to get 10 $\sigma$ signal
with a 10000 s exposure. Since $\sigma sim \sqrt{t}$, 10 $\sigma$
signal with 10000s exposure is generally equivalent to  1 $\sigma$
signal with 100s exposure. Comparing to IRAC, NIRCam is more than 30
times sensible with wide band filter and more than 10 times sensible
with narrow band filter.\par

\begin{deluxetable}{lrrrr}
  \tablewidth{0pt}
  \tablecaption{IRAC Sensitivity (1 $\sigma$, $\mu$Jy), cite{Fazio
      2004}\label{tab:irac-sen}}
  \tablehead{Frame Time (s)&$3.6\mu$m&$4.5\mu$m&$5.8\mu$m&$8.0\mu$m}
  \startdata
200&0.40&0.84&5.5&6.9 \\
100&0.60&1.2&8.0&9.8 \\
30&1.4&2.4&16&18 \\
12&3.3&4.8&27&29 \\
2&32&38&150&92 \\
0.6&180&210&630&250 \\
0.4&86&75&270&140 \\
0.510&470&910&420 \\
0.02&7700&7200&11000&4900\\
  \enddata
\end{deluxetable}


\section{Application in Exoplanet}

Initially, the most important in defining the IRAC design was the
study of the early universe (cite{Fazio 2014}). However, with the
advantage in infrared wavelength coverage and high sensitivity, IRAC
plays a great role in exoplanet observation. Although IRAC is limited
by its design, e.g. the large FWHM of PSF is an obstacle for IRAC to
carry out direct imaging of exoplanet, IRAC is the best devices for
infrared transiting photometry before the launch of JWST. \par

Exoplanets' low surface temperatures locate their thermal emission
peaks in infrared. Specially hot Jupiters whose effective
temperatures are about 1000 K, are most ideal targets to detect
thermal emission with. cite{Charbonneau 2005} took advantage of IRAC's
high photometry precision to first detect the secondary eclipse of
exoplanet TrES-1 which confirmed the thermal emission coming from the
planet. They observed the target two channels, (4.5 and 8.0 $\mu$m),
observed the field, and obtained 1518 full-array images in each of the
two bandpasses cite{Charbonneau}. The great accuracy of spitzer
photometry data made it possible for them to construct the light curve
presented in figure \ref{fig:irac-lc} and detect the secondary eclipse
dip.\par

\begin{figure}
  \centering
  \plotone{second_eclipse}
  \caption{Light curve of TrES observed by IRAC Channel 4 (8$\mu$m)
    cite{Charbonneau 2005}.}
  \label{fig:irac-lc}
\end{figure}

In subarray mode, the high read out speed of IRAC is utilized to
monitor the temporal variability of exoplanet. cite{Knutson 2007}
observed HD189733 that is an eclipse planetary system containing a hot
Jupiter and constructed a ’map’ of the distribution of temperatures by
modeling the variability of the light curve. They monitored HD 189733
continuously over a 33.1 hour period using the 8 $\mu$m channel of
IRAC in subarray mode with a cadence of 0.4 s. With the high cadence,
high precision time series data (\ref{fig:iraf-lc2}), they saw a
distinct rise in flux beginning shortly after the end of the transit
and continuing until a time just prior to the beginning of the
secondary eclipse, from which they created the map of exoplanet
temperatures.\par

\begin{figure}
  \centering
  \plotone{irac-lc2}
  \caption{Light curve of high cadence observation of HD189733 system}
  \label{fig:iraf-lc2}
\end{figure}

The combination of 4 channel photometries of IRAC could open a window
for the study of exoplanet atmosphere cite{Burrows 2014}. By comparing
multi-band eclipse/secondary eclipse photometry with model spectrum,
water and dust cloud feature can be traced cite{O'Donovan 2010}.


\section{Summary}

\end{document}

%%% Local Variables:
%%% mode: latex
%%% TeX-master: t
%%% End:

%  LocalWords:  NIRCam pickoff
